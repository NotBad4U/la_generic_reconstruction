\section{Alethe proof}
\label{sect:alethe}

The Alethe proof trace format \cite{alethespec} for SMT solvers comprises two parts: the trace language based on SMT-LIB and a collection of proof rules. Traces witness proofs of unsatisfiability of a set of constraints.
They are sequences $a_1 \dots a_m~t_1 \dots t_n$ where
the $a_i$ corresponds to the constraints of the original SMT problem being refuted, each $t_i$ is a clause inferred from previous elements of the sequence, and $t_n$ is $\bot$ (the empty clause).
In the following, we designate the SMT-LIB problem as the \emph{input problem}.

\begin{lstlisting}[language=SMT,label={lst:smtexampleinput}]
(set-logic QF_LIA)
(declare-const x Int)
(declare-const y Int)
(assert (= 0 y))
(assert (= x 2))
(assert (or (< (+ x y) 1) (< 3 x)))
(get-proof)
\end{lstlisting}

\begin{center}
\lightning
\end{center}

\begin{lstlisting}[language=SMT,caption={The following example is the proof for the unsatisfiability of $(x+y < 1) \lor (3<x), x = 2$ and $0 = y$.},label={lst:smtexampleproof}]
(assume a0 (or (< (+ x y) 1) (< 3 x)))
(assume a1 (= x 2))
(assume a2 (= 0 y))
(step t1 (cl (< (+ x y) 1) (< 3 x)) :rule or :premises (a0))
(step t2 (cl (not (< 3 x)) (not (= x 2))) :rule la_generic :args (1/1 1/1))
(step t3 (cl (not (< 3 x))) :rule resolution :premises (a1 t2))
(step t4 (cl (< (+ x y) 1)) :rule resolution :premises (t1 t3))
(step t5 (cl (not (< (+ x y) 1)) (not (= x 2)) (not (= 0 y))) :rule la_generic :args (1/1 -1/1 1/1))
(step t6 (cl) :rule resolution :premises (t5 t4 a1 a2))
\end{lstlisting}

We will use the input problem shown in the top part of \cref{lst:smtexampleinput} with its Alethe proof (found by cvc5) in the bottom part as a running example to provide an overview of Alethe concepts and to illustrate our reconstruction of linear arithmetic step in Lambdapi.

\subsection{Alethe Trace Format Overview}

An Alethe proof trace inherits the declarations of its input problem. All symbols (sorts, functions, assertions, etc.) declared or defined in the input problem remain declared or defined, respectively. Furthermore, the syntax for terms, sorts, and annotations uses the syntactic rules defined in SMT-LIB \cite[\S 3]{smtlib} and the SMT signature context defined in \cite[\S 5.1 and \S 5.2]{smtlib}.
In the following we will represent an Alethe step as

\smallskip

\renewcommand{\eqnhighlightshade}{35}

\begin{equation}
\label{eq:step}
\tag{\textcolor{purple}{1}}
\eqnmarkbox[indexClr]{node2}{i}. \quad \eqnmarkbox[blue]{node1}{\Gamma} ~\triangleright~ \eqnmarkbox[green]{node3}{l_1 \dots l_n} \quad (\eqnmarkbox[purple]{node4}{\mathcal{R}}~\eqnmarkbox[red]{node5}{p_1 \dots p_m})~\eqnmarkbox[orange]{node6}{[a_1 \dots a_r]}
\end{equation}

\vspace{0.3em}
\annotate[yshift=-0.5em]{below, left}{node2}{index}
\annotate[yshift=-0.5em]{below, right}{node1}{context}
\annotate[yshift=0.5em]{above, left}{node3}{clause}
\annotate[yshift=-0.5em]{below, right}{node4}{rule}
\annotate[yshift=-0.5em]{below, right}{node5}{premises}
\annotate[yshift=-0.5em]{below, right}{node6}{arguments}

\vspace{0.3em}

\medskip

A step %\cref{eq:step} 
consists of an index \colorbox{indexClr!30}{$i$} $\in \mathbb{I}$ where $\mathbb{I}$ is a countable infinite set of indices (e.g. \kw{a0}, \kw{t1}), and a clause of formulae \colorbox{green!30}{$l_1, \dots, l_n$} representing an $n$-ary disjunction. Steps that are not assumptions are justified by a proof rule \colorbox{purple!30}{$\mathcal{R}$} that depends on a possibly empty set of premises $\{\colorbox{red!30}{$p_1 \dots  p_m$}\} \subseteq \mathbb{I}$ that only references earlier steps such that the proof forms
a directed acyclic graph. A rule might also depend on a list of arguments \colorbox{orange!30}{$[a_1 \dots a_r]$} where each argument $a_i$ is either a term or a pair $(x_i, t_i)$ where $x_i$ is a variable and $t_i$ is a term. The interpretation of the arguments is rule specific. The context \colorbox{blue!30}{$\Gamma$} of a step is a list $c_1 \dots c_l $ where each element $c_j$ is either a variable or a variable-term tuple denoted $x_j \mapsto t_j$. Therefore, steps with a non-empty context contain variables $x_j$ that appear in \colorbox{green!30}{$l_i$} and will be substituted by $t_j$. Proof rules \colorbox{purple!30}{$\mathcal{R}$} include theory lemmas and \texttt{resolution}, which corresponds to hyper-resolution on ground first-order clauses. 


\begin{table}[]
    \centering
    \begin{tabular}{ll}
    Rule & Description \\ \hline
    la\_generic & Tautologous disjunction of linear inequalities. \\
    lia\_generic & Tautologous disjunction of linear integer inequalities. \\
    la\_disequality & $t_1 \approx t_2 \lor \neg (t_1 \geq t_2) \lor \neg (t_2 \geq t_1)$ \\
    la\_totality & $t_1 \geq t_2 \lor t_2 \geq t_1$ \\
    la\_tautology & A trivial linear tautology \\
    la\_mult\_pos & $t_1 > 0 \land (t_2 \bowtie t_3) \rightarrow t_1 * t_2 \bowtie t_1 * t_3$ and $\bowtie \in \{<, >, \geq, \leq, =\}$ \\
    la\_mult\_neg & $t_1 < 0 \land (t_2 \bowtie t_3) \rightarrow t_1 * t_2 \bowtie_{inv} t_1 * t_3$ \\
    la\_rw\_eq & $(t \approx u) \approx (t \geq u \land u \geq t)$ \\
    div\_simplify & Simplification of division. \\
    prod\_simplify & Simplification of products. \\
    unary\_minus\_simplify & Simplification of the unuary minus. \\
    minus\_simplify & Simplification of the substractions. \\
    sum\_simplify & Simplification of sums. \\
    comp\_simplify & Simplification of arithmetic comparisons. \\
    \end{tabular}
    \caption{Linear arithmetic rules in Alethe.}
    \label{table:linear-arith-rules}
\end{table}

We now have the key components to explain the guiding proof in the bottom part of \cref{lst:smtexampleproof}.
The proofs starts with \tt{assume} steps \tt{a0}, \tt{a1}, \tt{a2} that restate the assertions from the \textit{input problem} (\cref{lst:smtexampleproof}).
Step \tt{t1} transforms disjunction into clause by using the Alethe rule \tt{or}.
Steps \tt{t2} and \tt{t5} are tautologies introduced by the main rule \tt{la\_generic}
in Linear Real Arithmetic (LRA) logic and used also in LIA logic, where \colorbox{green!30}{$\neg l_1, \neg l_2,\dots, \neg l_n$} represent linear inequalities.
These logics use closed linear formulas over the \lstinline[language=SMT]{Real} signature and \lstinline[language=SMT]{Int} respectively.
The \lstinline[language=SMT]{Real} terms in \tt{LRA} logic are built over the Reals signature from SMT-LIB with free constant symbols, but containing only linear atoms; that is
atoms with no occurrences of the function symbols \lstinline[language=SMT]{*} and \lstinline[language=SMT]{/}, except in coefficient multiplications—specifically, terms of the form 
\lstinline[language=SMT]{c}, \lstinline[language=SMT]{(* c x)}, or \lstinline[language=SMT]{(* x c)}  where \lstinline[language=SMT]{x} is a free constant and  \lstinline[language=SMT]{c} is an integer or rational coefficient.
Similarly, the \lstinline[language=SMT]{Int} terms in \tt{LIA} logic are closed formulas built over the
Ints signature with free constant symbols, but whose terms are also all linear, such that there is no occurrences of the function symbols \lstinline[language=SMT]{*}, \lstinline[language=SMT]{/}, \lstinline[language=SMT]{div}, \lstinline[language=SMT]{mod}, and \lstinline[language=SMT]{abs}
, except terms with coefficients are also allowed, that is, terms of the form \lstinline[language=SMT]{c}, \lstinline[language=SMT]{(* c x)}, or \lstinline[language=SMT]{(* x c)}
where \lstinline[language=SMT]{x} is a free constant and \lstinline[language=SMT]{c} is a term of the form \lstinline[language=SMT]{n} or \lstinline[language=SMT]{(- n)} for some numeral \tt{n}.
A linear inequality is of term of the form

\begin{equation}
\sum_{i=0}^{n}c_i\times{}t_i + d_1\bowtie \sum_{i=n+1}^{m} c_i\times{}t_i + d_2
\label{eqn:inequality}
\end{equation}

where $\bowtie\;\in \{=, <, >, \leq, \geq\}$, where $m\geq n$, $c_i, d_1, d_2$ are either \lstinline[language=SMT]{Int} or \lstinline[language=SMT]{Real}
constants, and for each $i$ $c_i$ and $t_i$ have the same sort.
Checking the clause validity of \tt{t2} and \tt{t5} amounts to checking the unsatisfiability of the system of linear equations (we provide more details in \cref{ssect:la-in-alethe}) e.g. $x < 3$ and $x = 2$ in \tt{t2}.
A coefficient for each inequality are pass as arguments e.g. $(\frac{1}{1},\frac{1}{1})$ in \tt{t2}.
Steps \tt{t3} (and \tt{t4}) applies the \colorbox{purple!30}{\texttt{resolution}} rule to the premises \tt{a1}, \tt{t2} (respectively \tt{t1} \tt{t3}).
Finally, the step \texttt{t6} concludes the proof by generating the empty clause $\bot$, concretely denoted as \kw{(cl)} in \cref{lst:smtexampleproof}.
Notice that the contexts \colorbox{blue!30}{$\Gamma$} of each step are all empty in this proof.

\subsection{Linear arithmetic in Alethe}
\label{ssect:la-in-alethe}

Proofs for linear arithmetic steps use a number of straightforward rules listed in \cref{table:linear-arith-rules}, such as \tt{la\_totality}: $(t_1 \leq t_2 \lor t_2 \geq t_1)$.
Simplification rules \tt{*\_simplify}, such as \tt{sum\_simplify}, transform arithmetic formulas by applying equivalence-preserving operations repeatedly until a fixed point is reached;
these operations are no more complex than constant folding.

Following our method to encode Alethe described in \cite{ColtellacciMD24}, the linear arithmetic tautology rules \tt{la\_disequality}, \tt{la\_totality} and \tt{la\_mult\_*} are encoded as lemmas in our embedding of Alethe in Lambdapi.
The simplification rule \tt{comp\_simplify} is encoded as a lemma for each rewrite case and applied multiple times.
We do not support the remaining \tt{*\_simplify} rules and the \tt{la\_tautology} rule in this work, primarily because cvc5 does not follow the Alethe standard for simplification step.
Instead, it extends the Alethe format with the RARE simplification rules \cite{rare}. As a result, cvc5 did not generate any proofs using these standard rules for the SMT-LIB benchmarks.

A different approach is taken for the primary rules \tt{*\_generic},as they describe an algorithm.
While \tt{la\_generic rule} is primarily intended for LRA logic, it is also applied in LIA proofs when all variables in the (in)equalities are of integer sort.
A step of the rule \tt{la\_generic} represents a tautological clause of linear disequalities.  It can be checked by showing that the conjunction of
the negated disequalities is unsatisfiable. After the application of some strengthening rules, the resulting conjunction is unsatisfiable,
even if \lstinline[language=SMT]{Int} variables are assumed to be \lstinline[language=SMT]{Real} variables.
Although the rule may introduce rational coefficients, they often reduce to integers—as shown in \cref{lst:smtexampleproof}, where the coefficients are $(\frac{1}{1}, \frac{1}{1})$.
Cases where coefficients cannot be reduced to integers are rare in practice.
Let $\varphi_1,\dots, \varphi_n$ be linear inequalities and $a_1, \dots, a_n$ rational numbers, then a \tt{la\_generic} step has the general form

\[
\begin{matrix*}[c]
  i. & \ctxsep \quad & \varphi_1 , \dots , \varphi_n & \quad \tt{la\_generic}  & [a_1, \dots, a_n] \\
\end{matrix*}
\]

The constants $a_i$ are of sort \tt{Real} and must be printed using one of the productions $\grNT{rational}$ $\grNT{decimal}$, $\grNT{nonpositive\_decimal}$ described in \cref{app:alethe}.
To check the unsatisfiability of the negation of $\varphi_1, \dots, \varphi_n$ one performs the following steps for each literal. For each $i$, let $\varphi := \varphi_i$, $a := a_i$ and
we write $s1 \bowtie s2$ for \cref{eqn:inequality}.

\begin{enumerate}
    \item If $\varphi = s_1 > s_2$, then let $\varphi := s_1 \leq s_2$.
      If $\varphi = s_1 \geq s_2$, then let $\varphi := s_1 < s_2$.
      If $\varphi = s_1 < s_2$, then let $\varphi := s_1 \geq s_2$.
      If $\varphi = s_1 \leq s_2$, then let $\varphi := s_1 > s_2$. This negates
      the literal.
    
    \item If $\varphi = \neg (s_1 \bowtie s_2)$, then let $\varphi := s_1 \bowtie s_2$.
    
    \item If $\varphi = s_1 < s_2$, then let $\varphi :=   - s_1 > - s_2$.
      If $\varphi = s_1 \leq s_2$, then let $\varphi :=  s_1 \geq - s_2$.
      We want a canonical form that use only the operators $>, \geq$ and =.

    \item Replace $\varphi$ by $\sum_{i=0}^{n}c_i\times{}t_i - \sum_{i=n+1}^{m} c_i\times{}t_i
    \bowtie d$ where $d := d_2 - d_1$.
    
    \item \label{la_generic:str}Now $\varphi$ has the form $s_1 \bowtie d$. If all
    variables in $s_1$ are integer sorted: replace $\bowtie d$ according to
    the table below.

    \begin{center}
    \begin{tabular}{r|l|l}
        $\bowtie$  & If $d$ is an integer  & Otherwise \\
        \hline
        $>$        & $\geq d + 1$  & $\geq \lfloor d\rfloor + 1$  \\
        $\geq$     & $\geq d$      & $\geq \lfloor d\rfloor + 1$  \\
    \end{tabular}
    \end{center}

    \item If all variables of $\varphi$ are Int and coefficient $a_1 \dots a_n \in \mathbb{Q}$,
    then $a \coloneq a \times lcd(a_1 \dots a_n)$ where $lcd$ is the least common denominator of $[a_1 \dots a_n]$.
    
    \item If $\bowtie$ is $=$, then replace $\varphi$ by
    $\sum_{i=0}^{m}a\times{}c_i\times{}t_i = a\times{}d$, otherwise replace it by
    $\sum_{i=0}^{m}|a|\times{}c_i\times{}t_i = |a|\times{}d$. 
\end{enumerate}

Finally, the sum of the resulting literals is trivially contradictory.
The sum
\[
    \sum_{k=1}^{o}\sum_{i=1}^{m^o}c_i^k*t_i^k \bowtie \sum_{k=1}^{o}d^k
\]
where $c_i^k$ and $t_i^k$ are the constant and term from the literal $varphi_k$, and $d^k$ is the constant $d$ of $varphi_k$.
The operator $\bowtie$ is $=$ if all operators are $=$, $>$ if all are either $=$ or $>$, and $\geq$ otherwise. The $a_i$ must be such
that the sum on the left-hand side is $0$ and the right-hand side is $>0$ (or $\geq 0$ if $\bowtie$ is $>$).

The step 3 has been added by our own, as the subsequent steps in the original algorithm are designed for $>$ and $\geq$ and do not clearly address how to handle $<$ and $\leq$.
Additionally, step 6  was added to ensure that our construction is independent of $\mathbb{Q}$.

\begin{example}
Consider the $\tt{la\_generic}$ step in the logic \tt{QF\_UFLIA} with the uninterpreted function symbol \lstinline[language=SMT]|(f Int)|:
\begin{lstlisting}[language=SMT,label={lst:smtexampleinput}]
(step t11 (cl (not (<= f 0)) (<= (+ 1 (* 4 f)) 1))
  :rule la_generic :args (1/1 1/4))
\end{lstlisting}
After step 4, we get $- f > 0 ~\land~ 4 \times f > 0$. The step 5 applied and we can strengthen this to
$- f \geq 0 ~\land~ 4 \times f \geq 1$ and after multiplication of the normalized coefficients at step 6, we get $4 \times(- f) \geq 0 ~\land~ 4 \times f \geq 1$.
Which sums to the contradiction  $0 \geq 1$. 
\end{example}


The \tt{lia\_generic} is structurally similar to \tt{la\_generic}, but omits the coefficients.
Since this rule can introduce a disjunction of arbitrary linear integer inequalities without any additional hints, proof checking is \emph{NP-complete} \cite{Schrijver:lia}.

\section{The approach to reconstruct lia\_generic step}

The \tt{lia\_generic} represent a challenge for reconstruction because the coefficients are not provided by the solver in the trace i.e. \colorbox{orange!30}{$[a_1 \dots a_r]$} is empty.
We decided to leverage the elaboration process of $\tt{lia\_generic}$ performed by Carcara, as doing otherwise would require implementing Fourier-Motzkin elimination for integers, as demonstrated in \cite{omegatest,micromega} -
and therefore reimplementing the work done by the solver.

Carcara considers $\tt{lia\_generic}$ steps as holes in the proof,  as their checking is as hard as solving \cite[\S 3.2]{carcara}.
However, Carcara relies on an external tool that generates Alethe proofs to formulate the steps by solving corresponding problems in a proof-producing manner.
The proof is then imported, verified, and validated before replacing the original step.
However, at present, Carcara only use cvc5 for performing this task.
It is important to note that cvc5 has a limitation: its Alethe proofs may contain rewrite steps that are not yet modeled in the Alethe simplification rules, and as such, these steps are not supported by Carcara.
While these steps are considered holes, they typically involve simple simplification rules and, therefore, have much less impact than the more complex $\tt{lia\_generic}$ ones.

In detail, the elaboration method, when encountering a $\tt{lia\_generic}$ step S concluding the negated inequalities $ \neg l_1 \lor \dots \neg l_n$ , generates an SMT-LIB problem asserting $l_1 \land \dots \land l_n$ and invokes \emph{cvc5} on it, expecting an Alethe proof $\pi : (l_1 \land \dots \land l_n) \ra \bot$.
Carcara will check this subproof and then replace step $S$ in the original proof by a proof of the form:
\begin{lstlisting}[language=SMT,caption={Elaboration of $\tt{lia\_generic}$},label={lst:elab_lia}]
(anchor :step S.t_m+1)
(assume S.h_1 l1)
...
(assume S.h_n ln)
...
(step t.t_m (cl false) :rule ...)
(step t.t_m+1 (cl (not l1) ... (not ln) false) :rule subproof)
(step t.t_m+2 (cl (not false)) :rule false)
(step S (cl (not l1) ... (not ln)) :rule resolution :premises (S.t_m+1 S.t_m+2))
\end{lstlisting}

