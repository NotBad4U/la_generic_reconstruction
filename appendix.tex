
\appendix


\section{Alethe}
\label{app:alethe}


\subsection{The Syntax}

\begin{figure}[htb]%[H]
    \[
      \begin{array}{r c l}
     \grNT{proof}           &\grRule & \grNT{proof\_command}^{*} \\
     \grNT{proof\_command}  &\grRule & \textAlethe{(assume}\; \grNT{symbol}\; \grNT{proof\_term}\,\textAlethe{)} \\
                            &\grOr   & \textAlethe{(step}\; \grNT{symbol}\; \grNT{clause}
                                            \; \textAlethe{:rule}\; \grNT{symbol} \\
                            &        & \quad \grNT{premises\_annotation}^{?} \\
                            &        & \quad \grNT{context\_annotation}^{?}\;\grNT{attribute}^{*}\,\textAlethe{)} \\
                            & \grOr  & \textAlethe{(anchor :step}\; \grNT{symbol}\;
                                                \\
                            &        & \quad \grNT{args\_annotation}^{?}\;\grNT{attribute}^{*}\,\textAlethe{)} \\
                            & \grOr  & \textAlethe{(define-fun}\; \grNT{function\_def}\,\textAlethe{)} \\
     \grNT{clause}          &\grRule & \textAlethe{(cl}\; \grNT{proof\_term}^{*}\,\textAlethe{)} \\
     \grNT{proof\_term}     &\grRule & \grNT{term}\text{ extended with } \\
                            &        & \textAlethe{(choice (}\, \grNT{sorted\_var}\,\textAlethe{)}\; \grNT{proof\_term}\,\textAlethe{)}  \\
     \grNT{premises\_annotation} &\grRule & \textAlethe{:premises (}\; \grNT{symbol}^{+}\textAlethe{)} \\
     \grNT{args\_annotation}     &\grRule & \textAlethe{:args}\,\textAlethe{(}\,\grNT{step\_arg}^{+}\,\textAlethe{)}  \\
     \grNT{step\_arg}            &\grRule & \grNT{symbol} \grOr
                                              \textAlethe{(}\; \grNT{symbol}\; \grNT{proof\_term}\,\textAlethe{)} \\
     \grNT{context\_annotation}  &\grRule & \textAlethe{:args}\,\textAlethe{(}\,\grNT{context\_assignment}^{+}\,\textAlethe{)}  \\
     \grNT{context\_assignment}  &\grRule & \textAlethe{(}    \,\grNT{sorted\_var}\,\textAlethe{)}  \\
                                 & \grOr  & \textAlethe{(:=}\, \grNT{symbol}\;\grNT{proof\_term}\,\textAlethe{)} \\
      \end{array}
      \]
      \caption{Alethe grammar}
      \label{fig:grammar}
\end{figure}

%  Lambdapi Definitions for Integer and Binary Arithmetic
\section{Lambdapi Formalizations for Integer and Binary Number Operations}
\label{app:lambdapi-func-def}

\noindent
\begin{minipage}[t]{0.48\textwidth}
% \textbf{\ttfamily add : \bb{P} \ra \bb{P} \ra \bb{P}}

\begin{align*}
&\tt{add} : \bb{P} \ra \bb{P} \ra \bb{P} \\
& \tt{add}~(\tt{I}~x)~(\tt{I}~q) \re \tt{O}~(\tt{add\_c}~x~q) \\
& \tt{add}~(\tt{I}~x)~(\tt{O}~q) \re \tt{I}~(\tt{add}~x~q) \\
& \tt{add}~(\tt{O}~x)~(\tt{I}~q) \re \tt{I}~(\tt{add}~x~q) \\
& \tt{add}~(\tt{O}~x)~(\tt{O}~q) \re \tt{O}~(\tt{add}~x~q) \\
& \tt{add}~x~\tt{H} \re \tt{succ}~x \\
& \tt{add}~\tt{H}~y \re \tt{succ}~y \\
\end{align*}
\end{minipage}
\hfill
\begin{minipage}[t]{0.48\textwidth}
\begin{align*}
&\tt{add\_c} : \bb{P} \ra \bb{P} \ra \bb{P} \\
& \tt{add\_c}~(\tt{I}~x)~(\tt{I}~q) \re \tt{I}~(\tt{add\_c}~x~q) \\
& \tt{add\_c}~(\tt{I}~x)~(\tt{O}~q) \re \tt{O}~(\tt{add\_c}~x~q) \\
& \tt{add\_c}~(\tt{O}~x)~(\tt{I}~q) \re \tt{O}~(\tt{add\_c}~x~q) \\
& \tt{add\_c}~(\tt{O}~x)~(\tt{O}~q) \re \tt{I}~(\tt{add}~x~q) \\
& \tt{add\_c}~x~\tt{H} \re \tt{add}~x~(\tt{O}~\tt{H}) \\
& \tt{add\_c}~\tt{H}~y \re \tt{add}~(\tt{O}~\tt{H})~y \\
\end{align*}
\end{minipage}

\begin{align*}
&\tt{sub} : \bb{P} \ra \bb{P} \ra \bb{Z} \\
& \tt{sub}~(\tt{I}~p)~(\tt{I}~q) \re \tt{double}~(\tt{sub}~p~q) \\
& \tt{sub}~(\tt{I}~p)~(\tt{O}~q) \re \tt{succ\_double}~(\tt{sub}~p~q) \\
& \tt{sub}~(\tt{I}~p)~\tt{H} \re \ZPos (\tt{O}~p) \\
& \tt{sub}~(\tt{O}~p)~(\tt{I}~q) \re \tt{pred\_double}~(\tt{sub}~p~q) \\
& \tt{sub}~(\tt{O}~p)~(\tt{O}~q) \re \tt{double}~(\tt{sub}~p~q) \\
& \tt{sub}~(\tt{O}~p)~\tt{H} \re \ZPos (\tt{pos\_pred\_double}~p) \\
& \tt{sub}~\tt{H}~(\tt{I}~q) \re \ZNeg    (\tt{O}~q) \\
& \tt{sub}~\tt{H}~(\tt{O}~q) \re \ZNeg    (\tt{pos\_pred\_double}~q) \\
& \tt{sub}~\tt{H}~\tt{H} \re \tt{Z0} \\
\end{align*}

\begin{align*}
&\tt{compare\_acc} : \bb{P} \ra \tt{Comp} \ra \bb{P} \ra \tt{Comp} \\
& \tt{compare\_acc}~(\tt{I}~x)~c~(\tt{I}~q) \re \tt{compare\_acc}~x~c~q \\
& \tt{compare\_acc}~(\tt{I}~x)~\_~(\tt{O}~q) \re \tt{compare\_acc}~x~\tt{Gt}~q \\
& \tt{compare\_acc}~(\tt{I}~\_)~\_~\tt{H} \re \tt{Gt} \\
& \tt{compare\_acc}~(\tt{O}~x)~\_~(\tt{I}~q) \re \tt{compare\_acc}~x~\tt{Lt}~q \\
& \tt{compare\_acc}~(\tt{O}~x)~c~(\tt{O}~q) \re \tt{compare\_acc}~x~c~q \\
& \tt{compare\_acc}~(\tt{O}~\_)~\_~\tt{H} \re \tt{Gt} \\
& \tt{compare\_acc}~\tt{H}~\_~(\tt{I}~\_) \re \tt{Lt} \\
& \tt{compare\_acc}~\tt{H}~\_~(\tt{O}~\_) \re \tt{Lt} \\
& \tt{compare\_acc}~\tt{H}~c~\tt{H} \re c \\
\\
&\tt{compare}~x~y \coloneq \tt{compare\_acc}~x~\tt{Eq}~y \\
\end{align*}
  
\begin{minipage}[t]{0.45\textwidth}
\begin{align*}
&\tt{mul} : \bb{P} \ra \bb{P} \ra \bb{P} \\
& \tt{mul}~(\tt{I}~x)~y \re \tt{add}~y~(\tt{O}~(\tt{mul}~x~y)) \\
& \tt{mul}~(\tt{O}~x)~y \re \tt{O}~(\tt{mul}~x~y) \\
& \tt{mul}~\tt{H}~y \re y \\
& \tt{mul}~y~\tt{H} \re y \\
\end{align*}
\end{minipage}
\begin{minipage}[t]{0.48\textwidth}
\begin{align*}
&\tt{*} : \bb{Z} \ra \bb{Z} \ra \bb{Z} \\
& \tt{Z0} *~\_ \re \tt{Z0} \\
& \_ *~\tt{Z0} \re \tt{Z0} \\
& \ZPos x *~\ZPos y \re \ZPos (\tt{mul}~x~y) \\
& \ZPos x *~\ZNeg    y \re \ZNeg    (\tt{mul}~x~y) \\
& \ZNeg    x *~\ZPos y \re \ZNeg    (\tt{mul}~x~y) \\
& \ZNeg    x *~\ZNeg    y \re \ZPos (\tt{mul}~x~y) \\
\end{align*}
\end{minipage}

\subsection{Confluence of the rewriting rules of integers and positive binary number}
\label{app:confluence-int-pos}

The rules presented below represent the relations $\ra_\bb{Z}$ and $\ra_\bb{P}$ encoded in the TRS\footnote{\url{http://www.lri.fr/~marche/tpdb/format.html}} format accepted by the \cite{CSI} tool.
These rules can be used to rerun the tool in order to verify the confluence property.

\begin{lstlisting}[language=trs, caption=Rewriting rule of $\bb{Z}$ and $\bb{P}$ in the TRS format]
(VAR
  a: Z
  b: Z
  x : P
  q : P
  y : P
)
(RULES
  ~(Z0) -> Z0
  ~(Zpos(p)) -> Zneg(p)
  ~(Zneg(p)) -> Zpos(p)
  ~(~(a)) -> a

  double(Z0) -> Z0
  double(Zpos(p)) -> Zpos(O(p))
  double(Zneg(p)) -> Zneg(O(p))
  
  succ_double(Z0) -> Zpos(H)
  succ_double(Zpos(p)) -> Zpos(I(p))
  succ_double(Zneg(p)) -> Zneg(pos_pred_double(p))
  
  pred_double(Z0) -> Zneg(H)
  pred_double(Zpos(p)) -> Zpos(pos_pred_double(p))
  pred_double(Zneg(p)) -> Zneg(I(p))

  sub(I(p), I(q)) -> double(sub(p, q))
  sub(I(p), O(q)) -> succ_double(sub(p, q))
  sub(I(p), H) -> Zpos(O(p))
  sub(O(p), I(q)) -> pred_double(sub(p, q))
  sub(O(p), O(q)) -> double(sub(p, q))
  sub(O(p), H) -> Zpos(pos_pred_double(p))
  sub(H, I(q)) -> Zneg(O(q))
  sub(H, O(q)) -> Zneg(pos_pred_double(q))
  sub(H, H) -> Z0

  +(Z0,a) -> a
  +(a,Z0) -> a
  +(Zpos(x), Zpos(y)) -> Zpos(add(x, y))
  +(Zpos(x), Zneg(y)) -> sub(x, y)
  +(Zneg(x), Zpos(y)) -> sub(y, x)
  +(Zneg(x), Zneg(y)) -> Zneg(add(x, y))
  
  mult(Z0, a) -> Z0
  mult(a, Z0) -> Z0
  mult(Zpos(x), Zpos(y)) -> Zpos(mul(x, y))
  mult(Zpos(x), Zneg(y)) -> Zneg(mul(x, y))
  mult(Zneg(x), Zpos(y)) -> Zneg(mul(x, y))
  mult(Zneg(x), Zneg(y)) -> Zpos(mul(x, y))


  succ(I(x)) -> O(succ(x))
  succ(O(x)) -> I(x)
  succ(H) -> O(H)
  add(I(x), I(q)) -> O(addcarry(x, q))
  add(I(x), O(q)) -> I(add(x, q))
  add(O(x), I(q)) -> I(add(x, q))
  add(O(x), O(q)) -> O(add(x, q))
  add(x, H) -> succ(x)
  add(H, y) -> succ(y)

  addcarry(I(x), I(q)) -> I(addcarry(x, q))
  addcarry(I(x), O(q)) -> O(addcarry(x, q))
  addcarry(O(x), I(q)) -> O(addcarry(x, q))
  addcarry(O(x), O(q)) -> I(add(x, q))
  addcarry(x, H) -> add(x, O(H))
  addcarry(H, y) -> add(O(H), y)
  
  pos_pred_double(I(x)) -> I(O(x))
  pos_pred_double(O(x)) -> I(pos_pred_double(x))
  pos_pred_double(H) -> H
  
  mul(I(x), y) -> add(x, O(mul(x,y)))
  mul(O(x), y) -> O(mul(x, y))
  mul(H, y) -> y
)
\end{lstlisting}


% Proof:
%  Church Rosser Transformation Processor (no redundant rules):
%   strict:
   
%   weak:
   
% critical peaks: 4
%   add(O(H()),H()) <-6|[]- addcarry(H(),H()) -7|[]-> add(H(),O(H()))
%   add(H(),O(H())) <-7|[]- addcarry(H(),H()) -6|[]-> add(O(H()),H())
%   succ(H()) <-12|[]- add(H(),H()) -13|[]-> succ(H())
%   succ(H()) <-13|[]- add(H(),H()) -12|[]-> succ(H())

%   Church Rosser Transformation Processor (critical pair closing system, Thm 2.4):
%   add(x,H()) -> succ(x)
%   add(H(),y) -> succ(y)
%   critical peaks: joinable
%   Matrix Interpretation Processor: dim=1
    
%     interpretation:
%     [succ](x0) = 4x0,
    
%     [add](x0, x1) = 4x0 + 4x1 + 2,
    
%     [H] = 4
%     orientation:
%     add(x,H()) = 4x + 18 >= 4x = succ(x)
    
%     add(H(),y) = 4y + 18 >= 4y = succ(y)
%     problem:
    
%     Qed


% Critical pair diagram

\begin{lemma}[Confluence of $\ra_\bb{Z}$ and $\ra_\bb{P}$]
\begin{proof}
CSI automatically proves the confluence of $\ra_\bb{Z}$ and $\ra_\bb{P}$ by giving the polynomial interpretation:
\begin{align*}
[\tt{succ}(x)] = 4*x & &[\tt{add}(x, y)] = 4 * x + 4 * y + 2  & &[ {\tt{H}} ] = 4 \\
\end{align*}
\end{proof}
\label{lemma:confluenceZP}
\end{lemma}

\section{Proofs of Normalization Lemmas}

\begin{lemma} 
	$\ACcanon$ is locally confluent on AC-canonical terms.
\end{lemma}
\begin{proof}
We show that every critical pair is joinable using $\ACcanon$ and confluence of $\ra_\bb{Z}$ and $\ra_\bb{P}$ from \cref{lemma:confluenceZP}.
We use the notation $\MRAC$ to denote a multistep rewriting with AC-canonical form.

% t ≔ opp (add (var $k $x) (var $c $x))
% t ↪[] add (opp (var $k $x)) (opp (var $c $x)) ↪* var (— $k + — $c) $x
%   with opp (add $x' $y') ↪ add (opp $0') (opp $1')
% t ↪[1] opp (var ($0 + $2) $1) ↪* var (— ($0 + $2)) $1
%   with add (var $k $x) (var $c $x) ↪ var ($0 + $2) $1
\begin{center}
\cp
{
  \opp{\underline{\var{c_1}{x} \oplus \var{c_2}{x}}}
}
{
  \opp{\var{c_1}{x}} \oplus \opp{\var{c_2}{x}}
}
{
  \opp{\var{(c_1 + c_2)}{x}}
}
{11}{2}
\end{center}

But $\opp{\var{c_1}{x}} \oplus \opp{\var{c_2}{x}} \MRAC \var{(\sim c_1 + \sim c_2)}{x}$
and $\opp{\var{(c_1 + c_2)}{x}} \MRAC \var{\sim (c_1 + c_2)}{x}$ converge by the confluence of $\ra_\bb{Z}$.

% t ≔ opp (add (var $k $x) (add (var $l $x) $y))
% t ↪[] add (opp (var $k $x)) (opp (add (var $l $x) $y)) ↪* add (var (— $k + — $l) $x) (opp $y)
%   with opp (add $x' $y') ↪ add (opp $0') (opp $1')
% t ↪[1] opp (add (var ($0 + $2) $1) $3) ↪* add (var (— ($0 + $2)) $1) (opp $3)
%   with add (var $k $x) (add (var $l $x) $y) ↪ add (var ($0 + $2) $1) $3
\cp
{
  \opp{\underline{(\var{c_1}{x} \oplus (\var{c_2}{x} \oplus y))}}
}
{
  \opp{\var{c_1}{x}} \oplus \opp{(\var{c_2}{x} \oplus y)}
}
{
  \opp{\var{(c_1 + c_2)}{x} \oplus y}
}
{11}{3}

We note that $\opp{\var{c_1}{x}} \oplus \opp{(\var{c_2}{x} \oplus y)} \MRAC \add{\var{(\sim c_1 + \sim c_2)}{x}}{(\opp{y})}$
and $\opp{\var{(c_1 + c_2)}{x} \oplus y} \MRAC \add{\var{(\sim (c_1 + c_2))}{x}}{(\opp{y})}$
both reduce to the same term, as guaranteed by the confluence of $\ra_\bb{Z}$.

% t ≔ opp (add (cst $k) (cst $l))
% t ↪[] add (opp (cst $k)) (opp (cst $l)) ↪* cst (— $k + — $l)
%   with opp (add $x' $y') ↪ add (opp $0') (opp $1')
% t ↪[1] opp (cst ($0 + $1)) ↪* cst (— ($0 + $1))
%   with add (cst $k) (cst $l) ↪ cst ($0 + $1)
\cp{
  \opp{\underline{\cst{c_1} \oplus \cst{c_2}}}
}
{
  \opp{\cst{c_1}} \oplus \opp{\cst{c_2}}
}
{
  \opp{\cst{(c_1 + c_2)}}
}
{11}{4}

We observe that $\opp{\cst{c_1}} \oplus \opp{\cst{c_2}} \MRAC \cst{(\sim c_1) + (\sim c_2)}$
and $\opp{\cst{(c_1 + c_2)}} \MRAC \cst{\sim (c_1 + c_2)}$ reduce to the same result due to the confluence of $\ra_\bb{Z}$.


% t ≔ opp (add (cst $k) (add (cst $l) $y))
% t ↪[] add (opp (cst $k)) (opp (add (cst $l) $y)) ↪* add (cst (— $k + — $l)) (opp $y)
%   with opp (add $x' $y') ↪ add (opp $0') (opp $1')
% t ↪[1] opp (add (cst ($0 + $1)) $2) ↪* add (cst (— ($0 + $1))) (opp $2)
%   with add (cst $k) (add (cst $l) $y) ↪ add (cst ($0 + $1)) $2
\cp
{
  \opp{\underline{\cst{c_1} \oplus (\cst{c_2} \oplus y)}}
}
{
  \opp{\cst{c_1}} \oplus \opp{(\cst{c_2} \oplus y)}
}
{
  \opp{(\cst{c_1 + c_2} \oplus y)}
}
{11}{5}

The expressions $\opp{\cst{c_1}} \oplus \opp{(\cst{c_2} \oplus y)} \MRAC \cst{(\sim c_1) + (\sim c_2)} \oplus \opp{y}$
and $ \opp{(\cst{c_1 + c_2} \oplus y)} \MRAC \add{ \cst{\sim (c_1 + c_2)} }{\opp{y}}$
both reduce to a common term, which follows from the confluence property of $\ra_\bb{Z}$.

% t ≔ add (var $k' $x) (add (var $k $x) (add (var $l $x) $y))
% t ↪[] add (var ($0' + $k) $x) (add (var $l $x) $y) ↪* add (var (($0' + $k) + $l) $x) $y
%   with add (var $k' $x') (add (var $l' $x') $y') ↪ add (var ($0' + $2') $1') $3'
% t ↪[1] add (var $k' $x) (add (var ($0 + $2) $1) $3) ↪* add (var ($k' + ($0 + $2)) $x) $3
%   with add (var $k $x) (add (var $l $x) $y) ↪ add (var ($0 + $2) $1) $3

\cp
{
  \var{c_1}{x} \oplus \underline{(\var{c_2}{x} \oplus (\var{c_3}{x} \oplus y))}
}
{
  \var{(c_1 + c_2)}{x} \oplus ((\var{c_3}{x} \oplus y))
}
{
  \var{c_1}{x} \oplus (\var{(c_2 + c_3)}{x} \oplus y))
}{3}{3}

The terms $\var{(c_1 + c_2)}{x} \oplus ((\var{c_3}{x} \oplus y)) \MRAC \add{\var{(c_1 + (c_2 + c_3))}{x}}{y}$
and $\var{c_1}{x} \oplus (\var{(c_2 + c_3)}{x} \oplus y)) \MRAC \add{\var{((c_1 + c_2) + c_3)}{x}}{y}$
both reduce to the same result due to the confluence of $\ra_\bb{Z}$.

% t ≔ add (cst $k') (add (cst $k) (add (cst $l) $y))
% t ↪[] add (cst ($0' + $k)) (add (cst $l) $y) ↪* add (cst (($0' + $k) + $l)) $y
%   with add (cst $k') (add (cst $l') $y') ↪ add (cst ($0' + $1')) $2'
% t ↪[1] add (cst $k') (add (cst ($0 + $1)) $2) ↪* add (cst ($k' + ($0 + $1))) $2
%   with add (cst $k) (add (cst $l) $y) ↪ add (cst ($0 + $1)) $2

\cp
{\cst{c_1} \oplus \underline{(\cst{c_2} \oplus (\cst{c_3} \oplus y))}}
{\cst{(c_1 + c_2)} \oplus (\cst{c_3} \oplus y) }
{\cst{c_1} \oplus (\cst{(c_2 + c_3)} \oplus y) }
{5}{5}

We observe that $\cst{c_1 + c_2} \oplus (\cst{c_3} \oplus y) \MRAC \add{\cst{((c_1 + c_2) + c_3)}}{y}$
and $\cst{c_1} \oplus (\cst{(c_2 + c_3)} \oplus y) \MRAC \add{\cst{(c_1 + (c_2 + c_3)}}{y}$
both reduce to the same expression due to the confluence property of $\ra_\bb{Z}$.

% t ≔ mul $k' (mul $k (mul $l $z))
% t ↪[] mul ($0' * $k) (mul $l $z) ↪* mul (($0' * $k) * $l) $z
%   with mul $k' (mul $l' $z') ↪ mul ($0' * $1') $2'
% t ↪[1] mul $k' (mul ($0 * $1) $2) ↪* mul ($k' * ($0 * $1)) $2
%   with mul $k (mul $l $z) ↪ mul ($0 * $1) $2
\cp
{ \mul{c_1}{\underline{(\mul{c_2}{(\mul{c_3}{z})})}} } 
{ \mul{(c_1)}{ (\mul{(c_2 * c_3)}{z}) } }
{ \mul{(c_1 * c_2)}{(\mul{c_3}{z}})} 
{16}{16}
\end{proof}

Finally, we have $\mul{(c_1)}{(\mul{(c_2 * l)}{z})} \MRAC \mul{((c_1 * c_2) * c_3)}{z})$
and $\mul{(c_1 * c_2)}{(\mul{c_3}{z}}) \MRAC \mul{(c_1 * (c_2 * c_3))}{z})$ that converge by the confluence of $\ra_\bb{Z}$.

\section{Translation example}
\label{app:example-translation}

We present in \cref{lst:translation-example} the proof translated using our module integrated into Carcara.
Our focus is on the translation of the $\tt{la\_generic}$ rule applied in steps \tt{t2} and \tt{t5}.
The proof for \tt{t2} begins at line 28, while the proof for \tt{t5} starts at line 92.
In what follows, we detail only the translation of step \tt{t2}; the translation of \tt{t5} proceeds analogously.

The proof script from lines 29 to 59 corresponds to the algorithm described in \cref{sssect:la-in-alethe}.
The step \lstinline[language=Lambdapi,basicstyle=\ttfamily\footnotesize\upshape]|rewrite Zinv_lt_eq| at line 32 reflects the normalization procedure in Step 1, where the lemma rewrites $\neg (3 < x)$ as $\neg (\sim 3 > \sim x)$.

The lemmas \lstinline[language=Lambdapi,basicstyle=\ttfamily\footnotesize\upshape]|Z_diff_gt_Z0| and \lstinline[language=Lambdapi,basicstyle=\ttfamily\footnotesize\upshape]|Z_diff_eq_Z0_eq|,
used at lines 33–34, implement Step 2 by rewriting expressions such as $\neg (x = 2)$ into $\neg (x - 2 = 0)$, bringing all terms to one side.
Since the goal involves a $>$ inequality, we can apply Step 3 that strengthen it using the lemma \lstinline[language=Lambdapi,basicstyle=\ttfamily\footnotesize\upshape]|Zgt_le_succ_r_eq|,
which rewrites $\neg (\sim 3 + x > 0)$ into $\neg (\sim 3 + x \geq 1)$, acknowledging that $x$ is of type \tt{Int}, i.e., $\bb{Z}$.

The coefficients involved in step \tt{t2}, namely $[\frac{1}{1}, -\frac{1}{1}]$, are trivially cast into $\bb{Z}$ in Step 4,
resulting in $[1, -1]$. Lines 36 and 37 illustrate the application of these coefficients to their respective inequalities via the appropriate lemma.

Finally, the contradiction derived in Step 6 is constructed by summing the inequalities from lines 47 to 52.
This is facilitated by \cref{lem:conv}, which reifies the sum of the inequalities into $\bb{G}$, and by \cref{thm:normalization} with $\RAC$,
which derives the contradiction, allowing us to conclude the proof at line 58.

\lstinputlisting[language=Lambdapi,mathescape=true,label={lst:translation-example},caption={Lambdapi proof of \cref{lst:smtexampleproof}}]{lp_example.lp}