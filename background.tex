\section{Background}
\label{sect:background}

\subsection{An Overview of Lambdapi}
\label{ssect:lambdapi-overview}

Lambdapi is an implementation of $\lambda\Pi$ modulo theory ($\lpm$) \cite{lambdapi}, an extension of the Edinburgh Logical Framework $\lambda\Pi$ \cite{lf}, a simply typed $\lambda$-calculus with dependent types. $\lpm$ adds user-defined higher-order rewrite rules. Its syntax is given by
%
\begin{align*}
&\text{Universes}  &u &::= \tt{TYPE} ~|~ \tt{KIND} \\
&\text{Terms}   &t,v, A,B,C &::= c ~|~ x ~|~ u ~|~ \Pi\,x : A,\,B~|~ \lambda\,x : A,\,t ~|~t~v \\
&\text{Contexts}   &\Gamma &::= \langle \rangle ~|~ \Gamma, x : A \\
&\text{Signatures}  &\Sigma &::= \langle \rangle ~|~ \Sigma, c : C ~|~ \Sigma, c := t : C ~|~ \Sigma, t \hookrightarrow v 
\end{align*}
%
where $c$ is a constant and $x$ is a variable  (ranging over disjoint sets), $C$ is a closed term. \emph{Universes} are constants used to verify if a type is well-formed -- more details can be found in \cite[\S 2.1]{lf}. $\Pi\,x : A,\,B$ is the dependent product, and we write $A \rightarrow B$ when $B$ does not depend on $x$, $\lambda\,x : A.\,t$ is an abstraction, and  $t~v$ is an application. A \emph{(local) context} $\Gamma$ is a finite sequence of variable declarations $x:A$ introducing variables and their types.
A \emph{signature} $\Sigma$ representing the global context is a finite sequence of \emph{assumptions} $c : C$, indicating that constant $c$ is of type $C$, \emph{definitions} $c := t : C$, indicating that $c$ has the value $t$ and type $C$, and \emph{rewrite rules} $t \hookrightarrow v$ such that $t = c~v_1 \dots v_n$ where $c$ is a constant.

The relation $\hookrightarrow_{\beta\Sigma}$ is generated by $\beta$-reduction and by the rewrite rules of $\Sigma$. The relation $\hookrightarrow_{\beta\Sigma}^*$ denotes the reflexive and transitive closure of $\hookrightarrow_{\beta\Sigma}$, and the relation $\equiv_{\beta\Sigma}$ (called \emph{conversion}) the reflexive, symmetric, and transitive closure of $\hookrightarrow_{\beta\Sigma}$. 
The relation $\hookrightarrow_{\beta\Sigma}$ must be confluent, i.e.,
whenever $t \hookrightarrow_{\beta\Sigma}^* v_1$ and $t \hookrightarrow_{\beta\Sigma}^* v_2$, there exists a term $w$ such that $v_1 \hookrightarrow_{\beta\Sigma}^* w$ and $v_2 \hookrightarrow_{\beta\Sigma}^* w$, and it must preserve typing, i.e., 
whenever $\Gamma \vdash_\Sigma t: A$ and $t \hookrightarrow_{\beta\Sigma} v$ then $\Gamma \vdash_\Sigma v: A$ \cite{blanqui:LIPIcs.FSCD.2020.13}.

A Lambdapi typing judgment $\Gamma \vdash_\Sigma t : A$ asserts that term $t$ has type $A$ in the context $\Gamma$ and the signature $\Sigma$.
The typing rules of $\lpm$ are the one of  $\lambda\Pi$ \cite[\S 2]{lf}, except for the rule (Conv) where it use the version of \cref{fig:lp-typing-rules} that identifies types modulo~$\textcolor{orange}{\equiv_{\beta\Sigma}}$ instead of just modulo $\beta$-reduction. 

\begin{figure}
    \begin{center}
    \begin{prooftree}
    \hypo{\Gamma, \vdash_\Sigma B: u}
    \hypo{\Gamma \vdash_\Sigma t: A}
    \hypo{\textcolor{orange}{A \equiv_{\beta\Sigma} B}}
    \infer3[(Conv)]{ \Gamma \vdash_\Sigma t: B }
    \end{prooftree}
    \end{center}
    \caption{(Conv) rule in $\lpm$}
    \label{fig:lp-typing-rules}
\end{figure}

In our encoding presented \cite{ColtellacciMD24} we employ Tarski-style universe \cite[\S Universes]{intuitype} where types are represented by elements of a base type and interpreted via the decoding function.
We defined the constant $\prop: \type$ for the type of proposition and the decoding function $\pic: \prop \ra \type$ that maps each proposition to $\type$.
Since Lambdapi does not support quantifiy over a variable of type $\type$. More precisely, it is not possible to assign the type $\Pi X : \type,(X \ra \prop) \ra \prop$ to the universal quantifier $\forall$.
To circumvent this, we have the constant $\set : \type$ for the types of object-terms, and a decoding function ${\el}: \set \ra \type$ that embeds $\set$ into $\type$.
This permits us to define the quantifier as $\forall: \Pi x: \set, (\el~x \ra \prop) \ra \prop$.
To quantify over propositions, we further define a constant $o: \set$ and add the rewrite rule  $\el\, o \re \prop$.

\subsection{Alethe proof}
\label{ssect:alethe}

The Alethe proof trace format \cite{alethespec} for SMT solvers comprises two parts: the trace language based on SMT-LIB and a collection of proof rules. Traces witness proofs of unsatisfiability of a set of constraints.
They are sequences $a_1 \dots a_m~t_1 \dots t_n$ where
the $a_i$ corresponds to the constraints of the original SMT problem being refuted, each $t_i$ is a clause inferred from previous elements of the sequence, and $t_n$ is $\bot$ (the empty clause).
In the following, we designate the SMT-LIB problem as the \emph{input problem}.

\begin{lstlisting}[language=SMT,label={lst:smtexampleinput},caption={Input problem}]
(set-logic QF_LIA)
(declare-const x Int)
(declare-const y Int)
(assert (= 0 y))
(assert (= x 2))
(assert (or (< (+ x y) 1) (< 3 x)))
(get-proof)
\end{lstlisting}

\begin{center}
\lightning
\end{center}

\begin{lstlisting}[language=SMT,caption={The following example is the proof for the unsatisfiability of $(x+y < 1) \lor (3<x), x = 2$ and $0 = y$.},label={lst:smtexampleproof}]
(assume a0 (or (< (+ x y) 1) (< 3 x)))
(assume a1 (= x 2))
(assume a2 (= 0 y))
(step t1 (cl (< (+ x y) 1) (< 3 x)) :rule or :premises (a0))
(step t2 (cl (not (< 3 x)) (not (= x 2))) :rule la_generic :args (1/1 1/1))
(step t3 (cl (not (< 3 x))) :rule resolution :premises (a1 t2))
(step t4 (cl (< (+ x y) 1)) :rule resolution :premises (t1 t3))
(step t5 (cl (not (< (+ x y) 1)) (not (= x 2)) (not (= 0 y))) :rule la_generic :args (1/1 -1/1 1/1))
(step t6 (cl) :rule resolution :premises (t5 t4 a1 a2))
\end{lstlisting}

We will use the input problem shown in the top part of \cref{lst:smtexampleinput} with its Alethe proof (found by cvc5) in the bottom part as a running example to provide an overview of Alethe concepts and to illustrate our reconstruction of linear arithmetic step in Lambdapi.

\subsubsection{Alethe Trace Format Overview}
\label{sssect:alethe-trace-overview}

An Alethe proof trace inherits the declarations of its input problem. All symbols (sorts, functions, assertions, etc.) declared or defined in the input problem remain declared or defined, respectively. Furthermore, the syntax for terms, sorts, and annotations uses the syntactic rules defined in SMT-LIB \cite[\S 3]{smtlib} and the SMT signature context defined in \cite[\S 5.1 and \S 5.2]{smtlib}.
In the following we will represent an Alethe step as

\smallskip

\renewcommand{\eqnhighlightshade}{35}

\begin{equation}
\label{eq:step}
\tag{\textcolor{purple}{1}}
\eqnmarkbox[indexClr]{node2}{i}. \quad \eqnmarkbox[blue]{node1}{\Gamma} ~\triangleright~ \eqnmarkbox[green]{node3}{l_1 \dots l_n} \quad (\eqnmarkbox[purple]{node4}{\mathcal{R}}~\eqnmarkbox[red]{node5}{p_1 \dots p_m})~\eqnmarkbox[orange]{node6}{[a_1 \dots a_r]}
\end{equation}

\vspace{0.3em}
\annotate[yshift=-0.5em]{below, left}{node2}{index}
\annotate[yshift=-0.5em]{below, right}{node1}{context}
\annotate[yshift=0.5em]{above, left}{node3}{clause}
\annotate[yshift=-0.5em]{below, right}{node4}{rule}
\annotate[yshift=-0.5em]{below, right}{node5}{premises}
\annotate[yshift=-0.5em]{below, right}{node6}{arguments}

\vspace{0.3em}

\medskip

A step %\cref{eq:step} 
consists of an index \colorbox{indexClr!30}{$i$} $\in \mathbb{I}$ where $\mathbb{I}$ is a countable infinite set of indices (e.g. \kw{a0}, \kw{t1}), and a clause of formulae \colorbox{green!30}{$l_1, \dots, l_n$} representing an $n$-ary disjunction. Steps that are not assumptions are justified by a proof rule \colorbox{purple!30}{$\mathcal{R}$} that depends on a possibly empty set of premises $\{\colorbox{red!30}{$p_1 \dots  p_m$}\} \subseteq \mathbb{I}$ that only references earlier steps such that the proof forms
a directed acyclic graph. A rule might also depend on a list of arguments \colorbox{orange!30}{$[a_1 \dots a_r]$} where each argument $a_i$ is either a term or a pair $(x_i, t_i)$ where $x_i$ is a variable and $t_i$ is a term. The interpretation of the arguments is rule specific. The context \colorbox{blue!30}{$\Gamma$} of a step is a list $c_1 \dots c_l $ where each element $c_j$ is either a variable or a variable-term tuple denoted $x_j \mapsto t_j$. Therefore, steps with a non-empty context contain variables $x_j$ that appear in \colorbox{green!30}{$l_i$} and will be substituted by $t_j$. Proof rules \colorbox{purple!30}{$\mathcal{R}$} include theory lemmas and \texttt{resolution}, which corresponds to hyper-resolution on ground first-order clauses. 


\begin{table}[]
    \centering
    \begin{tabular}{ll}
    Rule & Description \\ \hline
    la\_generic & Tautologous disjunction of linear inequalities. \\
    lia\_generic & Tautologous disjunction of linear integer inequalities. \\
    la\_disequality & $t_1 \approx t_2 \lor \neg (t_1 \geq t_2) \lor \neg (t_2 \geq t_1)$ \\
    la\_totality & $t_1 \geq t_2 \lor t_2 \geq t_1$ \\
    la\_mult\_pos & $t_1 > 0 \land (t_2 \bowtie t_3) \rightarrow t_1 * t_2 \bowtie t_1 * t_3$ and $\bowtie \in \{<, >, \geq, \leq, =\}$ \\
    la\_mult\_neg & $t_1 < 0 \land (t_2 \bowtie t_3) \rightarrow t_1 * t_2 \bowtie_{inv} t_1 * t_3$ \\
    la\_rw\_eq & $(t \approx u) \approx (t \geq u \land u \geq t)$ \\
    comp\_simplify & Simplification of arithmetic comparisons. \\
    \end{tabular}
    \caption{Linear arithmetic rules in Alethe supported.}
    \label{table:linear-arith-rules}
\end{table}

We now have the key components to explain the guiding proof in the bottom part of \cref{lst:smtexampleproof}.
The proofs starts with \tt{assume} steps \tt{a0}, \tt{a1}, \tt{a2} that restate the assertions from the \textit{input problem} (\cref{lst:smtexampleproof}).
Step \tt{t1} transforms disjunction into clause by using the Alethe rule \tt{or}.
Steps \tt{t2} and \tt{t5} are tautologies introduced by the main rule \tt{la\_generic}
in Linear Real Arithmetic (LRA) logic and used also in LIA logic, where \colorbox{green!30}{$l_1, l_2,\dots, l_n$} represent linear inequalities.
These logics use closed linear formulas over the \lstinline[language=SMT,basicstyle=\ttfamily\footnotesize]{Real} signature and \lstinline[language=SMT,basicstyle=\ttfamily\footnotesize]{Int} respectively.
The \lstinline[language=SMT,basicstyle=\ttfamily\footnotesize]{Real} terms in \tt{LRA} logic are built over the Reals signature from SMT-LIB with free variables, but containing only linear atoms; that is
atoms of the form \lstinline[language=SMT,basicstyle=\ttfamily\footnotesize]{d}, \lstinline[language=SMT,basicstyle=\ttfamily\footnotesize]{(* d x)}, or \lstinline[language=SMT,basicstyle=\ttfamily\footnotesize]{(* x d)}  where \lstinline[language=SMT,basicstyle=\ttfamily\footnotesize]{x} is a free variable and  \lstinline[language=SMT,basicstyle=\ttfamily\footnotesize]{d} is an integer or rational constant.
Similarly, the \lstinline[language=SMT,basicstyle=\ttfamily\footnotesize]{Int} terms in \tt{LIA} logic are closed formulas built over the
Ints signature with free variables, but whose terms are also all linear, such that there is no occurrences of the function symbols \lstinline[language=SMT,basicstyle=\ttfamily\footnotesize]{*} (except variable multiplied by an \lstinline[language=SMT,basicstyle=\ttfamily\footnotesize]{Int} constant), \lstinline[language=SMT,basicstyle=\ttfamily\footnotesize]{/}, \lstinline[language=SMT,basicstyle=\ttfamily\footnotesize]{div}, \lstinline[language=SMT,basicstyle=\ttfamily\footnotesize]{mod}, and \lstinline[language=SMT,basicstyle=\ttfamily\footnotesize]{abs}.
A linear inequality is of term of the form

\begin{equation}
\sum_{i=0}^{n}c_i\times{}t_i + d_1\bowtie \sum_{i=n+1}^{m} c_i\times{}t_i + d_2
\label{eqn:inequality}
\end{equation}

where $\bowtie\;\in \{=, <, >, \leq, \geq\}$, where $m\geq n$, $c_i, d_1, d_2$ are either \lstinline[language=SMT,basicstyle=\ttfamily\footnotesize]{Int} or \lstinline[language=SMT,basicstyle=\ttfamily\footnotesize]{Real}
constants, and for each $i$ $c_i$ and $t_i$ have the same sort.
Checking the clause validity of \tt{t2} and \tt{t5} in \cref{lst:smtexampleproof}, amounts to checking the unsatisfiability of the system of linear equations (we provide more details in \cref{sssect:la-in-alethe}) e.g. $x < 3$ and $x = 2$ in \tt{t2}.
A coefficient for each inequality are pass as arguments e.g. $(\frac{1}{1},\frac{1}{1})$ in \tt{t2}.
Steps \tt{t3} (and \tt{t4}) applies the \colorbox{purple!30}{\texttt{resolution}} rule to the premises \tt{a1}, \tt{t2} (respectively \tt{t1} \tt{t3}).
Finally, the step \texttt{t6} concludes the proof by generating the empty clause $\bot$, concretely denoted as \kw{(cl)} in \cref{lst:smtexampleproof}.
Notice that the contexts \colorbox{blue!30}{$\Gamma$} of each step are all empty in this proof.

\subsubsection{Linear arithmetic in Alethe}
\label{sssect:la-in-alethe}

Proofs for linear arithmetic steps use a number of straightforward rules listed in \cref{table:linear-arith-rules}, such as \tt{la\_totality}: $(t_1 \leq t_2 \lor t_2 \geq t_1)$.
Simplification rules \tt{*\_simplify}, such as \tt{sum\_simplify}, transform arithmetic formulas by applying equivalence-preserving operations repeatedly until a fixed point is reached;
these operations are no more complex than constant folding.

Following our method to encode Alethe described in \cite{ColtellacciMD24}, the linear arithmetic tautology rules \tt{la\_disequality}, \tt{la\_totality} and \tt{la\_mult\_*} are encoded as lemmas in our embedding of Alethe in Lambdapi.
The simplification rule \tt{comp\_simplify} is encoded as a lemma for each rewrite case and applied multiple times.
We do not support the remaining \tt{*\_simplify} rules and the \tt{la\_tautology} rule in this work, primarily because cvc5 does not follow the Alethe standard for simplification step.
Instead, it extends the Alethe format with the RARE simplification rules \cite{rare}. As a result, cvc5 does not generate proofs using these standard rules for the SMT-LIB benchmarks.

A different approach is taken for the primary rules \tt{*\_generic},as they describe an algorithm.
While \tt{la\_generic rule} is primarily intended for LRA logic, it is also applied in LIA proofs when all variables in the (in)equalities are of integer sort.
A step of the rule \tt{la\_generic} represents a tautological clause of linear disequalities.  It can be checked by showing that the conjunction of
the negated disequalities is unsatisfiable. After the application of some strengthening rules, the resulting conjunction is unsatisfiable,
even if \lstinline[language=SMT,basicstyle=\ttfamily\footnotesize\upshape]{Int} variables are assumed to be \lstinline[language=SMT,basicstyle=\ttfamily\footnotesize\upshape]{Real} variables.
Although the rule may introduce rational coefficients, they often reduce to integers—as shown in \cref{lst:smtexampleproof}, where the coefficients are $(\frac{1}{1}, \frac{1}{1})$.
Cases where coefficients cannot be reduced to integers are rare in practice, however, we eliminate .
Let $\varphi_1,\dots, \varphi_n$ be linear inequalities and $a_1, \dots, a_n$ rational numbers, then a \tt{la\_generic} step has the general form

\[
\begin{matrix*}[c]
  i. & \ctxsep \quad & \varphi_1 , \dots , \varphi_n & \quad \tt{la\_generic}  & [a_1, \dots, a_n] \\
\end{matrix*}
\]

The constants $a_i$ are of sort \tt{Real}. To check the unsatisfiability of the negation of $\varphi_1, \dots, \varphi_n$ one performs the following steps for each literal. For each $i$, let $\varphi := \varphi_i$, $a := a_i$ and
we write $s1 \bowtie s2$ to denotes the left and right side of an inequality of \cref{eqn:inequality}.

\begin{enumerate}
    % \item If $\varphi = s_1 > s_2$, then let $\varphi := - s_1 \geq - s_2$.
    %   If $\varphi = s_1 \geq s_2$, then let $\varphi := - s_1 > - s_2$.
    %   If $\varphi = s_1 < s_2$, then let $\varphi := s_1 \geq s_2$.
    %   If $\varphi = s_1 \leq s_2$, then let $\varphi := s_1 > s_2$.
    \item If $\varphi =  \neg (s_1 < s_2)$  or $s_1 \geq s_2$, then let $\varphi := \neg(- s_1 \geq - s_2)$.
    If $\varphi =  \neg (s_1 \leq s_2)$ or $s_1 > s_2$, then let $\varphi := \neg(- s_1 > - s_2)$.
    If $\varphi = s_1 < s_2$, then let $\varphi := \neg(s_1 \geq s_2$).
    If $\varphi = s_1 \leq s_2$, then let $\varphi := \neg(s_1 > s_2$).
    This negates the literal. We want a canonical form that use only the operators $>, \geq$ and =.

    % \item If $\varphi = \neg (s_1 \bowtie s_2)$, then let $\varphi := s_1 \bowtie s_2$.
    
    % \item If $\varphi = s_1 < s_2$, then let $\varphi :=   - s_1 > - s_2$.
    %   If $\varphi = s_1 \leq s_2$, then let $\varphi :=  s_1 \geq - s_2$.
    %   We want a canonical form that use only the operators $>, \geq$ and =.

    \item Replace $\varphi = \sum_{i=0}^{n}c_i\times{}t_i + d_1 \bowtie \sum_{i=n+1}^{m} c_i\times{}t_i + d_2$  by $\sum_{i=0}^{n}c_i\times{}t_i - \sum_{i=n+1}^{m} c_i\times{}t_i
    \bowtie d_2 - d_1$.
    
    \item \label{la_generic:str}Now $\varphi$ has the form $s_1 \bowtie d$. If all
    variables in $s_1$ are integer sorted then replace $\bowtie d$ by $\bowtie \lceil d \rceil$,
    otherwise replace by $\bowtie \lfloor d\rfloor + 1$.

    % \begin{center}
    % \begin{tabular}{r|l|l}
    %     $\bowtie$  & If $d$ is an integer  & Otherwise \\
    %     \hline
    %     $>$        & $\geq d + 1$  & $\geq \lfloor d\rfloor + 1$  \\
    %     $\geq$     & $\geq d$      & $\geq \lfloor d\rfloor + 1$  \\
    % \end{tabular}
    % \end{center}

    \item If all variables of $\varphi$ are Int and coefficient $a_1 \dots a_n \in \mathbb{Q}$,
    then $a_i \coloneq a \times \mathit{lcd}(a_1 \dots a_n)$ where $\mathit{lcd}$ is the least common denominator of $[a_1 \dots a_n]$.
    
    \item If $\bowtie$ is $=$, then replace $\varphi$ by
    $\sum_{i=0}^{m}a\times{}c_i\times{}t_i = a\times{}d$, otherwise replace it by
    $\sum_{i=0}^{m}|a|\times{}c_i\times{}t_i \bowtie |a|\times{}d$.

    \item Finally, the sum of the resulting literals is trivially contradictory.
    The sum
    \[
        \sum_{k=1}^{n}\sum_{i=1}^{m}c_i^k*t_i^k \bowtie \sum_{k=1}^{n}d^k
    \]
  where $c_i^k$ and $t_i^k$ are the constant and term from the literal $\varphi_k$, and $d^k$ is the constant $d$ of $\varphi_k$.
  The operator $\bowtie$ is $=$ if all operators are $=$, $>$ if all are either $=$ or $>$, and $\geq$ otherwise. Finally, the sum on the left-hand side is $0$ and the right-hand side is $>0$ (or $\geq 0$ if $\bowtie$ is $>$).

\end{enumerate}

The step 1 has been added by our own, as the subsequent steps in the original algorithm are designed for $>$ and $\geq$ and do not clearly address how to handle $<$ and $\leq$.
Additionally, step 6  was added to ensure that our construction is independent of $\mathbb{Q}$.

\begin{example}
Consider the $\tt{la\_generic}$ step in the logic \tt{QF\_UFLIA} with the uninterpreted function symbol \lstinline[language=SMT,basicstyle=\ttfamily\footnotesize\upshape]|(f Int)|:
\begin{lstlisting}[language=SMT,label={lst:smtexampleinput}]
(step t11 (cl (not (<= f 0)) (<= (+ 1 (* 4 f)) 1))
  :rule la_generic :args (1/1 1/4))
\end{lstlisting} the algorithm run as follow in a natural deduction:
\begin{align}
&\vdash \neg (- f > 0),~ \neg(4f > 0) \label{eq:step2}\tag{Step 2}\\
&\vdash \neg (- f > 0),~ \neg(4f \geq 1) \label{eq:step3}\tag{Step 3}\\
&\text{Replace } a = [\frac{1}{1}, \frac{1}{4}] \text{ by } a = [4, 1] \label{eq:step4}\tag{Step 4}\\
&\vdash \neg (|4| * - f > |4| * 0 ), ~ \neg(|1| * 4f \geq |1| * 1) \label{eq:step5}\tag{Step 5} \\
&-4f + 4f \geq 1 \vdash \mathtt{False} \label{eq:step6}\tag{Step 6}
\end{align}
\todo[ac]{Je met $\vdash$ car je trouve que la dérivation ce comprend mieu de mon point de vue. Mais je peux l'enlever si trop de confusion est ajouté.}
Which sums to the contradiction  $0 \geq 1$. 
\end{example}

The \tt{lia\_generic} is structurally similar to \tt{la\_generic}, but omits the coefficients.
Since this rule can introduce a disjunction of arbitrary linear integer inequalities without any additional hints, proof checking is \emph{NP-complete} \cite{Schrijver:lia}.