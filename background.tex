\section{Background}
\label{sec:background}

\subsection{An Overview of Lambdapi}
\label{ssect:lambdapi-overview}

Lambdapi is an implementation of $\lambda\Pi$ modulo theory ($\lpm$) \cite{lambdapi}, an extension of the Edinburgh Logical Framework $\lambda\Pi$ \cite{lf} that is based on a simply typed $\lambda$-calculus with dependent types. $\lpm$ adds user-defined higher-order rewrite rules. Its syntax is given by
%
\begin{align*}
&\text{Universes}  &u &::= \tt{TYPE} ~|~ \tt{KIND} \\
&\text{Terms}   &t,v, A,B,C &::= c ~|~ x ~|~ u ~|~ \Pi\,x : A,\,B~|~ \lambda\,x : A,\,t ~|~t~v \\
&\text{Contexts}   &\Gamma &::= \langle \rangle ~|~ \Gamma, x : A \\
&\text{Signatures}  &\Sigma &::= \langle \rangle ~|~ \Sigma, c : C ~|~ \Sigma, c := t : C ~|~ \Sigma, t \hookrightarrow v 
\end{align*}
%
where $c$ is a constant, $x$ is a variable such that the sets of constants and variables are disjoint, and $C$ is a closed term. \emph{Universes} are constants used to verify if a type is well-formed -- more details can be found in \cite[\S 2.1]{lf}. $\Pi\,x : A,\,B$ is the dependent product (we write $A \rightarrow B$ when $B$ does not depend on $x$), $\lambda\,x : A,\,t$ is an abstraction, and  $t~v$ is an application. A \emph{(local) context} $\Gamma$ is a finite sequence of variable declarations $x:A$ introducing variables and their types.
A \emph{signature} $\Sigma$ representing the global context is a finite sequence of \emph{assumptions} $c : C$, indicating that constant $c$ is of type $C$, \emph{definitions} $c := t : C$, indicating that $c$ has value~$t$ and type~$C$, and \emph{rewrite rules} $t \hookrightarrow v$ such that $t = c~v_1 \dots v_n$ where $c$ is a constant.

The relation $\hookrightarrow_{\beta\Sigma}$ is generated by $\beta$-reduction and by the rewrite rules of $\Sigma$. The relation $\hookrightarrow_{\beta\Sigma}^*$ denotes the reflexive and transitive closure of $\hookrightarrow_{\beta\Sigma}$, and the relation $\equiv_{\beta\Sigma}$ (called \emph{conversion}) the reflexive, symmetric, and transitive closure of $\hookrightarrow_{\beta\Sigma}$. 
The relation $\hookrightarrow_{\beta\Sigma}$ must be confluent, i.e.,
whenever $t \hookrightarrow_{\beta\Sigma}^* v_1$ and $t \hookrightarrow_{\beta\Sigma}^* v_2$, there exists some $w$ such that $v_1 \hookrightarrow_{\beta\Sigma}^* w$ and $v_2 \hookrightarrow_{\beta\Sigma}^* w$, and it must preserve typing, i.e., 
whenever $\Gamma \vdash_\Sigma t: A$ and $t \hookrightarrow_{\beta\Sigma} v$ then $\Gamma \vdash_\Sigma v: A$ \cite{blanqui:LIPIcs.FSCD.2020.13}.

A Lambdapi typing judgment $\Gamma \vdash_\Sigma t : A$ asserts that term $t$ has type $A$ in the context $\Gamma$ and the signature $\Sigma$.
The typing rules of $\lpm$ are those of  $\lambda\Pi$ \cite[\S 2]{lf}, except for the rule (Conv), given below, that identifies types modulo~$\textcolor{orange}{\equiv_{\beta\Sigma}}$ instead of just modulo $\beta$-reduction. 
%% sm: removed figure to save some space
\[
    \begin{prooftree}
    \hypo{\Gamma, \vdash_\Sigma B: u}
    \hypo{\Gamma \vdash_\Sigma t: A}
    \hypo{\textcolor{orange}{A \equiv_{\beta\Sigma} B}}
    \infer3[(Conv)]{ \Gamma \vdash_\Sigma t: B }
    \end{prooftree}
\]

In our encoding presented in \cite{ColtellacciMD24},  we employ Tarski-style universes \cite{intuitype} where types are represented by elements of a base type and interpreted via a decoding function.
We define the constant $\prop: \type$ for the type of propositions and the decoding function $\pic: \prop \ra \type$ that maps each proposition to $\type$. This is necessary because Lambdapi does not support quantification over a variable of type $\type$: it is not possible to assign the type $\Pi X : \type,(X \ra \prop) \ra \prop$ to the universal quantifier $\forall$.
To circumvent this, we have the constant $\set : \type$ for the types of object-terms, and a decoding function ${\el}: \set \ra \type$ that embeds $\set$ into $\type$.
%% sm: added "universal" for a better line break
This allows us to define the universal quantifier as $\forall: \Pi x: \set, (\el~x \ra \prop) \ra \prop$.
To quantify over propositions, we further define a constant $o: \set$ and add the rewrite rule  $\el\, o \re \prop$.
This encoding is well-established in the literature on systems formalized in Lambdapi \cite{blanqui_et_al:LIPIcs.FSCD.2021.20}. 

\subsection{Alethe proofs}
\label{ssect:alethe}

The Alethe proof trace format \cite{alethespec} for SMT solvers comprises two parts: the trace language based on SMT-LIB and a collection of proof rules. Traces witness proofs of unsatisfiability of a set of constraints.
They are sequences $a_1 \dots a_m~t_1 \dots t_n$ where the assumptions $a_i$ correspond to the constraints of the original SMT problem being refuted, each $t_i$ is a clause inferred from previous elements of the sequence, and $t_n$ is $\bot$ (the empty clause).
In the following, we designate the SMT-LIB problem as the \emph{input problem}.

\lstinputlisting[language=SMT,label={lst:smtexampleinput},caption={Input problem.}]{example_lia.smt2}

\lstinputlisting[language=SMT,caption={Proof of unsatisfiability of the input problem of \protect{\cref{lst:smtexampleinput}}.},label={lst:smtexampleproof}]{proof_lia.alethe}

We will use the input problem shown in List.~\ref{lst:smtexampleinput} with its Alethe proof (found by cvc5) in List.~\ref{lst:smtexampleproof} as a running example to introduce Alethe concepts and illustrate our reconstruction of linear arithmetic step in Lambdapi.

\paragraph{Overview of the Alethe trace format.}

An Alethe proof trace inherits the declarations of the corresponding input problem. All symbols (sorts, functions, assertions, etc.) declared or defined in the input problem remain declared or defined.
Furthermore, the syntax for terms, sorts, and annotations uses the syntactic rules defined in SMT-LIB \cite[\S 3]{smtlib} and the SMT signature context defined in \cite[\S 5.1 and \S 5.2]{smtlib}.
In the following we will represent an Alethe step as

%\medskip

\renewcommand{\eqnhighlightshade}{35}

\begin{equation}
\label{eq:step}
\tag{\textcolor{purple}{1}}
\eqnmarkbox[indexClr]{node2}{i}. \quad \eqnmarkbox[blue]{node1}{\Gamma} ~\triangleright~ \eqnmarkbox[green]{node3}{l_1 \dots l_n} \quad (\eqnmarkbox[purple]{node4}{\mathcal{R}}~\eqnmarkbox[red]{node5}{p_1 \dots p_m})~\eqnmarkbox[orange]{node6}{[a_1 \dots a_r]}
\vspace{0.3em}
\annotate{below, left}{node2}{index}
\annotate{below, right}{node1}{context}
\annotate{below, right}{node3}{clause}
\annotate{below, right}{node4}{rule}
\annotate{below, right}{node5}{premises}
\annotate{below, right}{node6}{arguments}
\end{equation}

\bigskip

A step %\cref{eq:step} 
consists of an index \colorbox{indexClr!30}{$i$} $\in \mathbb{I}$ where $\mathbb{I}$ is a countable set of indices such as \kw{a0}, \kw{t1}), and a clause representing the disjunction of literals \colorbox{green!30}{$l_1, \dots, l_n$}.
Steps that are not assumptions are justified by a proof rule \colorbox{purple!30}{$\mathcal{R}$} that depends on a possibly empty set of premises $\{\colorbox{red!30}{$p_1, \dots,  p_m$}\} \subseteq \mathbb{I}$ that only contains earlier steps such that the proof forms
a directed acyclic graph. A rule might also depend on a list of arguments \colorbox{orange!30}{$a_1, \dots, a_r$} where each argument $a_i$ is either a term or a pair $(x_i, t_i)$ for a variable $x_i$ and a term $t_i$.
The interpretation of the arguments is rule-specific. The context \colorbox{blue!30}{$\Gamma$} of a step is a list $c_1, \dots, c_l $ where each element $c_j$ is either a variable or a variable-term tuple denoted $x_j \mapsto t_j$.
Therefore, steps with a non-empty context contain variables $x_j$ that appear in \colorbox{green!30}{$l_i$} and will be substituted by $t_j$. Proof rules \colorbox{purple!30}{$\mathcal{R}$} include theory lemmas and \texttt{resolution}, which corresponds to hyper-resolution on ground first-order clauses.

We now have the key components for explaining the proof in \cref{lst:smtexampleproof}.
The proofs starts with \tt{assume} steps \tt{a0}, \tt{a1}, \tt{a2} that restate the assertions from the input problem. % (\cref{lst:smtexampleproof}).
Step \tt{t1} transforms the disjunction \texttt{a0} into a clause by using the Alethe rule \tt{or}.
Steps \tt{t2} and \tt{t5} are tautologies introduced by the main rule \tt{la\_generic}
in Linear Real Arithmetic (LRA) logic and also used in LIA logic, where \colorbox{green!30}{$l_1, l_2,\dots, l_n$} represent linear inequalities.
% These logics use closed linear formulas over the signatures \lstinline[language=SMT,basicstyle=\ttfamily\footnotesize]{Real} and \lstinline[language=SMT,basicstyle=\ttfamily\footnotesize]{Int} respectively.
The \lstinline[language=SMT,basicstyle=\ttfamily\footnotesize]{Real} terms in LRA and LIA logic are built over the \lstinline[language=SMT,basicstyle=\ttfamily\footnotesize]{Real} and \lstinline[language=SMT,basicstyle=\ttfamily\footnotesize]{Int} signatures from SMT-LIB with free variables, but containing only linear atoms; that is
atoms of the form \lstinline[language=SMT,basicstyle=\ttfamily\footnotesize]{d}, \lstinline[language=SMT,basicstyle=\ttfamily\footnotesize]{(* d x)}, or \lstinline[language=SMT,basicstyle=\ttfamily\footnotesize]{(* x d)}  where \lstinline[language=SMT,basicstyle=\ttfamily\footnotesize]{x} is a free variable and  \lstinline[language=SMT,basicstyle=\ttfamily\footnotesize]{d} is an integer or rational constant.
% Similarly, the \lstinline[language=SMT,basicstyle=\ttfamily\footnotesize]{Int} terms in \tt{LIA} logic are closed formulas built over the \lstinline[language=SMT,basicstyle=\ttfamily\footnotesize]{Int} signature with free variables,
% but whose terms are also all linear, such that there are no occurrences of the function symbols \lstinline[language=SMT,basicstyle=\ttfamily\footnotesize]{*} (except a variable multiplied by an \lstinline[language=SMT,basicstyle=\ttfamily\footnotesize]{Int} constant),
% \lstinline[language=SMT,basicstyle=\ttfamily\footnotesize]{/}, \lstinline[language=SMT,basicstyle=\ttfamily\footnotesize]{div}, \lstinline[language=SMT,basicstyle=\ttfamily\footnotesize]{mod}, and \lstinline[language=SMT,basicstyle=\ttfamily\footnotesize]{abs}.
A linear inequality is an expression of the form

\begin{equation}
\sum_{i=0}^{n}c_i\times{}t_i + d_1\bowtie \sum_{i=n+1}^{m} c_i\times{}t_i + d_2
\label{eqn:inequality}
\end{equation}
%
where $\mathop{\bowtie} \mathrel{\in} \mathop{\{=, <, >, \leq, \geq\}}$, $m\geq n$, $c_i, d_1, d_2$ are either \lstinline[language=SMT,basicstyle=\ttfamily\footnotesize]{Int} or \lstinline[language=SMT,basicstyle=\ttfamily\footnotesize]{Real}
constants, and where $c_i$ and $t_i$ have the same sort for all $i$.
Checking the validity of the clauses \tt{t2} and \tt{t5} in List.~\ref{lst:smtexampleproof} amounts to checking the unsatisfiability of a system of linear (in)equations e.g. $x < 3$ and $x = 2$ in \tt{t2}.
Coefficients for each inequality are passed as arguments e.g. $(\frac{1}{1},\frac{1}{1})$ in \tt{t2}.
Steps \tt{t3} and \tt{t4} apply the \colorbox{purple!30}{\texttt{resolution}} rule to the premises \tt{a1}, \tt{t2} (respectively \tt{t1} and \tt{t3}).
Finally, the step \texttt{t6} concludes the proof by generating the empty clause $\bot$, denoted as \kw{(cl)} in List.~\ref{lst:smtexampleproof}.
Notice that the contexts \colorbox{blue!30}{$\Gamma$} of each step are all empty in this proof.

\begin{table}[tp]
  \centering
  \caption{Linear arithmetic rules in Alethe supported in our encoding.}

  \begin{tabular}{l@{\quad}l}
  Rule & Description \\ \hline
  la\_generic & Tautologous disjunction of linear inequalities \\
  lia\_generic & Tautologous disjunction of linear integer inequalities \\
  la\_disequality & $t_1 \approx t_2 \lor \neg (t_1 \geq t_2) \lor \neg (t_2 \geq t_1)$ \\
  la\_totality & $t_1 \geq t_2 \lor t_2 \geq t_1$ \\
  la\_mult\_pos & $t_1 > 0 \land (t_2 \bowtie t_3) \rightarrow t_1 * t_2 \bowtie t_1 * t_3$ and $\mathop{\bowtie} \mathrel{\in} \{<, >, \geq, \leq, \approx\}$ \\
  la\_mult\_neg & $t_1 < 0 \land (t_2 \bowtie t_3) \rightarrow t_1 * t_2 \bowtie_{\mathit{inv}} t_1 * t_3$ \\
  la\_rw\_eq & $(t \approx u) \approx (t \geq u \land u \geq t)$ \\
  comp\_simplify & Simplification of arithmetic comparisons \\
  % (define-rule arith-int-eq-elim ((t Int) (s Int)) (= t s) (and (>= t s) (<= t s)))
  arith-int-eq-elim & $(t \approx s) \rightarrow t \geq s \land t \leq s $\\
  % (define-rule arith-refl-geq ((t ?)) (>= t t) true)
  arith-refl-geq & $t \geq t \rightarrow \top$ \\
  % (define-rule arith-refl-lt ((t ?)) (< t t) false)
  arith-refl-lt & $t < t \rightarrow \bot$ \\
  % (define-rule arith-refl-leq ((t ?)) (<= t t) true)
  arith-refl-leq & $t \leq t \rightarrow \top$ \\
  % (define-rule arith-elim-leq ((t ?) (s ?)) (<= t s) (>= s t))
  arith-elim-leq & $t \leq s \rightarrow s \geq t$ \\
  % (define-rule arith-elim-gt ((t ?) (s ?)) (> t s) (not (<= t s)))
  arith-elim-gt & $t > s \rightarrow \neg (t \leq s)$ \\
  % (define-rule arith-leq-norm ((t Int) (s Int)) (<= t s) (not (>= t (+ s 1))))
  arith-leq-norm & $t \leq s \rightarrow \neg (t \geq s + 1)$ \\
  % (define-rule arith-geq-norm1 ((t ?) (s ?)) (>= t s) (>= (- t s) 0))
  arith-geq-norm1 & $t \geq s \rightarrow (t - s) \geq 0$ \\
  % (define-rule arith-geq-norm2 ((t ?) (s ?)) (>= t s) (<= (- t) (- s)))
  arith-geq-norm2 & $t \geq s \rightarrow - t \leq - s$ \\
  % (define-rule arith-geq-tighten ((t Int) (s Int)) (not (>= t s)) (>= s (+ t 1)))
  arith-geq-tighten & $\neg (t \geq s) \rightarrow s \geq t + 1$ \\
  arith-poly-norm & polynomial normalization \\
  evaluate & evaluate constant terms
  \end{tabular}

  \label{table:linear-arith-rules}
\end{table}

\paragraph{Linear arithmetic in Alethe.}
\label{sssect:la-in-alethe}

Proofs for linear arithmetic steps use a number of rules listed in Tab.~\ref{table:linear-arith-rules}, such as \tt{la\_totality} that asserts totality of the ordering~$\leq$. 
%\todo[sm]{Perhaps explain the meaning of $\rightarrow$ in these rules, i.e.\ logic vs.\ simplification rules.}
Besides arithmetical tautologies, the table also contains simplification rules, indicated with the symbol~$\rightarrow$.
Following our encoding of Alethe in Lambdapi as described in \cite{ColtellacciMD24}, the linear arithmetic rules \tt{la\_disequality}, \tt{la\_totality}, and \tt{la\_mult\_*} are implemented as lemmas.
We do not support the remaining arithmetic simplification rules, including the \tt{la\_tautology} rule from Alethe.
This omission is primarily due to the fact that cvc5 extends Alethe with the RARE simplification rules \cite{rare}, which it uses in place of the original ones.
Consequently, we support the RARE rules prefixed by \tt{arith-*}, as listed in Table~\ref{table:linear-arith-rules}, and we have selectively implemented those that appear in the proof traces of the benchmarks discussed in Sect.~\ref{sec:evaluation}.
In addition, support for the \tt{evaluate} rule is provided through the work described in Sect.~\ref{sec:encoding}, and support for \tt{arith-poly-norm} is realized through the normalization approach explained in Sect.~\ref{sec:lia-reconstruction}.

A different approach is taken for the rules \tt{la\_generic} and \tt{lia\_generic}, as they describe an algorithm.
While the \tt{la\_generic rule} is primarily intended for LRA logic, it is also applied in LIA proofs when all variables in the (in)equalities are of integer sort.
A step of the rule \tt{la\_generic} represents a tautological clause of linear inequalities.  It can be checked by showing that the conjunction of
the negated inequalities is unsatisfiable. After the application of some strengthening rules, the resulting conjunction is unsatisfiable,
even if \lstinline[language=SMT,basicstyle=\ttfamily\footnotesize\upshape]{Int} variables are assumed to be \lstinline[language=SMT,basicstyle=\ttfamily\footnotesize\upshape]{Real} variables.
Although the rule may introduce rational coefficients, they often reduce to integers—as shown in List.~\ref{lst:smtexampleproof}, where the coefficients are $(\frac{1}{1}, \frac{1}{1})$.
Cases where coefficients cannot be reduced to integers are rare in practice, however, we reduce them to integers by \emph{clearing denominators} with their least common denominator.
Let $\varphi_1,\dots, \varphi_n$ be linear inequalities or their negations, but different from $s_1 \approx s_2$ and $a_1, \dots, a_n$ rational numbers, then an \tt{la\_generic} step has the general form
%
\[
\begin{matrix*}[c]
  i. & \ctxsep \quad & \varphi_1 , \dots , \varphi_n & \quad \tt{la\_generic}  & [a_1, \dots, a_n] \\
\end{matrix*}
\]

The constants $a_i$ are of sort \tt{Real}. To check the unsatisfiability of the conjunction $\lnot \varphi_1 \land \ldots \land \lnot \varphi_n$, one performs the following steps for each literal. 
% For each $i$, let $\varphi := \varphi_i$, $a := a_i$ and
%We write $s1 \bowtie s2$ to denote the left- and right-hand sides of an inequality of the form \eqref{eqn:inequality}.

\begin{enumerate}
    \item If $\varphi_i$ is of the form $s_1 \geq s_2$ or $\neg (s_1 < s_2)$, then let $\psi_i = s_2 > s_1$.
    If $\varphi_i$ is of the form $s_1 > s_2$ or $\neg (s_1 \leq s_2)$, then let $\psi_i = s_2 \geq s_1$.
    If $\varphi_i$ is of the form $s_1 < s_2$ or $\neg (s_1 \geq s_2)$, then let $\psi_i = s_1 \geq s_2$.
    If $\varphi_i$ is of the form $s_1 \leq s_2$ or $\neg (s_1 > s_2)$, then let $\psi_i = s_1 > s_2$.
    If $\varphi_i$ is of the form $\neg (s_1 \approx s_2)$, then let $\psi_i = s_1 \approx s_2$.
    This step produces a positive literal that is equivalent to $\lnot \varphi_i$ and that only contains the operators $>$, $\geq$, and $\approx$.
    % The goal is to produce a canonical form that uses only the operators $>$, $\geq$, and $=$.\\
    % Note that equalities are assumed to already be in negated form.


    \item Replace $\psi_i = \sum_{j=0}^{k_i}c_j^i\times{}t_j^i + d_1^i \bowtie \sum_{j=k_i+1}^{m_i} c_j^i\times{}t_j^i + d_2^i$ by the literal $\left(\sum_{j=0}^{k_i}c_j^i\times{}t_j^i\right) - \left(\sum_{j=k_i+1}^{m_i} c_j^i\times{}t_j^i\right)
    \bowtie d_2^i - d_1^i$.
  
    \item \label{la_generic:str}Now $\psi_i$ has the form $s_1^i \bowtie d^i$. If all
    variables in $s_1^i$ are integer-sorted then replace $s_1^i > d^i$ by $s_1^i \geq \lfloor d^i \rfloor + 1$,
    respectively, replace $s_1^i \geq d^i$ by $s_1^i \geq \lfloor d^i \rfloor + 1$ if $d$ is not an integer.

    \item If all variables of $\psi_i$ are integer-sorted and the coefficients $a_1 \dots a_n$ are in $\mathbb{Q}$,
    then $a_i \coloneq a_i \times \mathit{lcd}(a_1 \dots a_n)$ where $\mathit{lcd}$ is the least common denominator of $\{a_1, \dots, a_n\}$.
    
    \item If $\bowtie$ is $\approx$, then replace $\psi_i$ by
    $\sum_{j=0}^{m_i}a_i\times{}c_j^i\times{}t_j^i = a_i\times{}d^i$, otherwise replace $\psi_i$ by
    $\sum_{j=0}^{m_i}|a_i|\times{}c_j^i\times{}t_j^i \bowtie |a_i|\times{}d^i$.

    \item Finally, the sum of the resulting literals is trivially contradictory,
    \[
        \sum_{i=1}^{n}\sum_{j=1}^{m_i} c_j^i*t_j^i \bowtie \sum_{i=1}^{n}d^i
    \]
  The operator $\bowtie$ is $=$ if all operators are $=$, $>$ if all are either $\approx$ or $>$ (but at least one operator is different from $\approx$), and $\geq$ otherwise. Finally, the sum on the left-hand side is $0$ and the right-hand side is $>0$ (or $\geq 0$ if $\bowtie$ is $>$).

\end{enumerate}

The above algorithm is adapted from the Alethe specification \cite{alethespec}, except that we clarified step 1: the subsequent steps in the original algorithm are designed for $>$ and $\geq$ and do not clearly address how to handle $<$ and $\leq$.
Additionally, we added step 4 in order to ensure that our construction is independent of $\mathbb{Q}$.

\begin{example}
Consider the following $\tt{la\_generic}$ step in the logic \tt{QF\_UFLIA} with the uninterpreted function symbol \lstinline[language=SMT,basicstyle=\ttfamily\upshape]|(f Int)|:
\begin{lstlisting}[language=SMT,label={lst:lageneric-example}]
(step t11 (cl (not (<= f 0)) (<= (+ 1 (* 4 f)) 1))
  :rule la_generic :args (1/1 1/4))
\end{lstlisting} 
%
The algorithm then performs the following steps: %\todo[sm]{adapted to the new description of the algorithm}
\begin{align}
&- f \geq 0,~ 4 \times f > 0 \label{eq:step2}\tag{Steps 1 and 2}\\
&- f \geq 0,~ 4 \times f \geq 1 \label{eq:step3}\tag{Step 3}\\
&\text{Replace arguments } [\frac{1}{1}, \frac{1}{4}] \text{ by } [4, 1] \text{ due to clearing denominators} \label{eq:step4}\tag{Step 4}\\
&|4| \times (- f) \geq |4| \times 0, ~ |1| \times 4 \times f \geq |1| \times 1) \label{eq:step5}\tag{Step 5} \\
&-4 \times f + 4 \times f \geq 1 \label{eq:step6}\tag{Step 6}
\end{align}
Which simplifies to the contradiction  $0 \geq 1$.
\label{ex:la_generic_example_red}
\end{example}

\begin{remark}
The operator \lstinline[language=SMT,basicstyle=\ttfamily\footnotesize]{to_real} is used in the \tt{LIA} theory to embed integers into the reals.
As a result, a proof for a problem formulated in \tt{LIA} may involve reasoning over real numbers.
Since our approach does not support the \lstinline[language=SMT,basicstyle=\ttfamily\footnotesize\upshape]{Real} theory, we do not attempt to reconstruct such proofs and instead let the translation process fail in this case.
\end{remark}

\subsubsection{Quantifiers in Linear Arithmetic.}

The Alethe format defines rules for quantifier instantiation, Skolemization, substitution, and other manipulations of bound variables.
For instance, the rule \tt{forall\_inst} is used to express quantifier instantiation. It produces a unit clause with a formula of the form $(\neg \forall \bar{x},\, \varphi) \lor (\varphi[\bar{t}])$,
where $\varphi$ is a term containing the free variables $\bar{x}$, and each term $t$ is a ground term of the same sort as the corresponding variable $x$.
Accordingly, an Alethe proof will first instantiate the quantifier to produce a ground term, enabling subsequent arithmetic steps such as those involving \tt{la\_generic} or \tt{lia\_generic} to operate solely on ground terms.
