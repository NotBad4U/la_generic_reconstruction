\section{Elaborating lia\_generic steps}
\label{sec:elaboration-lia}

\todo[sm]{Reviewer 1 asks us to improve the presentation.}
The rule \tt{lia\_generic} is similar to \tt{la\_generic}, but the SMT solver does not provide the coefficients,
i.e.\ \colorbox{orange!30}{$[a_1 \dots a_r]$} is empty.
%Since this rule can introduce a disjunction of arbitrary linear integer inequalities without any additional hints, proof checking is \emph{NP-complete} \cite{Schrijver:lia}.
We decided to leverage the elaboration process of \tt{lia\_generic} performed by Carcara, as doing otherwise would require implementing Fourier-Motzkin elimination for integers, as done in \cite{micromega,omegatest}, hence reimplementing work that was already done by the solver.

Carcara considers $\tt{lia\_generic}$ steps as holes in the proof, given that ``their checking is as hard as solving'' \cite[\S 3.2]{carcara}.
To address this, Carcara invokes an external SMT solver, such as cvc5, or any tool capable of reading SMT-LIB input and producing Alethe proofs, and attempts to generate Alethe proofs that avoid using $\tt{lia\_generic}$. 
Suppose the resulting proof still includes a $\tt{lia\_generic}$ step. In that case, Carcara repeats the process for up to three iterations, merging the final results if a complete proof without any $\tt{lia\_generic}$ steps is eventually found.
The proof is then imported and validated, replacing the original step.

\begin{example}[Sketch of lia\_generic elaboration]

Consider a step $S$ (List.~\ref{lst:pb_lia}) concluding the clause $\neg l_1 \lor \dots \neg l_n$ where all $l_i$ are inequalities and proved by $\tt{lia\_generic}$ rule.
    
\begin{lstlisting}[language=SMT,caption={Elaborated proof},label={lst:pb_lia}]
    (step S (cl (not l1) ... (not ln)) :rule lia_generic)
\end{lstlisting}
%
Carcara will generate an SMT-LIB problem asserting $l_1$, \dots, $l_n$ and invoke the solver cvc5 on it, expecting an Alethe proof \changed{sm}{of the unsatisfiability of $l_1 \land \dots \land l_n$}
that does not use \tt{lia\_generic}. Carcara will check this subproof and then replace the original step by a proof of the form shown in List.~\ref{lst:elab_lia}.
\todo[sm]{I don't understand the reference to \texttt{S.t\_m+1} since that step appears nowhere.}

\begin{lstlisting}[language=SMT,caption={Elaboration of \tt{lia\_generic}},label={lst:elab_lia}]
(anchor :step S.t_m+1)
(assume S.h_1 l1)
...
(assume S.h_n ln)
...
(step S.t_m (cl false) :rule ...)
(step t.t_p (cl (not l1) ... (not ln) false) :rule subproof)
(step t.t_q (cl (not false)) :rule false)
(step S (cl (not l1) ... (not ln)) :rule resolution :premises (S.t_p S.t_q))
\end{lstlisting}

In List.~\ref{lst:elab_lia}, steps $\tt{S.h\_1}$ until $\tt{S.t\_m}$ are imported from the cvc5 proof.
As a result the $\tt{lia\_generic}$ step $\tt{S}$ in the original proof (List.~\ref{lst:pb_lia}) will have been replaced by a detailed justification whose correctness can be independently established by Carcara.

\end{example}
% In the next section, we first present an overview of our embedding of Alethe in Lambdapi, and then our automatic procedure to reconstruct $\tt{la\_generic}$ step that appear in LIA problem.
