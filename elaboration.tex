\section{Elaborating lia\_generic steps}
\label{sec:elaboration-lia}

The rule \tt{lia\_generic} is similar to \tt{la\_generic}, but omits the coefficients,
i.e.\ \colorbox{orange!30}{$[a_1 \dots a_r]$} is empty..
%Since this rule can introduce a disjunction of arbitrary linear integer inequalities without any additional hints, proof checking is \emph{NP-complete} \cite{Schrijver:lia}.
We decided to leverage the elaboration process of \tt{lia\_generic} performed by Carcara, as doing otherwise would require implementing Fourier-Motzkin elimination for integers, as done in \cite{micromega,omegatest}, hence reimplementing work that was already done by the solver.

Carcara considers $\tt{lia\_generic}$ steps as holes in the proof, given that ``their checking is as hard as solving'' \cite[\S 3.2]{carcara}.
To address this, Carcara leverages an external tool that reformulates each \tt{lia\_generic} step as a separate problem and produces Alethe proofs not containing \tt{lia\_generic} steps.
The proof is then imported and validated, replacing the original step. 
%% sm: I don't see why this remark is relevant for this paper?
%However, at present, Carcara only uses the solver cvc5 for performing this task.
Thus, the step
% In detail, the elaboration method, when encountering a \tt{lia\_generic} step 
%
\begin{lstlisting}[language=SMT]
    (step S (cl (not l1) ... (not ln)) :rule lia_generic)
\end{lstlisting}
%
concluding the clause $\neg l_1 \lor \dots \neg l_n$ where all $l_i$ are inequalities, generates an SMT-LIB problem asserting $l_1$, \dots, $l_n$ and invokes the solver cvc5 on it, expecting an Alethe proof $\pi : (l_1 \land \dots \land l_n) \ra \bot$
that does not use \tt{lia\_generic}. Carcara will check this subproof and then replace the original step by a proof of the form

\begin{lstlisting}[language=SMT,caption={Elaboration of \tt{lia\_generic}},label={lst:elab_lia}]
(anchor :step S.t_m+1)
(assume S.h_1 l1)
...
(assume S.h_n ln)
...
(step t.t_m (cl false) :rule ...)
(step t.t_p (cl (not l1) ... (not ln) false) :rule subproof)
(step t.t_q (cl (not false)) :rule false)
(step S (cl (not l1) ... (not ln)) :rule resolution :premises (S.t_p S.t_q))
\end{lstlisting}

% In the next section, we first present an overview of our embedding of Alethe in Lambdapi, and then our automatic procedure to reconstruct $\tt{la\_generic}$ step that appear in LIA problem.
