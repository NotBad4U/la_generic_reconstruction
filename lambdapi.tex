\section{Encoding of Linear Integer Arithmetic in Lambdapi}
\label{sec:encoding}

\subsection{Encoding Integers in Lambdapi}


\begin{figure}
\centering
\begin{align*}\label{eq:eq1}
&\bb{P}: \type & &\bb{Z}: \type   & &\tt{Comp}: \type & &\bb{B}: \type \\
&|~\tt{H} : \bb{P} & &|~\tt{Z0}: \bb{Z} & &|~\tt{Eq}: \tt{Comp} & &|~\tt{true}: \bb{B} \\
&|~\tt{O}: \bb{P} \ra \bb{P} & &|~\tt{ZPos}: \bb{P} \ra \bb{Z} & &|~\tt{Lt}: \tt{Comp} & &|~\tt{false}: \bb{B} \\
&|~\tt{I}: \bb{P} \ra \bb{P} & &|~\tt{ZNeg}: \bb{P} \ra \bb{Z} & &|~\tt{Gt}: \tt{Comp} & &\\
&\tt{pos}: \set & &\tt{int}: \set & &\tt{comp}: \set & &\tt{bool}: \set \\
&\el~\tt{pos} \re \bb{P} & &\el~\tt{int} \re \bb{Z} & &\el~\tt{comp} \re \tt{Comp} & &\el~\tt{bool} \re \bb{B}
\end{align*}
\caption{Type definitions for binary positive number, integers, comparison and Booleans.}
\label{fig:sorts-constructors}
\end{figure}

The definition we use of integers in Lambdapi in \cref{fig:sorts-constructors} follows a common encoding found in many other theories, including the one adopted in the Rocq standard library \cite{Rocq-refman}.
First, the type $\bb{P}$  is an inductive type representing strictly positive integers in binary form.
Starting from 1 (represented by the constructor \tt{H}), one can add a new least significant digit via the constructor \tt{O} (digit 0) or the constructor \tt{I} (digit 1). 
The type $\bb{Z}$ represents integers in binary form.
An integer is either zero (with constructor \tt{Z0}) or a strictly positive number \tt{Zpos} (coded as a $\bb{P}$) or a strictly negative number \tt{Zneg}.
We make use of Lambdapi's \lstinline[language=Lambdapi,basicstyle=\ttfamily\footnotesize]|builtin| mechanism to enable decimal notation for numeric values, for instance, writing $2$ for $\ZPos (\mathop{\tt{O}} \tt{H})$.
to convert natural representations of values into elements of $\bb{Z}$.
%
We also introduce enumeration types \tt{Comp} and $\bb{B}$ representing comparison operators and Booleans. 
%
To enable quantification over elements of these types, we introduce constants such as $\tt{int}: \set$ that represent codes for these types along with a rule for rewriting codes to their corresponding types, for example $\el~\tt{int} \re \bb{Z}$.
Figure~\ref{fig:arith-ops} introduces operations on these types, including addition $(+)$ and comparison $(\doteq)$ over $\bb{Z}$. The functions \texttt{add} and \texttt{sub} implement addition and subtraction for binary positive numbers, respectively.
We have also defined the operations of multiplication ($\mathbin{*}$) and subtraction ($\mathbin{-}$) over $\bb{Z}$. A detailed description of their implementation is provided in \cref{app:lambdapi-func-def}.
The function $\texttt{cmp}: \bb{P} \to \bb{P} \to \texttt{Comp}$ performs a comparison between two binary positive numbers. 
These operators are defined by rewriting rules that we will refer to as $\ra_\bb{Z}$ and $\ra_\bb{P}$ (for operations on $\bb{Z}$ and $\bb{P}$) in the following sections.
The confluence of these rewriting rules has been proven using CSI \cite{CSI}. A detailed proof of confluence can be found in \cref{app:confluence-int-pos}.

\begin{figure}
\centering
\begin{minipage}[t]{0.48\textwidth}
\begin{align*}
&+: \bb{Z} \ra \bb{Z} \ra \bb{Z} \\
& \tt{Z0} + y \re y \\
& x + \tt{Z0} \re x \\
& (\tt{Zpos x}) + (\tt{Zpos y}) \re (\tt{Zpos}~(\tt{add}~x~y))  \\
& (\tt{Zpos x}) + (\tt{Zneg y}) \re (\tt{sub}~x~y)  \\
& (\tt{Zneg x}) + (\tt{Zpos y}) \re (\tt{sub}~y~x)  \\
& (\tt{Zneg x}) + (\tt{Zneg y}) \re \tt{Zneg}~(\tt{add}~x~y)
\end{align*}
\hfill
\end{minipage}
\begin{minipage}[t]{0.48\textwidth}
\begin{align*}
&\doteq : \bb{Z} \ra \bb{Z} \ra \tt{Comp} \\
& \tt{Z0} \doteq \tt{Z0} \re \tt{Eq} \\
& \tt{Z0} \doteq \tt{Zpos}~\_ \re \tt{Lt} \\
& \tt{Z0} \doteq \tt{Zneg}~\_ \re \tt{Gt} \\
& \tt{Zpos}~\_ \doteq \tt{Z0} \re \tt{Gt} \\
& \tt{Zpos}~p \doteq \tt{Zpos}~q \re \tt{cmp}~p~q \\
& \tt{Zpos}~\_ \doteq \tt{Zneg}~\_ \re \tt{Gt} \\
& \tt{Zneg}~\_ \doteq \tt{Z0} \re \tt{Lt} \\
& \tt{Zneg}~\_ \doteq \tt{Zpos}~\_ \re \tt{Lt} \\
& \tt{Zneg}~p \doteq \tt{Zneg}~q \re \tt{cmp}~q~p
\end{align*}
\end{minipage}
\noindent
\[
\begin{array}{l@{\hspace{4em}}l@{\hspace{4em}}l}
\begin{aligned}
  &\tt{isEq} : \tt{Comp} \ra \bb{B} \\
  &\tt{isEq}~\tt{Eq} \re \tt{true} \\
  &\tt{isEq}~\tt{Lt} \re \tt{false} \\
  &\tt{isEq}~\tt{Gt} \re \tt{false} 
\end{aligned}
&
\begin{aligned}
  &\tt{isLt} : \tt{Comp} \ra \bb{B} \\
  &\tt{isLt}~\tt{Eq} \re \tt{false} \\
  &\tt{isLt}~\tt{Lt} \re \tt{true} \\
  &\tt{isLt}~\tt{Gt} \re \tt{false} \\
\end{aligned}
&
\begin{aligned}
  &\tt{isGt} : \tt{Comp} \ra \bb{B} \\
  &\tt{isGt}~\tt{Eq} \re \tt{false} \\
  &\tt{isGt}~\tt{Lt} \re \tt{false} \\
  &\tt{isGt}~\tt{Gt} \re \tt{true} 
\end{aligned}
\end{array}
\]
\noindent
\begin{align*}
&\leq: \bb{Z} \ra \bb{Z} \ra \prop  \coloneq \lambda x,\lambda y, \neg (\tt{istrue}(\tt{isGt}(x \doteq y))) & &\tt{istrue} : \bb{B} \ra \prop \\
&<: \bb{Z} \ra \bb{Z} \ra \prop  \coloneq \lambda x,\lambda y, (\tt{istrue}(\tt{isLt}(x \doteq y))) & &\tt{istrue}~\tt{true} \re \top \\
&\geq: \bb{Z} \ra \bb{Z} \ra \prop  \coloneq \lambda x,\lambda y, \neg (x < y) & &\tt{istrue}~\tt{false} \re \bot \\
&>: \bb{Z} \ra \bb{Z} \ra \prop  \coloneq \lambda x,\lambda y, \neg (x \leq y) & &
\end{align*}
\caption{Definitions for operators over $\bb{Z}$.}
\label{fig:arith-ops}
\end{figure}

Finally, we define inequality operators for $\bb{Z}$ as binary predicates by reducing them to the decidable comparison $\doteq$. They reduce to $\top$, $\bot$ (or negated) by applying rules of $\ra_\bb{Z}$ and $\ra_\bb{P}$.
For example, $1 < 2 \hookrightarrow \tt{istrue}(\tt{isLt}(1 \doteq 2)) \hookrightarrow \tt{istrue}(\tt{isLt}(\tt{Lt})) \hookrightarrow \tt{istrue}(\tt{true}) \hookrightarrow \top$, with $1 = \mathop{\tt{Zpos}} \tt{H}$ and $2 = \mathop{\tt{Zpos}} (\mathop{\tt{O}} \tt{H})$.

\subsection{Functions used in the translation}

We now outline the encoding of arithmetic expressions from SMT-LIB \cite[\S 5.2.1]{smtlib}. This extends the approach introduced in \cite{ColtellacciMD24} to handle arithmetic constructs.
To avoid notational conflicts with the Lambdapi signature $\Sigma$, we denote the set of SMT-LIB sorts as $\Theta^\mathcal{S}$, the set of function symbols $\Theta^\cal{F}$, and the set of variables $\Theta^\cal{X}$.
Our translation is based on the following functions:

\begin{itemize}
\item $\cal{D}$ translates declarations of sorts and functions in $\Theta^\cal{S}$ and $\Theta^\mathcal{F}$ into constants,
\item $\cal{S}$ maps sorts to $\Sigma$ types,
\item $\cal{E}$ translates SMT expression to $\lpm$ terms,
\item $\cal{C}$ translates a list of commands  $c_1 \dots c_n$ of the form $i.~\Gamma \triangleright~\varphi~(\mathcal{R}~P)[A]$ to typing judgments $\Gamma \vdash_\Sigma i := M: N$.
\end{itemize}

\begin{definition}%[Function $\mathcal{D}$ translating SMT sort and function symbol declarations]
For each sort symbol $s$ with arity $n$ in $\Theta^\cal{S}$ we create a constant $s: \set \ra^1 \dots \ra^{n} \set$.
For each function symbol $(f~\sigma^+)$ in $\Theta^\cal{F}$ we create a constant $f: \cal{S}(\sigma^+)$.
\end{definition}

\begin{definition}%[Function $\mathcal{S}$ translating sorts of expression] 
  The function $\mathcal{S}(s)$ mapping SMT sorts to $\lpm$ types is defined as follows.
  \begin{itemize}
    \item Case $s = \textbf{Bool}$, then $\Sort{s} = \el\,o$,
    \item Case $s = \textbf{Int}$, then $\Sort{s} = \el~\texttt{int}$,
    \item Case $s = \sigma_1\,\sigma_2 \dots \sigma_n$ then $\Sort{s} = \el{} (\mathcal{S}(\sigma_1) \leadsto \dots \leadsto \mathcal{S}(\sigma_n))$,
    \item otherwise $\Sort{s} = \el\, \mathcal{D}(s)$.
  \end{itemize}
\end{definition}

\begin{definition}%[Function $\mathcal{E}$ translating SMT expressions]
  The definition of $\E{e}$ mapping expressions from SMT to $\lpm$ is as follows.
  \begin{itemize}
  \setlength{\parskip}{0pt}
  \item Case e $= (+~t_1\dots~t_n)$, then $\E{e} = \E{t_1} + ~\dots~ +\E{t_n}$.
  \item Case e $= (*~t_1\dots~t_n)$, then $\E{e} = \E{t_1} * ~\dots~ *\E{t_n}$.
  \item Case e $= (-t)$, then $\E{e} = \mathop{\sim} \E{t_1}$.
  \item Case e $= (-~t_1\dots~t_n)$, then $\E{e} = \E{t_1} - ~\dots~ - \E{t_n}$.
  \item Case e $= (\mathop{\bowtie} t_1~t_2)$ then $\E{e} = (\E{t_1} \bowtie \E{t_2})$ with $\mathop{\bowtie} = \{\approx, <,>,\leq,\geq \}$.
  \item Case $e = (x: \sigma )$ with $x \in \Theta^\mathcal{X}$ a sorted variable, then $\E{e} = x: \cal{S}(\sigma)$.
  \item Case $e = (c: \mathit{\tt{Int}})$  a constant integer value, then $\E{e} = c$.
  \end{itemize}
\end{definition}
% \lstinline[basicstyle=\ttfamily,language=SMT]|Int|
We define the function $\pid: \tt{Clause} \ra \type$ \cite[\S 3]{ColtellacciMD24}, mapping each clause~$c$ to the type $\pid~c$ of its proofs.
Clauses are encoded in the type $\tt{Clause}$%: \type$ 
as lists with the constructors $\nil: \tt{clause}$ representing the empty clause and $\veedot: \prop \ra \tt{Clause} \ra \tt{Clause}$.
The function $\cal{C}$ encodes each step by invoking the functions $\cal{D},\cal{S}$, and $\cal{E}$, and provides a proof term corresponding to the Alethe rule applied at that step.

\begin{example}
The translation of the steps t2 and t5 in \cref{lst:smtexampleproof} with the input problem definitions give us:
\begin{lstlisting}[language=Lambdapi,mathescape=true]
symbol x: El int;
symbol y: El int;
...
opaque symbol t2: $\pid$ (¬ (3 < x)  ⟇ ¬ (x = 2) ⟇ ▩) ≔ begin ... end;
opaque symbol t5: $\pid$ (¬ (x + y < 1) ⟇ (¬ (x = 2))  ⟇ (¬ (0 = y)) ⟇ ▩) ≔ begin ... end;
...
\end{lstlisting}
\end{example}

The proof terms generated by $\mathcal{C}$ for steps \texttt{t2} and \texttt{t5} must faithfully represent the algorithm presented in \cref{sssect:la-in-alethe}.
While steps 1 through 5 of the algorithm correspond to explicit rewriting steps, the final step (step 6) — which involves summing all inequalities — represents a multi-step rewriting sequence.
This sequence reduces the initial sum to a comparison between constants, which can be evaluated to a Boolean (e.g. $0 > 1 \re \bot$), serving as the conclusion of the reduction to prove a contradiction.


The reflection technique introduced by \cite{reflection-origin-coq} leverages the reduction system of the proof assistant to produce an efficient decidable automatic
procedure for solving arithmetic goals over $\bb{Q}$, $\bb{R}$, and $\bb{Z}$. We followed this approach to implement our decision procedure for evaluating inequality.
In the following section, we describe the corresponding extension of the function $\mathcal{C}$ defined in \cite{ColtellacciMD24}.
