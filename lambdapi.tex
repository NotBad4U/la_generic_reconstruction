\section{Reconstruction of \tt{la\_generic} step for LIA logic}
\label{sect:recon-lambdapi}

\subsection{An Overview of Lambdapi}
\label{ssect:lambdapi-overview}

Lambdapi is an implementation of $\lambda\Pi$ modulo theory ($\lpm$) \cite{lambdapi}, an extension of the Edinburgh Logical Framework $\lambda\Pi$ \cite{lf}, a simply typed $\lambda$-calculus with dependent types. $\lpm$ adds user-defined higher-order rewrite rules. Its syntax is given by
%
\begin{align*}
&\text{Universes}  &u &::= \type ~|~ \kind \\
&\text{Terms}   &t,v, A,B,C &::= c ~|~ x ~|~ u ~|~ \Pi\,x : A,\,B~|~ \lambda\,x : A,\,t ~|~t~v \\
&\text{Contexts}   &\Gamma &::= \langle \rangle ~|~ \Gamma, x : A \\
&\text{Signatures}  &\Sigma &::= \langle \rangle ~|~ \Sigma, c : C ~|~ \Sigma, c := t : C ~|~ \Sigma, t \hookrightarrow v 
\end{align*}
%
where $c$ is a constant and $x$ is a variable  (ranging over disjoint sets), $C$ is a closed term. \emph{Universes} are constants used to verify if a type is well-formed -- more details can be found in \cite[\S 2.1]{lf}. $\Pi\,x : A.\,B$ is the dependent product, and we write $A \rightarrow B$ when $x$ does not appear free in $B$, $\lambda\,x : A.\,t$ is an abstraction, and  $t~v$ is an application. A \emph{(local) context} $\Gamma$ is a finite sequence of variable declarations $x:A$ introducing variables and their types.
A \emph{signature} $\Sigma$ representing the global context is a finite sequence of \emph{assumptions} $c : C$, indicating that constant $c$ is of type $C$, \emph{definitions} $c := t : C$, indicating that $c$ has the value $t$ and type $C$, and \emph{rewrite rules} $t \hookrightarrow v$ such that $t = c~v_1 \dots v_n$ where $c$ is a constant.

The relation $\hookrightarrow_{\beta\Sigma}$ is generated by $\beta$-reduction and by the rewrite rules of $\Sigma$. The relation $\hookrightarrow_{\beta\Sigma}^*$ denotes the reflexive and transitive closure of $\hookrightarrow_{\beta\Sigma}$, and the relation $\equiv_{\beta\Sigma}$ (called \emph{conversion}) the reflexive, symmetric, and transitive closure of $\hookrightarrow_{\beta\Sigma}$. 
The relation $\hookrightarrow_{\beta\Sigma}$ must be confluent, i.e.,
whenever $t \hookrightarrow_{\beta\Sigma}^* v_1$ and $t \hookrightarrow_{\beta\Sigma}^* v_2$, there exists a term $w$ such that $v_1 \hookrightarrow_{\beta\Sigma}^* w$ and $v_2 \hookrightarrow_{\beta\Sigma}^* w$, and it must preserve typing, i.e., 
whenever $\Gamma \vdash_\Sigma t: A$ and $t \hookrightarrow_{\beta\Sigma} v$ then $\Gamma \vdash_\Sigma v: A$ \cite{blanqui:LIPIcs.FSCD.2020.13}.

A Lambdapi typing judgment $\Gamma \vdash_\Sigma t : A$ asserts that term $t$ has type $A$ in the context $\Gamma$ and the signature $\Sigma$.
The typing rules of $\lpm$ are the one of  $\lambda\Pi$ \cite[\S 2]{lf}, except for the rule (Conv) where it use the version of \cref{fig:lp-typing-rules} that identifies types modulo~$\textcolor{orange}{\equiv_{\beta\Sigma}}$ instead of just modulo $\beta$-reduction. 

\begin{figure}
    \begin{center}
    \begin{prooftree}
    \hypo{\Gamma, \vdash_\Sigma B: u}
    \hypo{\Gamma \vdash_\Sigma t: A}
    \hypo{\textcolor{orange}{A \equiv_{\beta\Sigma} B}}
    \infer3[(Conv)]{ \Gamma \vdash_\Sigma t: B }
    \end{prooftree}
    \end{center}
    \caption{(Conv) rule in $\lpm$}
    \label{fig:lp-typing-rules}
  \end{figure}

We now provide an overview of the encoding of Alethe linear integers arithmetic in Lambdapi.

\subsection{Encoding of Integers in Lambdapi}


\begin{figure}
\centering
\begin{align*}\label{eq:eq1}
&\bb{Z}: \type & &\bb{P}: \type  & &\tt{Comp}: \type & &\bb{B}: \type \\
&|~\tt{Z0}: \bb{Z} & &|~\tt{H} : \bb{P} & &|~\tt{Eq}: \tt{Comp} & &|~\tt{true}: \bb{B} \\
&|~\tt{ZPos}: \bb{P} \ra \bb{Z} & &|~\tt{O}: \bb{P} \ra \bb{P} & &|~\tt{Lt}: \tt{Comp} & &|~\tt{false}: \bb{B} \\
&|~\tt{ZNeg}: \bb{P} \ra \bb{Z} & &|~\tt{I}: \bb{P} \ra \bb{P} & &|~\tt{Gt}: \tt{Comp} & &\\
&\tt{int}: \set & &\tt{pos}: \set & &\tt{comp}: \set & &\tt{bool}: \set \\
&\el~\tt{int} \re \bb{Z} & &\el~\tt{pos} \re \bb{P} & &\el~\tt{comp} \re \tt{Comp} & &\el~\tt{comp} \re \bb{B}
\end{align*}
\caption{Overview of sorts, constructors, constants, and element relations}
\label{fig:sorts-constructors}
\end{figure}

The definition we use of integers in Lambdapi in \cref{fig:sorts-constructors} follows a common encoding found in many other theories, including the one adopted in the Rocq standard library \cite{Rocq-refman}.
First, the type $\bb{P}$  is an inductive type representing strictly positive integers in binary form.
Starting from 1 (represented by constructor \tt{H}), one can add a new least significant digit via the constructor \tt{O} (digit 0) or constructor \tt{I} (digit 1). 
The type $\bb{Z}$ is an inductive type representing integers in binary form.
An integer is either zero (with constructor \tt{Z0}) or a strictly positive number \tt{Zpos} (coded as a $\bb{P}$) or a strictly negative number \tt{Zneg} (whose opposite is stored as a $\bb{P}$ value).
%
As discussed in our previous work \cite{ColtellacciMD24}, $\lpm$ does not support quantifiy over a variable of type $\type$. More precisely, it is not possible to assign the type $\Pi X : \type,(X \ra \prop) \ra \prop$ to the universal quantifier $\forall$, where $\prop: \type$ is the type of proposition.
To address this, we introduce a constant $\set : \type$ for the types of object-terms, and a constant ${\el}$ to embed the terms of type $\set$ into terms of type $\type$ giving us the quantifier $\forall: \Pi x: \set, (\el~x \ra \prop) \ra \prop$.
%
To enable quantification over types such as integers, positive binary numbers, booleans, and comparison results, we introduce a constant of type $\set$ (e.g. $\tt{int}: \set$) that represents codes for these types — similar to the Tarski-style universe \cite[\S Universes]{intuitype},
where types are represented by elements of a base type and interpreted via the decoding function. In our setting, the decoding function $\el$  is realized through a rewriting rule that reduces the term to its corresponding type; for example, $\el~\tt{int} \re \bb{Z}$.
The comparison datatype $\tt{Comp}$ is utilized to define the decidable equality $\doteq$ between the $\bb{Z}$ and the function \tt{cmp} for $\bb{P}$ (as defined in \cref{app:lambdapi-func-def}).


\begin{figure}
\centering
\begin{minipage}[t]{0.48\textwidth}
\begin{align*}
&+: \bb{Z} \ra \bb{Z} \ra \bb{Z} \\
& \tt{Z0} + y \re y \\
& x + \tt{Z0} \re \tt{Z0} \\
& (\tt{Zpos x}) + (\tt{Zpos y}) \re (\tt{Zpos}~(\tt{add}~x~y))  \\
& (\tt{Zpos x}) + (\tt{Zneg y}) \re (\tt{sub}~x~y)  \\
& (\tt{Zneg x}) + (\tt{Zpos y}) \re (\tt{sub}~y~x)  \\
& (\tt{Zneg x}) + (\tt{Zneg y}) \re \tt{Zpos}(\tt{add}~x~y)  \\
\end{align*}
\hfill
\end{minipage}
\begin{minipage}[t]{0.48\textwidth}
\begin{align*}
&\doteq : \bb{Z} \ra \bb{Z} \ra \tt{Comp} \\
& \tt{Z0} \doteq \tt{Z0} \re \tt{Eq} \\
& \tt{Z0} \doteq \tt{Zpos}~\_ \re \tt{Lt} \\
& \tt{Z0} \doteq \tt{Zneg}~\_ \re \tt{Gt} \\
& \tt{Zpos}~\_ \doteq \tt{Z0} \re \tt{Gt} \\
& \tt{Zpos}~p \doteq \tt{Zpos}~q \re \tt{cmp}~p~q \\
& \tt{Zpos}~\_ \doteq \tt{Zneg}~\_ \re \tt{Gt} \\
& \tt{Zneg}~\_ \doteq \tt{Z0} \re \tt{Lt} \\
& \tt{Zneg}~\_ \doteq \tt{Zpos}~\_ \re \tt{Lt} \\
& \tt{Zneg}~p \doteq \tt{Zneg}~q \re \tt{cmp}~q~p \\
\end{align*}
\end{minipage}
\caption{Decidable equality and $+$ operator definition for $\bb{Z}$}
\label{fig:arith-ops}
\end{figure}

\[
\begin{array}{l@{\hspace{4em}}l@{\hspace{4em}}l}
\begin{aligned}
  &\tt{isEq} : \tt{Comp} \ra \bb{B} \\
  &\tt{isEq}~\tt{Eq} \re \tt{true} \\
  &\tt{isEq}~\tt{Lt} \re \tt{false} \\
  &\tt{isEq}~\tt{Gt} \re \tt{false} \\
\end{aligned}
&
\begin{aligned}
  &\tt{isLt} : \tt{Comp} \ra \bb{B} \\
  &\tt{isLt}~\tt{Eq} \re \tt{false} \\
  &\tt{isLt}~\tt{Lt} \re \tt{true} \\
  &\tt{isLt}~\tt{Gt} \re \tt{false} \\
\end{aligned}
&
\begin{aligned}
  &\tt{isGt} : \tt{Comp} \ra \bb{B} \\
  &\tt{isGt}~\tt{Eq} \re \tt{false} \\
  &\tt{isGt}~\tt{Lt} \re \tt{false} \\
  &\tt{isGt}~\tt{Gt} \re \tt{true} \\
\end{aligned}
\end{array}
\]

\begin{align*}
&\leq: \bb{Z} \ra \bb{Z} \ra \prop  \coloneq \lambda x,\lambda y, \neg (\tt{istrue}(\tt{isGt}(x \doteq y))) & &\tt{istrue} : \bb{B} \ra \prop \\
&<: \bb{Z} \ra \bb{Z} \ra \prop  \coloneq \lambda x,\lambda y, (\tt{istrue}(\tt{isLt}(x \doteq y))) & &\tt{istrue}~\tt{true} \re \top \\
&\geq: \bb{Z} \ra \bb{Z} \ra \prop  \coloneq \lambda x,\lambda y, \neg (x < y) & &\tt{istrue}~\tt{false} \re \bot \\
&>: \bb{Z} \ra \bb{Z} \ra \prop  \coloneq \lambda x,\lambda y, \neg (x \leq y) & & \\
\end{align*}

The arithmetic operator such as \tt{add}, \tt{sub}, and others, as presented in \cref{fig:arith-ops} are constants defined by rewriting rules. In the following sections, we will refers 
to the rewriting rules for integers as $\ra_\bb{Z}$ and positive binary numbers as $\ra_\bb{P}$.
The confluence of the rewriting rules for the arithmetic of $\mathbb{Z}$ and $\mathbb{P}$ has been proven using CSI \cite{CSI}. A detailed proof of confluence can be found in \cref{app:confluence-int-pos}.
The inequality symbols for $\bb{Z}$ are binary predicates defined by rewriting rules over the decidable equality $\doteq$. They reduce to $\top$, $\bot$ (or negated) by $\equivL$ with rules of $\ra_\bb{Z}$ and $\ra_\bb{P}$.
For example, $1 < 2 \hookrightarrow \tt{istrue}(\tt{isLt}(1 \doteq 2)) \hookrightarrow \tt{istrue}(\tt{isLt}(\tt{Lt})) \hookrightarrow \tt{istrue}(\tt{true}) \hookrightarrow \top$.

\subsection{Functions used in the translation}

We now provide an overview of how input problems expressed in a given SMT-LIB signature \cite[\S 5.2.1]{smtlib} are encoded.
A comprehensive description of the encoding can be found in \cite{ColtellacciMD24}, we will focus here on the arithmetic.
In order to avoid a notational clash with the Lambdapi signature $\Sigma$, we denote the set of SMT-LIB sorts as $\Theta^\mathcal{S}$, the set of function symbols $\Theta^\cal{F}$, and the set of variables $\Theta^\cal{X}$.
Our translation is based on the following functions:

\begin{itemize}
\item $\cal{D}$ translates declarations of sorts and functions in $\Theta^\cal{S}$ and $\Theta^\mathcal{F}$ into constants,
\item $\cal{S}$ maps sorts to $\Sigma$ types,
\item $\cal{E}$ translates SMT expression to $\lpm$ terms,
\item $\cal{C}$ translates a list of commands  $c_1 \dots c_n$ of the form $i.~\Gamma \triangleright~\varphi~(\mathcal{R}~P)[A]$ to typing judgments $\Gamma \vdash_\Sigma i := M: N$.
\end{itemize}

\begin{definition}[Function $\mathcal{D}$ translating SMT sort and function symbol declarations]
For each sort symbol $s$ with arity $n$ in $\Theta^\cal{S}$ we create a constant $s: \set \ra \dots \ra \set$.
For each function symbol $f~\sigma^+$ in $\Theta^\cal{F}$ we create a constant $f: \cal{S}(\sigma^+)$.
\end{definition}

\begin{definition}[Function $\mathcal{S}$ translating sorts of expression] 
  The definition of $\mathcal{S}$(s) is as follows.
  \begin{itemize}
    \item Case $s = \textbf{Bool}$, then $\Sort{s} = \el\,o$,
    \item Case $s = \textbf{Int}$, then $\Sort{s} = \el~\texttt{int}$,
    \item Case $s = \sigma_1\,\sigma_2 \dots \sigma_n$ then $\Sort{s} = \el{} (\mathcal{S}(\sigma_1) \leadsto \dots \leadsto \mathcal{S}(\sigma_n))$,
    \item otherwise $\Sort{s} = \el\, \mathcal{D}(s)$.
  \end{itemize}
  with the constant $o: \set$ and $\el\, o \re \prop$ to quantify over propositions.
\end{definition}

\begin{definition}[Function $\mathcal{E}$ translating SMT expressions]
The definition of $\E{e}$ is as follows.
\begin{itemize}
\setlength{\parskip}{0pt}
\item Case e $= (p~t_1~t_2\dots~t_n)$ and $p$ a logical connector,\\
  then $\E{e} = \E{t_1}~p^c~\dots~p^c~\E{t_n}$.
\item Case e $= (+~t_1\dots~t_n)$, then $\E{e} = \E{t_1} + ~\dots~ +\E{t_n}$.
\item Case e $= (*~t_1\dots~t_n)$, then $\E{e} = \E{t_1} * ~\dots~ *\E{t_n}$.
\item Case e $= (-t)$, then $\E{e} = ~\E{t_1}$.
\item Case e $= (-~t_1\dots~t_n)$, then $\E{e} = \E{t_1} - ~\dots~ - \E{t_n}$.
\item Case e $= (\approx~t_1~t_2)$ then $\E{e} = (\E{t_1} = \E{t_2})$.
\item Case e $= (Q~x_1 : \sigma_1  \dots x_n : \sigma_n ~t)$ where $Q\in \{\kw{forall}, \kw{exists}\}$, then $\E{e} = Q^c x_1: \cal{S}(\sigma_1), \dots, Q^c x_n: \cal{S}(\sigma_n), \E{t}$. 
\item Case $e = (x: \sigma )$ with $x \in \Theta^\mathcal{X}$ a sorted variable, then $\E{e} = x: \cal{S}(\sigma)$.
\end{itemize}
\end{definition}

\begin{example}{Translation of the prelude in \todo[ac]{cref looks broken, should be Listing}{\cref{lst:smtexampleinput}}}
\begin{lstlisting}[language=Lambdapi]
symbol x: El int;
symbol y: El int;
\end{lstlisting}
\end{example}

\begin{definition}
We define the function $\cal{C}$ that translates an Alethe step of the form $i.~\Gamma~\triangleright~l_1 \dots l_n\,(R\,P)[A]$ into a constant
$i: \pid (\E{l_1}\cons\dots\cons\E{l_n}\cons \nil) \coloneq M$ where $M$ is a proof term of appropriate type.
The function $\cal{C}$ is defined by cases on the rule $R$.
\end{definition}

We introduce an embedding $\pid: \tt{Clause} \ra \type$ of clause into types, mapping each clause $C$ to the type $\pid~C$ of its proofs.
Similarly, the constant $\pic: \prop \ra \type$ maps each proposition A. The type $\tt{Clause}: \type$ represent the type of clause encoded as list \cite[\S 3]{ColtellacciMD24}
with the constructor $\veedot: \prop \ra \tt{Clause} \ra \tt{Clause}$ and the empty clause $\nil: \tt{clause}$.

\begin{example}{Translation of the proof goal of step t2 and t5 in \cref{lst:smtexampleproof}}
\begin{lstlisting}[language=Lambdapi,mathescape=true]
opaque symbol t2: $\pid$ (¬ (3 < x)  ⟇ ¬ (x = 2) ⟇ ▩) ≔ begin ... end;
opaque symbol t5: $\pid$ (¬ (x + y < 1) ⟇ (¬ (x = 2))  ⟇ (¬ (0 = y)) ⟇ ▩) ≔ begin ... end;
\end{lstlisting}
\end{example}
  
% (step t2 (cl (not (< 3 x)) (not (= x 2))) :rule la_generic :args (1/1 1/1))