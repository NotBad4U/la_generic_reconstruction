\section{Reconstruction of linear arithmetic for LIA logic}

We will now describe an alternative technique based on computational reflection that allows us to prove the permutation of clauses efficiently.
Proof by computational reflection is a technique introduced in \cite{reflected_origin} that benefits from the internal reduction system of the proof assistant in order to reduce the size of the proof term computed and consequently speed up its checking. In Lambdapi, we can take advantage of the fact that rewriting rules are part of the internal reduction system ($\equiv_{\beta\cal{R}}$), which makes proof by reflection convenient to set up and implement.
Relying on the rewriting facilities of Lambdapi, we implemented a decision procedure that checks equality between clauses by rewriting modulo AC-canonization.

The core idea is to put clauses with pivots in different positions into a canonical form, allowing them to be compared.
If two clauses are determined to be equal, the current clause can be substituted with one where the pivot is placed at the head position, allowing for the subsequent application of \cref{lemma:resolution}.
To handle associative and commutative symbols, Lambdapi provides the modifiers \texttt{associative} and \texttt{commutative},
ensuring that terms are systematically placed into a canonical form given a builtin ordering relation, following the technique described in \cite{ACorigin} and \cite[\S 5]{univAC}.



% https://q.uiver.app/#q=WzAsOCxbMSwyLCJcXG1hdGhiYntafSJdLFsxLDAsIlxcbWF0aGNhbHtSfSJdLFszLDAsIlxcbWF0aGNhbHtSfSJdLFszLDIsIlxcbWF0aGJie1p9Il0sWzAsMiwidF8xID1fXFxtYXRoYmJ7Wn0gdF8yIl0sWzQsMiwiZGVub3RlKHRfMSkgPV9cXG1hdGhiYntafSBkZW5vdGUodF8yKSJdLFswLDAsInJlaWZ5KHRfMSkgPV9cXG1hdGhjYWx7Un0gcmVpZnkodF8yKSJdLFs0LDAsIm5vcm0odF8xKSA9X1xcbWF0aGNhbHtSfSBub3JtKHRfMikiXSxbMCwxLCJyZWlmeSJdLFsxLDIsIlxccmlnaHRhcnJvd197QUN9IiwwLHsic3R5bGUiOnsiYm9keSI6eyJuYW1lIjoiZG90dGVkIn19fV0sWzIsMywiZGVub3RlIl0sWzAsMywiXFxlcXVpdiIsMSx7InN0eWxlIjp7ImJvZHkiOnsibmFtZSI6ImRvdHRlZCJ9LCJoZWFkIjp7Im5hbWUiOiJub25lIn19fV1d
\begin{tikzcd}[ampersand replacement=\&,column sep=small]
	{reify(t_1) =_\mathcal{R} reify(t_2)} \& {\mathcal{R}} \&\& {\mathcal{R}} \& {{t_1\downarrow_{AC}} =_\mathcal{R} t_2\downarrow_{AC}} \\
	\\
	{t_1 =_\mathbb{Z} t_2} \& {\mathbb{Z}} \&\& {\mathbb{Z}} \& {denote({t_1\downarrow_{AC}}) =_\mathbb{Z} denote({t_2\downarrow_{AC}})}
	\arrow["{\rightarrow_{AC}}", dotted, from=1-2, to=1-4]
	\arrow["denote", from=1-4, to=3-4]
	\arrow["reify", from=3-2, to=1-2]
	\arrow["\iff"{description}, dotted, no head, from=3-2, to=3-4]
\end{tikzcd}


\begin{definition}[$\mathcal{R}$]
\begin{align*}
\add{\var{x}{c_1}}{\var{x}{c2}} &\re \kw{var}~x~(c_1 + c_2) \\
\add{\var{x}{c_1}}{(\add{\var{x}{c_2}}{y})} &\re \add{\var{x}{c_1 + c_2}}{y} \\
\add{\cst{c_1}}{\cst{c_2}} &\re \cst{c_1 + c_2} \\
\add{\cst{c_1}}{(\add{\cst{c_2}}{y})} &\re \add{\cst{c_1 + c_2}}{y} \\
\add{\cst{0}}{x} &\re x \\
\add{x}{\cst{0}} &\re x \\s
\opp{\var{x}{c}} &\re \var{x}{(-c)} \\
\opp{\cst{c}} &\re \cst{(-c)} \\
\opp{\opp{x}} &\re x \\
\opp{\add{x}{y}} &\re \add{(\opp{x})}{(\opp{y})} \\
\mul{k}{\var{x}{c}} &\re \var{x}{(k \times c)} \\
\mul{k}{\opp{x}} &\re \mul{(-k)}{x} \\
\mul{k}{(\add{x}{y})} &\re \add{(\mul{k}{x})}{(\mul{k}{y})} \\
\mul{k}{\cst{c}} &\re \cst{k \times c} \\
\mul{c_1}{(\mul{c_2}{x})} &\re \mul{(c_1 \times c_2)}{x} \\
\end{align*}
\end{definition}


\begin{definition}[reify]
\begin{align*}
\reify{0} &\re \cst{0} \\
\reify{(-x)} &\re \opp{\reify{x}} \\
\reify{(x + y)} &\re \add{\reify{x}}{\reify{y}} \\
\reify{x} &\re \var{x}{1} \\
\end{align*}
\end{definition}

\begin{definition}[denote]
\begin{align*}
\den{\var{c}{x}} &\re  c \times x \\
\den{\cst{c}} &\re c \\
\den{\opp{x}} &\re  - (\den{x}) \\
\den{\mul{c}{x}} &\re  c \times \den{x} \\
\den{\add{x}{y}} &\re \den{x} + \den{y}\\
\end{align*}
\end{definition}


\begin{definition}
    Let $\mathit{aliens}_{\sqcup}: \mathcal{C} \rightarrow \mathcal{C}^+$ be the function mapping every term in $\mathcal{C}$ to a non-empty list of terms such that $\mathit{aliens}_{\sqcup}(t) = \mathit{aliens}_{\sqcup}(u) \circ \mathit{aliens}_{\sqcup}(v)$ if $t = u \sqcup v$, and $\mathit{aliens}_{\sqcup}(t) = [t]$ otherwise.
  
    Conversely, let $\mathit{comb}_{\sqcup} \colon \mathcal{C}^+ \rightarrow \mathcal{C}$ be the function mapping a non-empty list of $\mathcal{C}$-terms to a term such that  $\mathit{comb}_{\sqcup}[t] = t$ and for all $n \geq 2,  \mathit{comb}_{\sqcup}[t_1, \dots, t_n] = t_1 \sqcup \mathit{comb}_{\sqcup}[t_2,\dots,t_n]$.
  \end{definition}
  
  \smallskip 
  
  For example $\mathit{aliens}_{\sqcup}((x \sqcup y) \sqcup z) = [x, y, z]$ and $\mathit{comb}_{\sqcup}[x, y, z] = ((x \sqcup y) \sqcup z)$.
  
  \smallskip
  
\begin{definition}[AC-canonical form]
Let $\leq$ be any total order on $\cal{C}$-terms with $\epsilon$ the least element such that for all $x$ and $b$ we have $\epsilon < \var{b}{x}$, and $\var{b}{x} \leq \var{b'}{y}$ iff $x < y$ or else $x = y$ and $b \leq b'$ with the order $\texttt{\false} < \texttt{\true}$.
The AC-canonization of a term $t$ of $\cal{C}$ is defined as $[t]_{AC} = \mathit{comb}_{\sqcup} [\texttt{sort}(\mathit{aliens}_{\sqcup}(t))]$, where $\texttt{sort(l)}$ is the list of the elements of $l$ in increasing order with respect to $\leq$. The relation associating every term $t$ with its AC-canonization $[t]_{AC}$ is denoted $\RRAC$. Two terms $t$ and $t'$ are AC-equivalent if $[t]_{AC} = [t']_{AC}$.
The term $t$ is in AC-canonical form if $t = [t]_{AC}$ and if every strict subterm of $t$ is in AC-canonical form. 
\end{definition}


\smallskip

\begin{example}
Assuming that the terms $x$ and $y$ are ordered $x < y$, the AC-canonical form of $XXX$ is $XXX$.
\end{example}
  
\smallskip
  
\begin{definition}[Rewriting modulo AC-canonization]
Let $\RAC = \RRAC \R$, where $\mathcal{R}$ is defined by the rewrite rules of ??.
\label{def:RAC}
\end{definition}
  
\smallskip
  
An $\RAC$ step is an AC-canonization followed by a standard $\lra_{\mathcal{R}}$ step with syntactic matching.
  
  