\section{Reconstruction of linear arithmetic for LIA logic}
\label{sec:lia-reconstruction}

Proof by reflection \cite{reflection-origin-coq} is a technique for writing certified procedures for automated reasoning. It reduces the validity of a logical statement to a symbolic computation.
Let $P: Z \ra \prop$ be a predicate over a data type Z and $f: Z \ra \tt{bool}$ be a function such that the following theorem holds:

\begin{equation*}
\tt{f\_correct} : \forall z: Z, (f~z = \tt{true}) \ra (P~z)
\end{equation*}

If $\mathop{f} z$ reduces to \tt{true}, then the proof term  $\tt{f\_correct}~z~(\tt{refl}~\tt{bool}~\tt{true})$ with $\tt{refl}: \Pi A: \set,\, \Pi x: \el\,A,\, \pic (x = x)$, constitutes a proof of predicate $(P~z)$. In step 6 of checking an application of rule \tt{la\_generic},
the primary challenge lies in reasoning modulo associativity and commutativity when manipulating expressions over $\bb{Z}$.
The key idea is to provide a normalization function that transforms a $\bb{Z}$ expression into a canonical form.
% such that it can be reduced to a constant because variables will cancel each other, as is the case with the constant $f$ in \cref{ex:la_generic_example_red}.


\subsection{Representation}
\label{ssec:representation}

The procedure is based on an algebraic group structure, denoted as  $\bb{G}$ defined in \cref{fig:grp}, which represents linear polynomials.
The base type for its elements is $\bb{G}: \type$. The unary operator $\tt{cst}$ injects constants from $\bb{Z}$ into $\bb{G}$.
The term $\tt{var}~c~x$ is intended for representing expressions $c \times x$ that appear as constituents of linear inequalities, where $c$ is an integer coefficient and $x$ a $\bb{Z}$ term, in particular a variable.
The constructor $\tt{mul}$  represents the multiplication of an element of $\bb{G}$ by a constant. The constructor $\tt{opp}$ corresponds to unary minus.
Lastly, the constructor $\add{}{}$ represents the addition between two elements of $\bb{G}$.

Lambdapi provides modifiers %\lstinline[language=Lambdapi,basicstyle=\ttfamily\footnotesize\upshape]{associative commutative} 
for supporting associative and commutative operations,
ensuring that terms are systematically transformed into a canonical form w.r.t.\ a builtin ordering relation \cite{univAC,ACorigin}. We declare the operator $\add{}{}$ as \lstinline[language=Lambdapi,basicstyle=\ttfamily\footnotesize\upshape]{associative commutative},
ensuring that expressions involving sums of elements of $\bb{G}$ are systematically canonicalized. 
%In particular, terms of the form $\tt{var}~c~x$ for equal variables will be placed next to each other, facilitating simplification.

\begin{figure}[t]
\begin{align*}
& \bb{G}: \type & & \reify{} : \bb{Z} \ra \bb{G} & & \den{}: \bb{G} \ra \bb{Z} \\
&|~\add{}{}: \bb{G} \ra \bb{G} \ra \bb{G} & & \reify{\tt{Z0}} \re \cst{\tt{Z0}} & & \den{\cst{c}} \re c \\
&|~\tt{var}: \bb{Z} \ra \bb{Z} \ra \bb{G} & & \mathop{\reify{\ZPos}} c \re \cst{(\ZPos c)} & & \den{\opp{x}} \mathrel{\re}  \mathop{\sim (\den{x})} \\
&|~\tt{mul}: \bb{Z} \ra \bb{G} \ra \bb{G} & & \mathop{\reify{\ZNeg}} c \re \cst{(\ZNeg c)} & & \den{\mul{c}{x}} \re  c \times (\den{x}) \\
&|~\tt{opp}: \bb{G} \ra \bb{G} & & \reify{(x + y)} \re \add{(\reify{x})}{(\reify{y})} & & \den{\add{x}{y}} \re (\den{x}) + (\den{y}) \\
&|~\tt{cst}: \bb{Z} \ra \bb{G} & & \reify{(\sim x)} \re \opp{\reify{x}} & & \den{\var{c}{x}} \re  c \times x \\
&\tt{grp}: \set & & \mathop{\reify{((\mathop{\ZPos} c) * x})} \re \mul{(\mathop{\ZPos} c)}{(\reify{x})}  & & \\
&\el~\tt{grp} \re \bb{G} & & \mathop{\reify{((\mathop{\ZNeg} c) * x)}} \re \mul{(\mathop{\ZNeg} c)}{(\reify{x})} & & \\
& & & \mathop{\reify{(x * (\mathop{\ZPos} c))}} \re \mul{(\mathop{\ZPos} c)}{(\reify{x})}  & & \\
& & & \mathop{\reify{(x * (\mathop{\ZNeg} c))}} \re \mul{(\mathop{\ZNeg} c)}{(\reify{x})} & & \\
& & & \mathop{\reify{(x * \tt{Z0})}} \re \cst{0} & & \\
& & & \mathop{\reify{(\tt{Z0} * x)}} \re \cst{0} & & \\
& & & \mathop{\reify{x}} \re \var{1}{x} & &
\end{align*}
\caption{Definition of $\bb{G}$  Algebra and its reification ($\reify{}$) and denotation ($\den{}$) functions.}
\label{fig:grp}
\end{figure}

\subsection{Associative Commutative Normalization}
\label{ssec:normalization}

\begin{figure}[t]
\begin{align}
&\add{\var{c_1}{x}}{\var{c_2}{x}} \re \var{(c_1 + c_2)}{x} \\
&\add{\var{c_1}{x}}{(\add{\var{c_2}{x}}{y})} \re \add{\var{(c_1 + c_2)}{x}}{y} \\
&\add{\cst{c_1}}{\cst{c_2}} \re \cst{c_1 + c_2} \\
&\add{\cst{c_1}}{(\add{\cst{c_2}}{y})} \re \add{\cst{c_1 + c_2}}{y} \\
&\add{\cst{0}}{x} \re x \\
&\add{x}{\cst{0}} \re x \\
&\opp{\var{c}{x}} \re \var{(-c)}{x} \\
&\opp{\cst{c}} \re \cst{(-c)} \\
&\opp{(\opp{x})} \re x \\
&\opp{(\add{x}{y})} \re \add{(\opp{x})}{(\opp{y})} \\
&\opp{(\mul{k}{x})} \re \mul{(-k)}{x} \\
&\mul{k}{\var{c}{x}} \re \var{(k * c)}{x} \\
&\mul{k}{(\opp{x})} \re \mul{(-k)}{x} \\
&\mul{k}{(\add{x}{y})} \re \add{(\mul{k}{x})}{(\mul{k}{y})} \\
&\mul{k}{\cst{c}} \re \cst{(k * c)} \\
&\mul{c_1}{(\mul{c_2}{x})} \re \mul{(c_1 * c_2)}{x}
\end{align}
\caption{Rewrite system on canonical forms.}
\label{fig:grp-rw}
\end{figure}

%% sm: not sure if this is important. If you think it is, it should go to the section where AC and canonicalization are introduced.
%For \lstinline[language=Lambdapi,basicstyle=\ttfamily\footnotesize\upshape]{associative commutative} symbols, Lambdapi does not use matching modulo AC \cite{matching-mod-AC,kirchner_rsp} as this problem is NP-complete \cite{ac-modulo-np-complete}.
The transformation to canonical form implemented in Lambdapi ensures that sum expressions of the form $\var{c_1}{x_1} \oplus \cst{k_1} \oplus \var{c_2}{x_2} \oplus \dots \oplus \cst{k_m} \oplus \dots \oplus \var{c_n}{x_n}$ will be normalized such that any pair of terms $\var{p}{x}$ and $\var{q}{x}$ involving the same variable $x$ are placed next to each other,
and all $\cst{k_i}$ will be placed at the left before the first variable $\var{c_j}{x_j}$.
We will use the rewriting rules shown in \cref{fig:grp-rw} for reducing $\bb{G}$ expressions. Notably, the resulting normal forms do not contain the constructors $\tt{mul}$ and $\tt{opp}$, as the associated rewrite rules eliminate them in favor of $\tt{var}, \tt{add}$ and $\tt{cst}$.

\begin{definition}%[AC-canonical form]
The $\leq$ builtin total order on $\bb{G}$-terms is defined as follows:
Terms are ordered such that $\tt{cst}(c_1) \leq \tt{cst}(c_2) < \var{p}{x}$ for any constants $c_1, c_2$ and any variable term $\var{p}{x}$, with $c_1 \leq c_2$.
For variable terms, $\var{p}{x} \leq \var{q}{y}$ if either $x < y$, or $x = y$ and $p \leq q$.
Let $\twoheadrightarrow^{AC}$ be the relation mapping every term t to its unique AC-canonical form denoted $[t]$.
\end{definition}

Two terms $t$ and $u$ are AC-equivalent (written $t \simeq_{AC} u$) iff their AC-canonical forms are equal.

\begin{definition}%[Rewriting modulo AC-canonization]
The relation $\ACcanon$ is defined as $\re_\Sigma\,\twoheadrightarrow^{AC}$, where $\Sigma$ contains the rewrite rules of \cref{fig:arith-ops,fig:grp-rw}. 
\end{definition}

An $\ACcanon$ step is a standard $\re_\Sigma$ step with syntactic matching followed by AC-canonicalization. We now prove that the relation $\ACcanon$ terminates and is confluent.

\begin{lemma} 
	The relation $\mathop{\rwModAC} \mathrel{=} \mathop{\simeq_{AC} \, \re_\Sigma \, \simeq_{AC}}$ of matching modulo AC, which contains $\ACcanon$, terminates.
\end{lemma}
\begin{proof} 
	AProVE \cite{aprove} automatically proves the termination of $\rwModAC$. 
\qed
\end{proof}


\begin{lemma} 
	$\ACcanon$ is locally confluent on AC-canonical terms.
\end{lemma}
\begin{proof}
	We show that every critical pair is joinable using $\ACcanon$ and confluence of $\ra_\bb{Z}$ and $\ra_\bb{P}$.
% In the following, the terms that are not between square brackets are in AC-canonical form. We also write $[\add{p}{q}]$ to denote either $\add{p}{q}$ or $\add{q}{p}$.
\qed
\end{proof}

We compare two $\bb{Z}$-terms $t_1$ and $t_2$ wrt $\ACcanon$ by reifying them into their corresponding $\bb{G}$-terms, denoted $[g_1]$ and $[g_2]$, using the reification function $\reify{}$,
and normalizing them using $\twoheadrightarrow^{AC}$. Following the reduction rules specified in \cref{fig:grp-rw}, we can then compare their corresponding $\bb{Z}$-terms by applying the denotation function $\den{}$.
To validate this procedure, it is necessary to establish the correctness of the following diagram, formally expressed by \cref{thm:normalization}.

\begin{figure}
\centering
% % https://q.uiver.app/#q=WzAsOCxbMSwyLCJcXGJ1bGxldCJdLFsxLDAsIlxcYnVsbGV0Il0sWzMsMiwiXFxidWxsZXQiXSxbMywwLCJcXGJ1bGxldCJdLFswLDIsInRfMSA9X1xcbWF0aGJie1p9IHRfMiJdLFswLDAsIlxcRG93bmFycm93KFxcVXBhcnJvdyh0XzEpKSA9X1xcbWF0aGJie1p9IFxcRG93bmFycm93KFxcVXBhcnJvdyh0XzIpKSJdLFs0LDAsIlxcRG93bmFycm93KFtnXzFdKSA9X1xcbWF0aGJie1p9IFxcRG93bmFycm93KFtnXzJdKSJdLFs0LDIsImdfMSA9X1xcbWF0aGJie1p9IGdfMiJdLFswLDEsIlxcRG93bmFycm93KFxcVXBhcnJvdyhcXF8pKSJdLFswLDIsIlxcaWZmIiwxLHsic3R5bGUiOnsiYm9keSI6eyJuYW1lIjoiZG90dGVkIn0sImhlYWQiOnsibmFtZSI6Im5vbmUifX19XSxbMSwzLCJbXFxfXSIsMCx7InN0eWxlIjp7ImJvZHkiOnsibmFtZSI6ImRhc2hlZCJ9fX1dLFszLDIsIlxcRG93bmFycm93KFxcXykiXSxbMSwyLCIiLDAseyJzdHlsZSI6eyJuYW1lIjoiY29ybmVyIn19XV0=
\(\begin{tikzcd}[ampersand replacement=\&]
	{\Uparrow(t_1) =_\bb{G} (\Uparrow(t_2)} \& \bb{G} \&\& \bb{G} \& {[g_1] =_\bb{G} [g_2]} \\
	\\
	{t_1 =_\bb{Z} t_2} \& \bb{Z} \&\& \bb{Z} \& {\den{g_1} =_\bb{Z} \den{g_2}}
	\arrow["{[\_]}", dashed, from=1-2, to=1-4]
	\arrow["\lrcorner"{anchor=center, pos=0.125}, draw=none, from=1-2, to=3-4]
	\arrow["{\Downarrow(\_)}", from=1-4, to=3-4]
	\arrow["{\Uparrow(\_)}", from=3-2, to=1-2]
	\arrow["\iff"{description}, dotted, no head, from=3-2, to=3-4]
\end{tikzcd}\)
\end{figure}

\begin{theorem}[Correctness of normalization]\label{thm:normalization}
For all $\bb{G}$-terms $t$, we have $(\den{[t]}) = (\den{t})$ where $[t]$ is the AC-canonical form of $t$ with respect to $\longrightarrow^{AC}_\Sigma$.
\end{theorem}
\begin{proof}
  The proof proceeds by induction on $t$, and the key case is the one where $t = t_1 \oplus t_2$.
% This is a meta-level proof, proceeding by structural induction on $t$.
% Assume that the normalization function $[\_]$ is implemented using a merge sort algorithm over the multiset of subterms $\bb{G}$, with respect to the associative-commutative operator $\oplus$.
% We focus on the inductive case where $t = t_1 \oplus t_2$.
We have to show that $\mathop{\den{[t_1 \oplus t_2]}} = \mathop{(\den{t_1}) + (\den{t_2})}$.
By the induction hypothesis, we have
$\mathop{\den{[t_1]}} = \mathop{\den{t_1}}$ and $\mathop{\den{[t_2]}} = \mathop{\den{t_2}}$.
Hence,
\[
  (\den{t_1}) + (\den{t_2})
  = (\den{[t_1]}) + (\den{[t_2]})
  = \mathop{\den{([t_1] \oplus [t_2])}}
\]
It remains to show that
$\mathop{\den{([t_1] \oplus [t_2])}} = \mathop{\den{[t_1 \oplus t_2]}}$.
Now, $[t_1]$, $[t_2]$, and $[t_1 \oplus t_2]$ are terms built solely from \tt{cst}, \tt{var}, and $\oplus$ since the remaining operators have been eliminated by applying the rules in \cref{fig:grp-rw}, and the terms on both sides of the equation contain the same multisets of subterms. The two terms are therefore identified by AC-canonicalization.
%
% $[t_1 \oplus t_2]$ is the canonical (i.e., sorted) form of the multiset of subterms in $t_1$ and $t_2$,
% and that merge-sorting two normalized (i.e., already sorted) linear polynomials yields the same result as normalizing (i.e sorting) their merge.
% Therefore, since the normalization preserves denotation, we conclude that $\den{[t]} = \mathop{\den{t}}$ for all $\bb{G}$-terms $t$.
\qed
\end{proof}



We make use of \cref{lem:conv} to embed $\bb{Z}$-terms into $\bb{G}$ to normalize and subsequent comparison.
In addition, we leverage this normalization process to support the \tt{arith-poly-norm} rule.

\begin{lemma}[Conversion]\label{lem:conv}
For all $x: \bb{Z}$, we have $x = (\mathop{\den{(\reify{x})}})$.
\end{lemma}
\begin{proof}
By induction on $x$. We consider the three cases: $x = \ZPos(n)$, $x = \ZNeg(n)$, and $x = \tt{Z0}$.  
In each case, $\reify{(x)}$ yields the corresponding constant $\cst{x}$, and by definition of the denotation function, $x = \mathop{\den{(\cst{x})}}$.
Hence, $x = (\mathop{\den{(\reify{x})}})$ in all cases.
\end{proof}

The full proof \cref{lst:smtexampleproof} translated into Lambdapi can be found in \cref{app:example-translation}.
