
\section{Overview of the linear Arithmetic rules in Alethe}

All linear arithmetic tautology rules, such as \tt{la\_disequalities}, \tt{la\_totality},
and simplification rules like \tt{comp\_simplify}, are encoded as lemmas in our embedding of Alethe in Lambdapi, as presented in \cref{sect:recon-lambdapi}.
The \tt{la\_generic} rule, however, must be reconstructed using a different approach, as it involves following the algorithm described below.

A step of the rule \tt{la\_generic} represents a tautological clause of linear
disequalities.  It can be checked by showing that the conjunction of
the negated disequalities is unsatisfiable. After the application of
some strengthening rules, the resulting conjunction is unsatisfiable,
even if integer variables are assumed to be real variables.

A linear inequality is of term of the form
\[
\sum_{i=0}^{n}c_i\times{}t_i +
d_1\bowtie \sum_{i=n+1}^{m} c_i\times{}t_i + d_2
\]

where $\bowtie\;\in \{=, <,
>, \leq, \geq\}$, where $m\geq n$, $c_i, d_1, d_2$ are either integer or real
constants, and for each $i$ $c_i$ and $t_i$ have the same sort. We will write
$s_1 \bowtie s_2$.

Let $l_1,\dots, l_n$ be linear inequalities and $a_1, \dots, a_n$
rational numbers, then a \tt{la\_generic} step has the form

\[
\begin{matrix*}[c]
  i. & \ctxsep & \varphi_1 , \dots , \varphi_n & \tt{la\_generic}  & [a_1, \dots, a_n] \\
\end{matrix*}
\]

\smallskip

The constants $a_i$ are of sort \tt{Real} and must be printed
using one of the productions $\grNT{rational}$
$\grNT{decimal}$, $\grNT{nonpositive\_decimal}$.

\smallskip

To check the unsatisfiability of the negation of $\varphi_1, \dots, \varphi_n$ one performs the following steps for each literal. For
each $i$, let $\varphi := \varphi_i$ and $a := a_i$.


\begin{enumerate}
    \item If $\varphi = s_1 > s_2$, then let $\varphi := s_1 \leq s_2$.
      If $\varphi = s_1 \geq s_2$, then let $\varphi := s_1 < s_2$.
      If $\varphi = s_1 < s_2$, then let $\varphi := s_1 \geq s_2$.
      If $\varphi = s_1 \leq s_2$, then let $\varphi := s_1 > s_2$. This negates
      the literal.
    
    \item If $\varphi = \neg (s_1 \bowtie s_2)$, then let $\varphi := s_1 \bowtie s_2$.
    
    \item If $\varphi = s_1 < s_2$, then let $\varphi :=   - s_1 > - s_2$.
      If $\varphi = s_1 \leq s_2$, then let $\varphi :=  s_1 \geq - s_2$.

    \item Replace $\varphi$ by $\sum_{i=0}^{n}c_i\times{}t_i - \sum_{i=n+1}^{m} c_i\times{}t_i
    \bowtie d$ where $d := d_2 - d_1$.
    
    \item \label{la_generic:str}Now $\varphi$ has the form $s_1 \bowtie d$. If all
    variables in $s_1$ are integer sorted: replace $\bowtie d$ according to
    the table below.
    
    \item If $\bowtie$ is $=$ replace $l$ by
    $\sum_{i=0}^{m}a\times{}c_i\times{}t_i = a\times{}d$, otherwise replace it by
    $\sum_{i=0}^{m}|a|\times{}c_i\times{}t_i = |a|\times{}d$.
    Coefficients are put on the same denominator to keep whole numbers as coefficients.
\end{enumerate}

The replacements that can be performed by step~\ref{la_generic:str} above
are

\begin{tabular}{r l l}
$\bowtie$  & If $d$ is an integer  & Otherwise \\
$>$        & $\geq d + 1$  & $\geq \lfloor d\rfloor + 1$  \\
$\geq$     & $\geq d$      & $\geq \lfloor d\rfloor + 1$  \\
\end{tabular}

Finally, the sum of the resulting literals is trivially contradictory.
The sum
\[
    \sum_{k=1}^{o}\sum_{i=1}^{m^o}c_i^k*t_i^k \bowtie \sum_{k=1}^{o}d^k
\]
where $c_i^k$ is the constant $c_i$ of literal $l_k$, $t_i^k$ is the term
$t_i$ of $l_k$, and $d^k$ is the constant $d$ of $l_k$. The operator
$\bowtie$ is $=$ if all operators are $=$, $>$ if all are
either $=$ or $>$, and $\geq$ otherwise. The $a_i$ must be such
that the sum on the left-hand side is $0$ and the right-hand side is $>0$ (or
$\geq 0$ if $\bowtie$ is $>$).

The step 3 has been added by our own since the following steps in the original algorithm work with $>$ and $\geq$ and does not
mention clearly what to do with $<$ and $\leq$.

\begin{example}
Consider the $\tt{la\_generic}$ step in the logic \tt{LIA}:
\begin{lstlisting}[language=SMT,label={lst:smtexampleinput}]
(step t11 (cl (not (<= f 0)) (<= (+ 1 (* 4 f)) 1))
  :rule la_generic :args (1.0 1/4))
\end{lstlisting}
After step 4, we get $- f > 0 ~\land~ 4 \times f > 0$. The step 5 applied and we can strengthen this to
$- f \geq 0 ~\land~ 4 \times f \geq 1$ and after multiplication of the normalized coefficients we get $4 \times(- f) \geq 0 ~\land~ 4 \times f \geq 1$.
Which sums to the contradiction  $0 \geq 1$. 
\end{example}

In the next section, we first present an overview of our embedding of Alethe in Lambdapi, and then our automatic procedure to reconstruct $\tt{la\_generic}$ step that appear in LIA problem.
