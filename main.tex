\documentclass[runningheads]{llncs}
\usepackage[T1]{fontenc}
\usepackage{graphicx}
\usepackage{listings}
\usepackage{mathtools}
\usepackage[dvipsnames]{xcolor}
\usepackage{wasysym} % \veedot symbol
\usepackage{annotate-equations}
\usepackage{amssymb}
\usepackage{quiver}
\usepackage{cleveref}
\usepackage{ebproof}


%%%% configuration and new commands  %%%%

% % ===== Unicode Alethe encoding in Lambdapi =========
\DeclareUnicodeCharacter{03C0}{\texttt{Prf}}
\DeclareUnicodeCharacter{27F9}{$\Rightarrow$}
\DeclareUnicodeCharacter{2254}{$\coloneqq$}
\DeclareUnicodeCharacter{2091}{$_e$}
\DeclareUnicodeCharacter{22A5}{$\bot$}
\DeclareUnicodeCharacter{25A9}{$\blacksquare$}
\DeclareUnicodeCharacter{0307}{$^{\bullet}$}
\DeclareUnicodeCharacter{27C7}{$\veedot$}
\DeclareUnicodeCharacter{2228}{$\lor$}
\DeclareUnicodeCharacter{1D9C}{$^c$}
\DeclareUnicodeCharacter{1D62}{$_i$}
\DeclareUnicodeCharacter{2081}{$_1$}
\DeclareUnicodeCharacter{2097}{$_l$}
\DeclareUnicodeCharacter{1D63}{$_r$}
\DeclareUnicodeCharacter{21AA}{$\hookrightarrow$}
\DeclareUnicodeCharacter{03C4}{\textcolor{blue}{\texttt{El}}}
\DeclareUnicodeCharacter{2115}{$\mathbb{N}$}
\DeclareUnicodeCharacter{03A0}{$\Pi$}
\DeclareUnicodeCharacter{2227}{$\land$}
\DeclareUnicodeCharacter{21D2}{$\Rightarrow$}
\DeclareUnicodeCharacter{2200}{$\forall$}
\DeclareUnicodeCharacter{2933}{$\leadsto$}
\DeclareUnicodeCharacter{2203}{$\exists$}
\DeclareUnicodeCharacter{03F5}{$\epsilon$}
\DeclareUnicodeCharacter{21D4}{$\Leftrightarrow$}
\DeclareUnicodeCharacter{22A4}{$\top$}
\DeclareUnicodeCharacter{2250}{$\doteq$}
\DeclareUnicodeCharacter{25A1}{$\square$}
\DeclareUnicodeCharacter{2082}{$_2$}

% % ===== Grammar Alethe =========

\definecolor{SmtBlue}{HTML}{00007f}
\definecolor{SmtGreen}{HTML}{3b7f31}
\definecolor{SmtStepId}{HTML}{3b7f31}
\definecolor{indexClr}{HTML}{ffcc00}

\newcommand{\grNT}[1]{\textcolor{SmtGreen}{\langle\texttt{#1}\rangle}}
\newcommand{\grT}[1]{\textcolor{SmtBlue}{\texttt{#1}}}
\newcommand{\grRule}{=}
\newcommand{\grOr}{|}
\newcommand{\concat}{~+\!\!\!+~}

% % ===== Command for expression in the paper =========

\newcommand\textAlethe[1]{\texttt{#1}}
\newcommand\C[1]{\mathcal{C}(#1)}
\newcommand\D[1]{\mathcal{D}(#1)}
\newcommand\F[1]{\mathcal{F}(#1)}
\newcommand\E[1]{\mathcal{E}(#1)}
\newcommand\Sort[1]{\mathcal{S}(#1)}
\newcommand\equivL{\equiv_{\beta\Sigma}}
\newcommand\pid{\textcolor{purple}{\texttt{\upshape{Prf}}^\bullet}}
\newcommand\pic{\textcolor{purple}{\texttt{\upshape{Prf}}^c}}

% % ===================== Alias =====================
\let\cal\mathcal
\let\bb\mathbb
\let\cons\veedot
\let\nil\blacksquare
\let\eps\epsilon
\let\ra\rightarrow
\let\lra\longrightarrow
\let\re\hookrightarrow
\let\ctxsep\triangleright
\let\tt\texttt
\newcommand\kw[1]{\ensuremath{\texttt{\upshape{#1}}}}
\newcommand\bool[0]{\kw{bool}}
\newcommand\true[0]{\kw{true}}
\newcommand\false[0]{\kw{false}}
\newcommand\eqb[0]{\kw{eqb}}
\newcommand\andb[0]{\kw{andb}}
\newcommand\eq[0]{\kw{eq}}


% Reification commands
\newcommand\cst[1]{\ensuremath{(\kw{cst}~#1)}}
\newcommand\opp[1]{\ensuremath{\kw{opp}~#1}}
\newcommand\mul[2]{\ensuremath{\kw{mul}~#1~#2}}
\newcommand\var[2]{\ensuremath{(\kw{var}~#1~#2)}}
\newcommand\add[2]{\ensuremath{\kw{add}~#1~#2}}

\newcommand\RAC{\longrightarrow_{\mathcal{R}}^{AC}}
\newcommand\RsAC{\longrightarrow_{\mathcal{R} \slash AC}}
\newcommand\R{\longrightarrow_{\mathcal{R}}}
\newcommand\RRAC{\twoheadrightarrow^{AC}}

\newcommand\reify[1]{\ensuremath{\kw{reify}~#1}}
\newcommand\den[1]{\ensuremath{\kw{den}~#1}}


\newcommand\lpm{\lambda\Pi/\mathop{\equiv}}
\newcommand\pre{\Sigma_{pre}}
\newcommand\prf{\textcolor{purple}{\texttt{\upshape{Prf}}}}
\newcommand\prop{\textcolor{blue}{\texttt{\upshape{Prop}}}}
\newcommand\el{\textcolor{purple}{\texttt{\upshape{El}}}}
\newcommand\set{\textcolor{blue}{\texttt{\upshape{Set}}}}
\newcommand\type{\textcolor{orange}{\texttt{\upshape{TYPE}}}}
\newcommand\kind{\textcolor{YellowOrange}{\texttt{KIND}}}

\input{listing-config}

\begin{document}

\title{Contribution Title}

\author{First Author\inst{1}\orcidID{0000-1111-2222-3333} \and
Second Author\inst{2,3}\orcidID{1111-2222-3333-4444} \and
Third Author\inst{3}\orcidID{2222--3333-4444-5555}}
%
\authorrunning{F. Author et al.}
% First names are abbreviated in the running head.
% If there are more than two authors, 'et al.' is used.
%
\institute{Princeton University, Princeton NJ 08544, USA \and
Springer Heidelberg, Tiergartenstr. 17, 69121 Heidelberg, Germany
\email{lncs@springer.com}\\
\url{http://www.springer.com/gp/computer-science/lncs} \and
ABC Institute, Rupert-Karls-University Heidelberg, Heidelberg, Germany\\
\email{\{abc,lncs\}@uni-heidelberg.de}}
%
\maketitle
%
\begin{abstract}
The abstract should briefly summarize the contents of the paper in
150--250 words.

\keywords{Linear arithmetic  \and SMT \and normal form \and Lambdapi}
\end{abstract}


\section{Alethe proof}
\label{sect:alethe}

The Alethe proof trace format \cite{alethespec} for SMT solvers comprises two parts: the trace language based on SMT-LIB and a collection of proof rules. Traces witness proofs of unsatisfiability of a set of constraints.
They are sequences $a_1 \dots a_m~t_1 \dots t_n$ where
the $a_i$ corresponds to the constraints of the original SMT problem being refuted, each $t_i$ is a clause inferred from previous elements of the sequence, and $t_n$ is $\bot$ (the empty clause).
In the following, we designate the SMT-LIB problem as the \emph{input problem}.

\begin{lstlisting}[language=SMT,label={lst:smtexampleinput}]
(set-logic QF_LIA)
(declare-const x Int)
(declare-const y Int)
(assert (= 0 y))
(assert (= x 2))
(assert (or (< (+ x y) 1) (< 3 x)))
(get-proof)
\end{lstlisting}

\begin{center}
\lightning
\end{center}

\begin{lstlisting}[language=SMT,caption={The following example is the proof for the unsatisfiability of $(x+y < 1) \lor (3<x), x = 2$ and $0 = y$.},label={lst:smtexampleproof}]
(assume a0 (or (< (+ x y) 1) (< 3 x)))
(assume a1 (= x 2))
(assume a2 (= 0 y))
(step t1 (cl (< (+ x y) 1) (< 3 x)) :rule or :premises (a0))
(step t2 (cl (not (< 3 x)) (not (= x 2))) :rule la_generic :args (1/1 1/1))
(step t3 (cl (not (< 3 x))) :rule resolution :premises (a1 t2))
(step t4 (cl (< (+ x y) 1)) :rule resolution :premises (t1 t3))
(step t5 (cl (not (< (+ x y) 1)) (not (= x 2)) (not (= 0 y))) :rule la_generic :args (1/1 -1/1 1/1))
(step t6 (cl) :rule resolution :premises (t5 t4 a1 a2))
\end{lstlisting}

We will use the input problem shown in the top part of \cref{lst:smtexampleinput} with its Alethe proof (found by cvc5) in the bottom part as a running example to provide an overview of Alethe concepts and to illustrate our reconstruction of linear arithmetic step in Lambdapi.

An Alethe proof trace inherits the declarations of its input problem. All symbols (sorts, functions, assertions, etc.) declared or defined in the input problem remain declared or defined, respectively. Furthermore, the syntax for terms, sorts, and annotations uses the syntactic rules defined in SMT-LIB \cite[\S 3]{smtlib} and the SMT signature context defined in \cite[\S 5.1 and \S 5.2]{smtlib}.
In the following we will represent an Alethe step as

\smallskip

\renewcommand{\eqnhighlightshade}{35}

\begin{equation}
\label{eq:step}
\tag{\textcolor{purple}{1}}
\eqnmarkbox[indexClr]{node2}{i}. \quad \eqnmarkbox[blue]{node1}{\Gamma} ~\triangleright~ \eqnmarkbox[green]{node3}{l_1 \dots l_n} \quad (\eqnmarkbox[purple]{node4}{\mathcal{R}}~\eqnmarkbox[red]{node5}{p_1 \dots p_m})~\eqnmarkbox[orange]{node6}{[a_1 \dots a_r]}
\end{equation}

\vspace{0.3em}
\annotate[yshift=-0.5em]{below, left}{node2}{index}
\annotate[yshift=-0.5em]{below, right}{node1}{context}
\annotate[yshift=0.5em]{above, left}{node3}{clause}
\annotate[yshift=-0.5em]{below, right}{node4}{rule}
\annotate[yshift=-0.5em]{below, right}{node5}{premises}
\annotate[yshift=-0.5em]{below, right}{node6}{arguments}

\vspace{0.3em}

\medskip

A step %\cref{eq:step} 
consists of an index \colorbox{indexClr!30}{$i$} $\in \mathbb{I}$ where $\mathbb{I}$ is a countable infinite set of indices (e.g. \kw{a0}, \kw{t1}), and a clause of formulae \colorbox{green!30}{$l_1, \dots, l_n$} representing an $n$-ary disjunction. Steps that are not assumptions are justified by a proof rule \colorbox{purple!30}{$\mathcal{R}$} that depends on a possibly empty set of premises $\{\colorbox{red!30}{$p_1 \dots  p_m$}\} \subseteq \mathbb{I}$ that only references earlier steps such that the proof forms
a directed acyclic graph. A rule might also depend on a list of arguments \colorbox{orange!30}{$[a_1 \dots a_r]$} where each argument $a_i$ is either a term or a pair $(x_i, t_i)$ where $x_i$ is a variable and $t_i$ is a term. The interpretation of the arguments is rule specific. The context \colorbox{blue!30}{$\Gamma$} of a step is a list $c_1 \dots c_l $ where each element $c_j$ is either a variable or a variable-term tuple denoted $x_j \mapsto t_j$. Therefore, steps with a non-empty context contain variables $x_j$ that appear in \colorbox{green!30}{$l_i$} and will be substituted by $t_j$. Proof rules \colorbox{purple!30}{$\mathcal{R}$} include theory lemmas and \texttt{resolution}, which corresponds to hyper-resolution on ground first-order clauses. 


\begin{table}[]
    \centering
    \begin{tabular}{ll}
    Rule & Description \\ \hline
    la\_generic & Tautologous disjunction of linear inequalities. \\
    lia\_generic & Tautologous disjunction of linear integer inequalities. \\
    la\_disequality & $t_1 \approx t_2 \lor \neg (t_1 \geq t_2) \lor \neg (t_2 \geq t_1)$ \\
    la\_totality & $t_1 \geq t_2 \lor t_2 \geq t_1$ \\
    la\_tautology & A trivial linear tautology \\
    la\_rw\_eq & $(t \approx u) \approx (t \geq u \land u \geq t)$ \\
    div\_simplify & Simplification of division. \\
    prod\_simplify & Simplification of products. \\
    unary\_minus\_simplify & Simplification of the unuary minus. \\
    minus\_simplify & Simplification of the substractions. \\
    sum\_simplify & Simplification of sums. \\
    comp\_simplify & Simplification of arithmetic comparisons. \\
    \end{tabular}
    \caption{Linear arithmetic rules in Alethe.}
    \label{table:linear-arith-rules}
\end{table}

We now have the key components to explain the guiding proof in the bottom part of \cref{lst:smtexampleproof}.
The proofs starts with \tt{assume} steps \tt{a0}, \tt{a1}, \tt{a2} that restate the assertions from the \textit{input problem} (\cref{lst:smtexampleproof}).
Step \tt{t1} transforms disjunction into clause by using the Alethe rule \tt{or}.
Steps \tt{t2} and \tt{t5} are tautologies introduced by the main rule \tt{la\_generic}
in LA logic and used also in LIA logic, where \colorbox{green!30}{$\neg l_1, \neg l_2,\dots, \neg l_n$} are linear inequalities. Checking the validity of this clause amounts to checking the unsatisfiability of the
system of linear equations e.g. $x < 3$ and $x = 2$ in \tt{t2}. A coefficient for each inequality are pass as arguments e.g. $(\frac{1}{1},\frac{1}{1})$ in \tt{t2}.
Steps \tt{t3} (and \tt{t4}) applies the \colorbox{purple!30}{\texttt{resolution}} rule to the premises \tt{a1}, \tt{t2} (respectively \tt{t1} \tt{t3}).
Finally, the step \texttt{t6} concludes the proof by generating the empty clause $\bot$, concretely denoted as \kw{(cl)} in \cref{lst:smtexampleproof}. %(line 8) 
Notice that the contexts \colorbox{blue!30}{$\Gamma$} of each step are all empty in this proof.


Proofs for linear arithmetic steps use a number of straightforward rules listed in \cref{table:linear-arith-rules}, such as \tt{la\_totality} $(t_1 \leq t_2 \lor t_2 \geq t_1)$
and the main rules \tt{la\_generic} and \tt{lia\_generic}. The \tt{lia\_generic} rule takes the same form as
\tt{la\_generic}, without the additional coefficients. Since this rule can introduce a disjunction of arbitrary linear integer inequalities without any additional hints, proof checking can be NP-hard.
Although the \tt{la\_generic} rule seems primarily designed for LA logic, it is also employed in LIA proofs when all variables in the (in)equalities are integer-sorted.
While it can produce rational coefficients, it is rarely used in practice with CVC5 proofs.

\section{Lia elaboration}

Carcara considers $\tt{lia\_generic}$ steps as holes in the proof, as verifying them is as difficult.
Currently, Carcara relies on an external tool that generates Alethe proofs to formulate the steps by solving corresponding problems in a proof-producing manner.
The proof is then imported, verified, and validated before replacing the original step.
However, at present, Carcara only use cvc5 for performing this task.
It is important to note that cvc5 has a limitation: its Alethe proofs may contain rewrite steps that are not yet modeled in the Alethe simplification rules, and as such, these steps are not supported by Carcara.
While these steps are considered holes, they typically involve simple simplification rules and, therefore, have much less impact than the more complex $\tt{lia\_generic}$ ones.

In detail, the elaboration method, when encountering a $\tt{lia\_generic}$ step S concluding the negated inequalities $ \neg l_1 \lor \dots \neg l_n$ , generates an SMT-LIB problem asserting $l_1 \land \dots \land l_n$ and invokes \emph{cvc5} on it, expecting an Alethe proof $\pi : (l_1 \land \dots \land l_n) \ra \bot$.
Carcara will check each step in $\pi$ and, if they are not invalid, will replace step $S$ in the original proof by a proof of the form:
\begin{lstlisting}[language=SMT,caption={Elaboration of $\tt{lia\_generic}$},label={lst:smtexampleinput}]
(anchor :step S.t_m+1)
(assume S.h_1 l1)
...
(assume S.h_n ln)
...
(step t.t_m (cl false) :rule ...)
(step t.t_m+1 (cl (not l1) ... (not ln) false) :rule subproof)
(step t.t_m+2 (cl (not false)) :rule false)
(step S (cl (not l1) ... (not ln)) :rule resolution :premises (S.t_m+1 S.t_m+2))
\end{lstlisting}

We decided to leverage the elaboration process of $\tt{lia\_generic}$ performed by Carcara,
as doing otherwise would require implementing Fourier-Motzkin elimination for integers, as demonstrated in \cite{omegatest,micromega} -
and therefore reimplementing the work done by the solver.


\section{Overview of the linear Arithmetic rules in Alethe}

All linear arithmetic tautology rules, such as \tt{la\_disequalities}, \tt{la\_totality},
and simplification rules like \tt{comp\_simplify}, are encoded as lemmas in our embedding of Alethe in Lambdapi, as presented in \cref{sect:recon-lambdapi}.
The \tt{la\_generic} rule, however, must be reconstructed using a different approach, as it involves following the algorithm described below.

A step of the rule \tt{la\_generic} represents a tautological clause of linear
disequalities.  It can be checked by showing that the conjunction of
the negated disequalities is unsatisfiable. After the application of
some strengthening rules, the resulting conjunction is unsatisfiable,
even if integer variables are assumed to be real variables.

A linear inequality is of term of the form
\[
\sum_{i=0}^{n}c_i\times{}t_i +
d_1\bowtie \sum_{i=n+1}^{m} c_i\times{}t_i + d_2
\]

where $\bowtie\;\in \{=, <,
>, \leq, \geq\}$, where $m\geq n$, $c_i, d_1, d_2$ are either integer or real
constants, and for each $i$ $c_i$ and $t_i$ have the same sort. We will write
$s_1 \bowtie s_2$.

Let $l_1,\dots, l_n$ be linear inequalities and $a_1, \dots, a_n$
rational numbers, then a \tt{la\_generic} step has the form

\[
\begin{matrix*}[c]
  i. & \ctxsep & \varphi_1 , \dots , \varphi_n & \tt{la\_generic}  & [a_1, \dots, a_n] \\
\end{matrix*}
\]

\smallskip

The constants $a_i$ are of sort \tt{Real} and must be printed
using one of the productions $\grNT{rational}$
$\grNT{decimal}$, $\grNT{nonpositive\_decimal}$.

\smallskip

To check the unsatisfiability of the negation of $\varphi_1, \dots, \varphi_n$ one performs the following steps for each literal. For
each $i$, let $\varphi := \varphi_i$ and $a := a_i$.


\begin{enumerate}
    \item If $\varphi = s_1 > s_2$, then let $\varphi := s_1 \leq s_2$.
      If $\varphi = s_1 \geq s_2$, then let $\varphi := s_1 < s_2$.
      If $\varphi = s_1 < s_2$, then let $\varphi := s_1 \geq s_2$.
      If $\varphi = s_1 \leq s_2$, then let $\varphi := s_1 > s_2$. This negates
      the literal.
    
    \item If $\varphi = \neg (s_1 \bowtie s_2)$, then let $\varphi := s_1 \bowtie s_2$.
    
    \item If $\varphi = s_1 < s_2$, then let $\varphi :=   - s_1 > - s_2$.
      If $\varphi = s_1 \leq s_2$, then let $\varphi :=  s_1 \geq - s_2$.

    \item Replace $\varphi$ by $\sum_{i=0}^{n}c_i\times{}t_i - \sum_{i=n+1}^{m} c_i\times{}t_i
    \bowtie d$ where $d := d_2 - d_1$.
    
    \item \label{la_generic:str}Now $\varphi$ has the form $s_1 \bowtie d$. If all
    variables in $s_1$ are integer sorted: replace $\bowtie d$ according to
    the table below.
    
    \item If $\bowtie$ is $=$ replace $l$ by
    $\sum_{i=0}^{m}a\times{}c_i\times{}t_i = a\times{}d$, otherwise replace it by
    $\sum_{i=0}^{m}|a|\times{}c_i\times{}t_i = |a|\times{}d$.
    Coefficients are put on the same denominator to keep whole numbers as coefficients.
\end{enumerate}

The replacements that can be performed by step~\ref{la_generic:str} above
are

\begin{tabular}{r l l}
$\bowtie$  & If $d$ is an integer  & Otherwise \\
$>$        & $\geq d + 1$  & $\geq \lfloor d\rfloor + 1$  \\
$\geq$     & $\geq d$      & $\geq \lfloor d\rfloor + 1$  \\
\end{tabular}

Finally, the sum of the resulting literals is trivially contradictory.
The sum
\[
    \sum_{k=1}^{o}\sum_{i=1}^{m^o}c_i^k*t_i^k \bowtie \sum_{k=1}^{o}d^k
\]
where $c_i^k$ is the constant $c_i$ of literal $l_k$, $t_i^k$ is the term
$t_i$ of $l_k$, and $d^k$ is the constant $d$ of $l_k$. The operator
$\bowtie$ is $=$ if all operators are $=$, $>$ if all are
either $=$ or $>$, and $\geq$ otherwise. The $a_i$ must be such
that the sum on the left-hand side is $0$ and the right-hand side is $>0$ (or
$\geq 0$ if $\bowtie$ is $>$).

The step 3 has been added by our own since the following steps in the original algorithm work with $>$ and $\geq$ and does not
mention clearly what to do with $<$ and $\leq$.

\begin{example}
Consider the $\tt{la\_generic}$ step in the logic \tt{LIA}:
\begin{lstlisting}[language=SMT,label={lst:smtexampleinput}]
(step t11 (cl (not (<= f 0)) (<= (+ 1 (* 4 f)) 1))
  :rule la_generic :args (1.0 1/4))
\end{lstlisting}
After step 4, we get $- f > 0 ~\land~ 4 \times f > 0$. The step 5 applied and we can strengthen this to
$- f \geq 0 ~\land~ 4 \times f \geq 1$ and after multiplication of the normalized coefficients we get $4 \times(- f) \geq 0 ~\land~ 4 \times f \geq 1$.
Which sums to the contradiction  $0 \geq 1$. 
\end{example}

In the next section, we first present an overview of our embedding of Alethe in Lambdapi, and then our automatic procedure to reconstruct $\tt{la\_generic}$ step that appear in LIA problem.


\section{Reconstruction of \tt{la\_generic} step for LIA logic}
\label{sec:recon-lambdapi}

\subsection{Encoding of Integers in Lambdapi}


\begin{figure}
\centering
\begin{align*}\label{eq:eq1}
&\bb{P}: \type & &\bb{Z}: \type   & &\tt{Comp}: \type & &\bb{B}: \type \\
&|~\tt{H} : \bb{P} & &|~\tt{Z0}: \bb{Z} & &|~\tt{Eq}: \tt{Comp} & &|~\tt{true}: \bb{B} \\
&|~\tt{O}: \bb{P} \ra \bb{P} & &|~\tt{ZPos}: \bb{P} \ra \bb{Z} & &|~\tt{Lt}: \tt{Comp} & &|~\tt{false}: \bb{B} \\
&|~\tt{I}: \bb{P} \ra \bb{P} & &|~\tt{ZNeg}: \bb{P} \ra \bb{Z} & &|~\tt{Gt}: \tt{Comp} & &\\
&\tt{pos}: \set & &\tt{int}: \set & &\tt{comp}: \set & &\tt{bool}: \set \\
&\el~\tt{pos} \re \bb{P} & &\el~\tt{int} \re \bb{Z} & &\el~\tt{comp} \re \tt{Comp} & &\el~\tt{bool} \re \bb{B}
\end{align*}
\caption{Overview of inductive types for integers and manipulating them}
\label{fig:sorts-constructors}
\end{figure}

The definition we use of integers in Lambdapi in \cref{fig:sorts-constructors} follows a common encoding found in many other theories, including the one adopted in the Rocq standard library \cite{Rocq-refman}.
First, the type $\bb{P}$  is an inductive type representing strictly positive integers in binary form.
Starting from 1 (represented by constructor \tt{H}), one can add a new least significant digit via the constructor \tt{O} (digit 0) or constructor \tt{I} (digit 1). 
The type $\bb{Z}$ is an inductive type representing integers in binary form.
An integer is either zero (with constructor \tt{Z0}) or a strictly positive number \tt{Zpos} (coded as a $\bb{P}$) or a strictly negative number \tt{Zneg} (whose opposite is stored as a $\bb{P}$ value).
%
To enable quantification over integers, positive binary numbers, booleans, and comparison results, we introduce constants such as $\tt{int}: \set$ that represents codes for these types along with a rewrite rule that rewrite code to its corresponding type, for example $\el~\tt{int} \re \bb{Z}$.
The comparison inductive type $\tt{Comp}$ is used to specify comparison function. The infix function $\doteq$ define a decidable order comparison between two terms of type $\bb{Z}$. Similarly for $\bb{P}$ with the infix function $\tt{cmp}: \bb{P} \ra \bb{P} \ra \tt{Comp}$.

\begin{figure}
\centering
\begin{minipage}[t]{0.48\textwidth}
\begin{align*}
&+: \bb{Z} \ra \bb{Z} \ra \bb{Z} \\
& \tt{Z0} + y \re y \\
& x + \tt{Z0} \re x \\
& (\tt{Zpos x}) + (\tt{Zpos y}) \re (\tt{Zpos}~(\tt{add}~x~y))  \\
& (\tt{Zpos x}) + (\tt{Zneg y}) \re (\tt{sub}~x~y)  \\
& (\tt{Zneg x}) + (\tt{Zpos y}) \re (\tt{sub}~y~x)  \\
& (\tt{Zneg x}) + (\tt{Zneg y}) \re \tt{Zpos}(\tt{add}~x~y)  \\
\end{align*}
\hfill
\end{minipage}
\begin{minipage}[t]{0.48\textwidth}
\begin{align*}
&\doteq : \bb{Z} \ra \bb{Z} \ra \tt{Comp} \\
& \tt{Z0} \doteq \tt{Z0} \re \tt{Eq} \\
& \tt{Z0} \doteq \tt{Zpos}~\_ \re \tt{Lt} \\
& \tt{Z0} \doteq \tt{Zneg}~\_ \re \tt{Gt} \\
& \tt{Zpos}~\_ \doteq \tt{Z0} \re \tt{Gt} \\
& \tt{Zpos}~p \doteq \tt{Zpos}~q \re \tt{cmp}~p~q \\
& \tt{Zpos}~\_ \doteq \tt{Zneg}~\_ \re \tt{Gt} \\
& \tt{Zneg}~\_ \doteq \tt{Z0} \re \tt{Lt} \\
& \tt{Zneg}~\_ \doteq \tt{Zpos}~\_ \re \tt{Lt} \\
& \tt{Zneg}~p \doteq \tt{Zneg}~q \re \tt{cmp}~q~p \\
\end{align*}
\end{minipage}
\noindent
\[
\begin{array}{l@{\hspace{4em}}l@{\hspace{4em}}l}
\begin{aligned}
  &\tt{isEq} : \tt{Comp} \ra \bb{B} \\
  &\tt{isEq}~\tt{Eq} \re \tt{true} \\
  &\tt{isEq}~\tt{Lt} \re \tt{false} \\
  &\tt{isEq}~\tt{Gt} \re \tt{false} \\
\end{aligned}
&
\begin{aligned}
  &\tt{isLt} : \tt{Comp} \ra \bb{B} \\
  &\tt{isLt}~\tt{Eq} \re \tt{false} \\
  &\tt{isLt}~\tt{Lt} \re \tt{true} \\
  &\tt{isLt}~\tt{Gt} \re \tt{false} \\
\end{aligned}
&
\begin{aligned}
  &\tt{isGt} : \tt{Comp} \ra \bb{B} \\
  &\tt{isGt}~\tt{Eq} \re \tt{false} \\
  &\tt{isGt}~\tt{Lt} \re \tt{false} \\
  &\tt{isGt}~\tt{Gt} \re \tt{true} \\
\end{aligned}
\end{array}
\]
\noindent
\begin{align*}
&\leq: \bb{Z} \ra \bb{Z} \ra \prop  \coloneq \lambda x,\lambda y, \neg (\tt{istrue}(\tt{isGt}(x \doteq y))) & &\tt{istrue} : \bb{B} \ra \prop \\
&<: \bb{Z} \ra \bb{Z} \ra \prop  \coloneq \lambda x,\lambda y, (\tt{istrue}(\tt{isLt}(x \doteq y))) & &\tt{istrue}~\tt{true} \re \top \\
&\geq: \bb{Z} \ra \bb{Z} \ra \prop  \coloneq \lambda x,\lambda y, \neg (x < y) & &\tt{istrue}~\tt{false} \re \bot \\
&>: \bb{Z} \ra \bb{Z} \ra \prop  \coloneq \lambda x,\lambda y, \neg (x \leq y) & & \\
\end{align*}
\caption{Decidable equality, $+$ operator and inequalities relations definitions for $\bb{Z}$}
\label{fig:arith-ops}
\end{figure}

The arithmetic operator such as $+$ are constants defined by rewriting rules. In the following sections, we will refers to the rewriting rules for integers as $\ra_\bb{Z}$ and positive binary numbers as $\ra_\bb{P}$.
The confluence of the rewriting rules for the arithmetic of $\mathbb{Z}$ and $\mathbb{P}$ has been proven using CSI \cite{CSI}. A detailed proof of confluence can be found in \cref{app:confluence-int-pos}.
The inequality symbols for $\bb{Z}$ are binary predicates defined by rewriting rules over the decidable equality $\doteq$. They reduce to $\top$, $\bot$ (or negated) by $\equivL$ with rules of $\ra_\bb{Z}$ and $\ra_\bb{P}$.
For example, $1 < 2 \hookrightarrow \tt{istrue}(\tt{isLt}(1 \doteq 2)) \hookrightarrow \tt{istrue}(\tt{isLt}(\tt{Lt})) \hookrightarrow \tt{istrue}(\tt{true}) \hookrightarrow \top$.

\subsection{Functions used in the translation}

We now outline the encoding of arithmetic expressions from SMT-LIB \cite[\S 5.2.1]{smtlib}. This extends the approach introduced in \cite{ColtellacciMD24} to handle arithmetic constructs.
To avoid notational conflicts with the Lambdapi signature $\Sigma$, we denote the set of SMT-LIB sorts as $\Theta^\mathcal{S}$, the set of function symbols $\Theta^\cal{F}$, and the set of variables $\Theta^\cal{X}$.
Our translation is based on the following functions:

\begin{itemize}
\item $\cal{D}$ translates declarations of sorts and functions in $\Theta^\cal{S}$ and $\Theta^\mathcal{F}$ into constants,
\item $\cal{S}$ maps sorts to $\Sigma$ types,
\item $\cal{E}$ translates SMT expression to $\lpm$ terms,
\item $\cal{C}$ translates a list of commands  $c_1 \dots c_n$ of the form $i.~\Gamma \triangleright~\varphi~(\mathcal{R}~P)[A]$ to typing judgments $\Gamma \vdash_\Sigma i := M: N$.
\end{itemize}

\begin{definition}[Function $\mathcal{D}$ translating SMT sort and function symbol declarations]
For each sort symbol $s$ with arity $n$ in $\Theta^\cal{S}$ we create a constant $s: \set^0 \ra \dots \ra^{n-1} \set$.
For each function symbol $f~\sigma^+$ in $\Theta^\cal{F}$ we create a constant $f: \cal{S}(\sigma^+)$.
\end{definition}

\begin{definition}[Function $\mathcal{S}$ translating sorts of expression] 
  The definition of $\mathcal{S}$(s) is as follows.
  \begin{itemize}
    \item Case $s = \textbf{Bool}$, then $\Sort{s} = \el\,o$,
    \item Case $s = \textbf{Int}$, then $\Sort{s} = \el~\texttt{int}$,
    \item Case $s = \sigma_1\,\sigma_2 \dots \sigma_n$ then $\Sort{s} = \el{} (\mathcal{S}(\sigma_1) \leadsto \dots \leadsto \mathcal{S}(\sigma_n))$,
    \item otherwise $\Sort{s} = \el\, \mathcal{D}(s)$.
  \end{itemize}
\end{definition}

\begin{definition}[Function $\mathcal{E}$ translating SMT expressions]
The definition of $\E{e}$ is as follows.
\begin{itemize}
\setlength{\parskip}{0pt}
\item Case e $= (+~t_1\dots~t_n)$, then $\E{e} = \E{t_1} + ~\dots~ +\E{t_n}$.
\item Case e $= (*~t_1\dots~t_n)$, then $\E{e} = \E{t_1} * ~\dots~ *\E{t_n}$.
\item Case e $= (-t)$, then $\E{e} = \mathop{\sim} \E{t_1}$.
\item Case e $= (-~t_1\dots~t_n)$, then $\E{e} = \E{t_1} - ~\dots~ - \E{t_n}$.
\item Case e $= (\approx~t_1~t_2)$ then $\E{e} = (\E{t_1} = \E{t_2})$.
\item Case $e = (x: \sigma )$ with $x \in \Theta^\mathcal{X}$ a sorted variable, then $\E{e} = x: \cal{S}(\sigma)$.
\end{itemize}
\end{definition}

We defined the function $\pid: \tt{Clause} \ra \type$, mapping each clause $c$ to the type $\pid~c$ of its proofs.
The type $\tt{Clause}: \type$ represent the type of clause encoded as list \cite[\S 3]{ColtellacciMD24}
with the constructor $\veedot: \prop \ra \tt{Clause} \ra \tt{Clause}$ and the empty clause $\nil: \tt{clause}$.
The function $\cal{C}$ encodes each step by invoking the functions $\cal{D},\cal{S}$ and $\cal{E}$, and provides a proof term corresponding to the Alethe rule applied at that step.

\begin{example}
The translation of the steps t2 and t5 in \cref{lst:smtexampleproof} with the input problem definitions give us:
\begin{lstlisting}[language=Lambdapi,mathescape=true]
symbol x: El int;
symbol y: El int;
...
opaque symbol t2: $\pid$ (¬ (3 < x)  ⟇ ¬ (x = 2) ⟇ ▩) ≔ begin ... end;
opaque symbol t5: $\pid$ (¬ (x + y < 1) ⟇ (¬ (x = 2))  ⟇ (¬ (0 = y)) ⟇ ▩) ≔ begin ... end;
...
\end{lstlisting}
\end{example}

The proof terms generated by $\mathcal{C}$ for steps \texttt{t2} and \texttt{t5} must faithfully represent the algorithm presented in \cref{sssect:la-in-alethe}.
While steps 1 through 7 of the algorithm correspond to explicit rewriting steps, the final step (step 8) — which involves summing all inequalities — represents a multi-step rewriting sequence.
This sequence reduced the initial sum to a decidable comparison between constants (e.g. $0 > 1 \re \bot$), which serves as the conclusion of the reduction.


The reflection technique introduced by \cite{reflection-origin-coq} leverages the reduction system of the proof assistant to produce an efficient decidable automatic
procedure for solving arithmetic goals over $\bb{Q}, \bb{R}$ and $\bb{Z}$. We decided to follow this approach to implement our decision procedure for evaluating inequality.
In the following section, we describe how we implemented this procedure for our case, and how we extended the definition of $\mathcal{C}$.

\section{Reconstruction of linear arithmetic for LIA logic}
\label{sec:lia-reconstruction}

Proof by reflection is a technique used to write certified automation procedure by reducing the validity of a logical statement to a symbolic computation.
It relies on the following definitions: let $P: Z \ra \prop$ be a predicate over a data type Z and we have a function $f: Z \ra \tt{bool}$ such that the following theorem holds:

\begin{equation*}
\tt{f\_correct} : \forall z: Z, (f~z = \tt{true}) \ra (P~z)
\end{equation*}

If $\mathop{f} z$ reduces to \tt{true}, then the proof term  $\tt{f\_correct}~z~(\tt{relf}\,\tt{bool}\,\tt{true})$ with $\tt{refl}: \Pi A: \set, \Pi x: \el\,A, \pic (x = x)$, constitutes a proof of predicate $(P~z)$. In step 6 of the $\tt{la\_generic}$,
the primary challenge lies in reasoning modulo associativity and commutativity when manipulating expressions over $\bb{Z}$.
The key idea is to provide a normalization function that transforms a $\bb{Z}$ expression into a canonical form,
such that it can be reduce to a constant because variables will cancel each other, as is the case with $f$ in \cref{ex:la_generic_example_red}.


\subsection{Representation}
\label{ssec:representation}

The procedure is based on a group structure, denoted as  $\bb{G}$ defined in \cref{fig:grp} which represents linear polynomials.
The base type for the elements of this group is specified as $\bb{G}: \type$. The unary operator $\tt{cst}$ denotes a constant from $\bb{Z}$.
A $\tt{var}$ constructor for "catch-all" case for subexpressions that we cannot model. These subexpressions will correspond to actual variables in $\bb{Z}$.
The constructor $\tt{mul}$  represents the multiplication of an element of $\bb{G}$ by a constant. The constructor $\tt{opp}$ corresponds to the inverse operator within the group.
Lastly, the constructor $\tt{add}$ represent the addition between two elements of \cref{fig:grp}.

To support associative and commutative operations, Lambdapi provides the modifiers \lstinline[language=Lambdapi,basicstyle=\ttfamily\footnotesize\upshape]{associative commutative symbol},
ensuring that terms are systematically placed into a canonical form given a builtin ordering relation, as described in \cite{ACorigin} and \cite[\S 5]{univAC}.
Applying the \lstinline[language=Lambdapi,basicstyle=\ttfamily\footnotesize\upshape]{associative commutative} modifiers to the constructor $\tt{add}$,
ensures that expressions involving sums of products are systematically canonicalized. Thus, equal variables are placed next to each other, facilitating simplification.

\begin{figure}
\begin{align*}
& \bb{G}: \type & & \reify{} : \bb{Z} \ra \bb{G} & & \den{}: \bb{G} \ra \bb{Z} \\
&|~\tt{add}: \bb{G} \ra \bb{G} \ra \bb{G} & & \reify{\tt{Z0}} \re \cst{\tt{Z0}} & & \den{\cst{c}} \re c \\
&|~\tt{var}: \bb{Z} \ra \bb{Z} \ra \bb{G} & & \mathop{\reify{\tt{ZPos}}} c \re \cst{c} & & \den{\opp{x}} \re  \sim (\den{x}) \\
&|~\tt{mul}: \bb{Z} \ra \bb{G} \ra \bb{G} & & \mathop{\reify{\tt{ZNeg}}} c \re \cst{c} & & \den{\mul{c}{x}} \re  c \times (\den{x}) \\
&|~\tt{opp}: \bb{G} \ra \bb{G} & & \reify{(x + y)} \re \add{(\reify{x}})({\reify{y})} & & \den{\add{x}{y}} \re (\den{x}) + (\den{y}) \\
&|~\tt{cst}: \bb{Z} \ra \bb{G} & & \reify{(\sim x)} \re \opp{\reify{x}} & & \den{\var{c}{x}} \re  c \times x \\
&\tt{grp}: \set & & \mathop{\reify{(\mathop{\tt{Zpos}} c) * x}} \re \mul{c}{(\reify{x})}  & & \\
&\el~\tt{grp} \re \bb{G} & & \mathop{\reify{(\mathop{\tt{ZNeg}} c) * x}} \re \mul{c}{(\reify{x})} & & \\
& & & \mathop{\reify{x * (\mathop{\tt{Zpos}} c)}} \re \mul{c}{(\reify{x})}  & & \\
& & & \mathop{\reify{x * (\mathop{\tt{ZNeg}} c)}} \re \mul{c}{(\reify{x})} & & \\
& & & \mathop{\reify{x}} \re \var{1}{x} & &
\end{align*}
\caption{Definition of $\bb{G}$  Algebra and its reification ($\reify{}$) and denotation ($\den{}$) functions}
\label{fig:grp}
\end{figure}

\subsection{Normalization}
\label{ssec:normalization}


$\reify{(x + y + (- x) + (- y))} \re 
\var{1}{x} + \var{1}{y} + \tt{opp}(\var{1}{x}) + \tt{opp}(\var{1}{y}) \re 
\var{1}{x} + \var{1}{y} + \var{-1}{x} + \var{-1}{y} 
\simeq_{AC} \var{1}{x} + \var{-1}{x} + \var{1}{y} + \var{-1}{y}
\re \cst{0}$

\begin{definition}[AC-canonical form]
Let $\leq$ be any total order on $\bb{G}$-terms with $\tt{cst}(c) < \var{p}{x}$ for all $c$, and $\var{p}{x} \leq \var{q}{y}$ iff $x < y$ or else $x = y$ and $p \leq q$.
\end{definition}

% The AC-canonization of a term $t$ of $\cal{C}$ is defined as $[t]_{AC} = \mathit{comb}_{\sqcup} [\texttt{sort}(\mathit{aliens}_{\sqcup}(t))]$, where $\texttt{sort(l)}$ is the list of the elements of $l$ in increasing order with respect to $\leq$. The relation associating every term $t$ with its AC-canonization $[t]_{AC}$ is denoted $\RRAC$. Two terms $t$ and $t'$ are AC-equivalent if $[t]_{AC} = [t']_{AC}$.
% The term $t$ is in AC-canonical form if $t = [t]_{AC}$ and if every strict subterm of $t$ is in AC-canonical form. 

\begin{figure}
\begin{align*}
&\add{\var{x}{c_1}}{\var{x}{c2}} \re \kw{var}~x~(c_1 + c_2) \\
&\add{\var{x}{c_1}}{(\add{\var{x}{c_2}}{y})} \re \add{\var{x}{(c_1 + c_2)}}{y} \\
&\add{\cst{c_1}}{\cst{c_2}} \re \cst{c_1 + c_2} \\
&\add{\cst{c_1}}{(\add{\cst{c_2}}{y})} \re \add{\cst{c_1 + c_2}}{y} \\
&\add{\cst{0}}{x} \re x \\
&\add{x}{\cst{0}} \re x \\
&\opp{\var{x}{c}} \re \var{x}{(-c)} \\
&\opp{\cst{c}} \re \cst{(-c)} \\
&\opp{\opp{x}} \re x \\
&\opp{\add{x}{y}} \re \add{(\opp{x})}{(\opp{y})} \\
&\mul{k}{\var{x}{c}} \re \var{x}{(k * c)} \\
&\mul{k}{\opp{x}} \re \mul{(-k)}{x} \\
&\mul{k}{(\add{x}{y})} \re \add{(\mul{k}{x})}{(\mul{k}{y})} \\
&\mul{k}{\cst{c}} \re \cst{(k * c)} \\
&\mul{c_1}{(\mul{c_2}{x})} \re \mul{(c_1 * c_2)}{x} \\
\end{align*}
\caption{Rewrite system on canonical forms}
\label{fig:grp-rw}
\end{figure}
\todo[ac]{mettre les coeffs a gauche}


% https://q.uiver.app/#q=WzAsOCxbMSwyLCJcXG1hdGhiYntafSJdLFsxLDAsIlxcbWF0aGNhbHtSfSJdLFszLDAsIlxcbWF0aGNhbHtSfSJdLFszLDIsIlxcbWF0aGJie1p9Il0sWzAsMiwidF8xID1fXFxtYXRoYmJ7Wn0gdF8yIl0sWzQsMiwiZGVub3RlKHRfMSkgPV9cXG1hdGhiYntafSBkZW5vdGUodF8yKSJdLFswLDAsInJlaWZ5KHRfMSkgPV9cXG1hdGhjYWx7Un0gcmVpZnkodF8yKSJdLFs0LDAsIm5vcm0odF8xKSA9X1xcbWF0aGNhbHtSfSBub3JtKHRfMikiXSxbMCwxLCJyZWlmeSJdLFsxLDIsIlxccmlnaHRhcnJvd197QUN9IiwwLHsic3R5bGUiOnsiYm9keSI6eyJuYW1lIjoiZG90dGVkIn19fV0sWzIsMywiZGVub3RlIl0sWzAsMywiXFxlcXVpdiIsMSx7InN0eWxlIjp7ImJvZHkiOnsibmFtZSI6ImRvdHRlZCJ9LCJoZWFkIjp7Im5hbWUiOiJub25lIn19fV1d
\begin{tikzcd}[ampersand replacement=\&,column sep=small]
	{\reify{(t_1)} =_\bb{Z} \reify{(t_2)}} \& {\bb{G}} \&\& {\bb{G}} \& {{t_1\downarrow_{AC}} =_\bb{Z} t_2\downarrow_{AC}} \\
	\\
	{t_1 =_\mathbb{Z} t_2} \& {\mathbb{Z}} \&\& {\mathbb{Z}} \& {\den{({t_1\downarrow_{AC}})} =_\mathbb{Z} \den{({t_2\downarrow_{AC}})}}
	\arrow["{\rightarrow_{AC}}", dotted, from=1-2, to=1-4]
	\arrow["denote", from=1-4, to=3-4]
	\arrow["reify", from=3-2, to=1-2]
	\arrow["\iff"{description}, dotted, no head, from=3-2, to=3-4]
\end{tikzcd}

% \begin{definition}
%     Let $\mathit{aliens}_{\sqcup}: \mathcal{C} \rightarrow \mathcal{C}^+$ be the function mapping every term in $\mathcal{C}$ to a non-empty list of terms such that $\mathit{aliens}_{\sqcup}(t) = \mathit{aliens}_{\sqcup}(u) \circ \mathit{aliens}_{\sqcup}(v)$ if $t = u \sqcup v$, and $\mathit{aliens}_{\sqcup}(t) = [t]$ otherwise.
  
%     Conversely, let $\mathit{comb}_{\sqcup} \colon \mathcal{C}^+ \rightarrow \mathcal{C}$ be the function mapping a non-empty list of $\mathcal{C}$-terms to a term such that  $\mathit{comb}_{\sqcup}[t] = t$ and for all $n \geq 2,  \mathit{comb}_{\sqcup}[t_1, \dots, t_n] = t_1 \sqcup \mathit{comb}_{\sqcup}[t_2,\dots,t_n]$.
%   \end{definition}
  
%   \smallskip 
  
% For example $\mathit{aliens}_{\sqcup}((x \sqcup y) \sqcup z) = [x, y, z]$ and $\mathit{comb}_{\sqcup}[x, y, z] = ((x \sqcup y) \sqcup z)$.

% \begin{example}
% Assuming that the terms $x$ and $y$ are ordered $x < y$, the AC-canonical form of $XXX$ is $XXX$.
% \end{example}
  
\smallskip
  
\begin{definition}[Rewriting modulo AC-canonization]
Let $\RAC = \RRAC \R$, where $\mathcal{R}$ is defined by the rewrite rules of ??.
\label{def:RAC}
\end{definition}
  
\smallskip
  
An $\RAC$ step is an AC-canonization followed by a standard $\lra_{\mathcal{R}}$ step with syntactic matching.
  
\subsection{Generation of the proof term for \tt{la\_generic} in \tt{LIA}}
\label{ssec:gen-lia-proof}

\section{Evaluation}
\label{sec:evaluation}

\begin{table}[]
\centering
\begin{tabular}{|c|c|c|c|c|c|c|}
\hline                                                             % Succ - Err -Timeout   % Succ - Error        % Succ - Err - Timeout % Succ - Err - Timeout
\textbf{Logic}                & \textbf{Bench}  & \textbf{Samples} & \textbf{Proofs}       & \textbf{Elaborate} & \textbf{Translate} & \textbf{Check} \\ \hline
\multirow{4}{*}{\tt{LIA}}     & tptp            & 36               &  36 - 0 - 0           &  36 - 0             & 36 - 0 - 0          & 28 - 8 - 0       \\ \cline{2-7} 
                              & Ultimate        & 153              &  120 - 0 - 33         &  73 - 80            & 68 - 5 - 0          & 50 - 18 - 0      \\ \cline{2-7} 
                              & Svcomp'19      & 27               &  27 - 0 - 0           &  25 - 2             & 0 - 25 - 0          & 0                \\ \cline{2-7} 
                              & psyco           & 50               &  48 - 0 - 2           &  48 - 2             & 43 - 0 - 5          & 0 - 39 - 6       \\ \hline
\multirow{2}{*}{\tt{QFLIA}}   & SMPT            & 1568             &  1501 - 67 - 0        &  1491 - 32          & 1470 - 0 - 21       & 804 - 638 - 34   \\ \cline{2-7}
                              & rings           & 294              &  63 - 0 - 231         &  49 - 21            & 49 - 0 - 0          & 7 - 0 - 42       \\ \cline{2-7} 
                              & CAV\_2009       & 85               &  85 - 0 - 0           &  19 - 0             & 19 - 0 - 0          & 19 - 0 - 0       \\ \hline
\multirow{2}{*}{\tt{UFLIA}}   & sledgeh    & 1521             &  1343 - 0 - 178       &  1278 - 222         & 1258 - 13 - 7       & 713 - 467 - 80   \\ \cline{2-7} 
                              & tokeneer        & 1732             &  1732 - 0 - 0         & 1689 - 43           & 1689 - 0 - 0        & 1482 - 197 - 10  \\ \hline
\end{tabular}
\caption{Benchmark samples.}
\label{table:benchmarks-description}
\end{table}

Our benchmark suite, \cref{table:benchmarks-description} is composed of files from the SMT-LIB benchmarks\footnote{\url{https://smtlib.cs.uiowa.edu/benchmarks.shtml}}.
The suite includes a total of 5,466 samples drawn from 9 benchmarks spanning 3 SMT-LIB logics: \texttt{LIA}, \texttt{QFLIA}, and \texttt{UFLIA} that correspond to those covered by our method.
%These were selected because our previous work supported the \texttt{QF} and \texttt{UF} fragments.
Within the logics \texttt{QFLIA} and \texttt{UFLIA}, we prioritized benchmarks with the most significant number of available samples.
Table~\ref{table:benchmarks-description} provides a detailed breakdown of the benchmarks and corresponding results.
The \emph{Logic} column indicates the SMT theory used, while the \emph{Bench} column lists abbreviated benchmark names for conciseness.
The column \emph{Samples} describes the number of problems available with status \emph{unsat}. 
Each of the columns \emph{Proofs}, \emph{Translate}, and \emph{Check} reports a triple in the format \tt{success - error - timeout}, representing respectively the number of successful executions, failed attempts, and timeouts. 
The \emph{Proofs} column reports the number of proofs generated by cvc5\footnote{cvc5 version 1.2.1-dev.144.38fcc340e5 [git 38fcc340e5 on branch main]} that do not contain \lstinline[language=SMT,basicstyle=\ttfamily\footnotesize]|Real| or \lstinline[language=SMT,basicstyle=\ttfamily\footnotesize]|to_real| cast operator.
The \emph{Elaborate} column shows a pair of values: the number of proofs that were successfully elaborated by Carcara, and the number that failed. Only proofs successfully elaborated by Carcara are considered for translation.
The \emph{Translate} column gives the number of proof traces successfully translated into Lambdapi proofs, while the \emph{Check} column indicates the number of these translated proofs that were successfully type-checked by Lambdapi. 

We enforced a timeout of 30 seconds for cvc5 to find a proof and 30 seconds for the translation step with Carcara.
No timeout was imposed during the elaboration step, as the runtime is negligible.
A timeout of 20 seconds was set for Lambdapi when type-checking the final proofs to ensure that proof verification remains as fast as possible.\footnote{All benchmarks were executed in parallel using GNU parallel \cite{tange_2025_15071920} with an Apple silicon M2.} % Ask by the tool to be cited

All nine benchmarks demonstrated consistently reliable proof generation, with few or no timeouts, except for the \emph{rings} benchmark.
The elaboration phase was generally robust, except in the \emph{Ultimate} benchmark, where an error in the elaborator caused failures.
For \texttt{LIA}, performance in the translation and checking stages was mixed. In particular, for \emph{Svcomp'19} and \emph{psyco}, no or few proofs could be translated or verified, mainly due to unsupported simplification rules involving the \texttt{ite} operator.
The \texttt{QFLIA} benchmarks exhibited more reliable proof checking in \emph{SMPT} and \emph{CAV\_2009}, while only a small number of proofs from the \emph{rings} benchmark were successfully checked.
The errors encountered during the elaboration of the \emph{CAV\_2009} come from a bug in the elaborator. Since the samples in this benchmark are derived from a common base problem and increase incrementally in size, the bug propagated across all samples that depend on the same base instance.
This limitation is due to the presence of \texttt{la\_generic} terms with hundreds of uninterpreted variables; the current normalization mechanism, described in \cref{ssec:normalization}, relies on a built-in term ordering that is not sufficiently efficient in this setting.
Benchmarks under \texttt{UFLIA} performed better throughout the pipeline. Most proofs were successfully generated and elaborated, and a large portion were translated and verified by Lambdapi, particularly in the \emph{tokeneer} dataset.
The higher number of errors in \emph{SMPT} and \emph{sledgeh} is primarily due to unsupported RARE rule related to \tt{QF} and \tt{UF} logic, as well as unhandled cases in the \texttt{evaluate} rule for \texttt{LIA}, \texttt{QF}, and \texttt{UF}.

\begin{table}[]
\centering
\begin{tabular}{|c|c|c|c|c|c|}
\hline
\textbf{Bench}      & \textbf{Min} & \textbf{Q1} & \textbf{Mean} & \textbf{Q3} & \textbf{Max} \\ \hline
tptp                & 240          & 262         & 263           & 267         & 336          \\ \hline
Ultimate            & 82           & 96          & 100           & 111         & 346          \\ \hline
SMPT                & 52           & 55          & 55            & 56          & 293          \\ \hline
rings               & 670          & 704         & 773           & 826         & 1197         \\ \hline
CAV\_2009           & 54           & 296         & 403           & 498         & 683          \\ \hline
sledgeh             & 53           & 166         & 354           & 552         & 1594         \\ \hline
tokeneer            & 55           & 60          & 61            & 248         & 753          \\ \hline
\end{tabular}
\caption{Lambdapi checking times in milliseconds.}
\label{table:benchmarks-list}
\end{table}

Table~\ref{table:benchmarks-list} reports the Lambdapi proof-checking times in milliseconds, presenting the minimum, first quartile (Q1), mean, third quartile (Q3), and maximum durations for each benchmark.
In most cases, the checking time remains below one second.
% , with the exception of a few outliers particularly in the \emph{sledgehammer} and \emph{rings} benchmarks that exhibit higher upper tails.


\bibliographystyle{splncs04}
\bibliography{refs}


\appendix


\section{Alethe}
\label{app:alethe}


\subsection{The Syntax}

\begin{figure}[htb]%[H]
    \[
      \begin{array}{r c l}
     \grNT{proof}           &\grRule & \grNT{proof\_command}^{*} \\
     \grNT{proof\_command}  &\grRule & \textAlethe{(assume}\; \grNT{symbol}\; \grNT{proof\_term}\,\textAlethe{)} \\
                            &\grOr   & \textAlethe{(step}\; \grNT{symbol}\; \grNT{clause}
                                            \; \textAlethe{:rule}\; \grNT{symbol} \\
                            &        & \quad \grNT{premises\_annotation}^{?} \\
                            &        & \quad \grNT{context\_annotation}^{?}\;\grNT{attribute}^{*}\,\textAlethe{)} \\
                            & \grOr  & \textAlethe{(anchor :step}\; \grNT{symbol}\;
                                                \\
                            &        & \quad \grNT{args\_annotation}^{?}\;\grNT{attribute}^{*}\,\textAlethe{)} \\
                            & \grOr  & \textAlethe{(define-fun}\; \grNT{function\_def}\,\textAlethe{)} \\
     \grNT{clause}          &\grRule & \textAlethe{(cl}\; \grNT{proof\_term}^{*}\,\textAlethe{)} \\
     \grNT{proof\_term}     &\grRule & \grNT{term}\text{ extended with } \\
                            &        & \textAlethe{(choice (}\, \grNT{sorted\_var}\,\textAlethe{)}\; \grNT{proof\_term}\,\textAlethe{)}  \\
     \grNT{premises\_annotation} &\grRule & \textAlethe{:premises (}\; \grNT{symbol}^{+}\textAlethe{)} \\
     \grNT{args\_annotation}     &\grRule & \textAlethe{:args}\,\textAlethe{(}\,\grNT{step\_arg}^{+}\,\textAlethe{)}  \\
     \grNT{step\_arg}            &\grRule & \grNT{symbol} \grOr
                                              \textAlethe{(}\; \grNT{symbol}\; \grNT{proof\_term}\,\textAlethe{)} \\
     \grNT{context\_annotation}  &\grRule & \textAlethe{:args}\,\textAlethe{(}\,\grNT{context\_assignment}^{+}\,\textAlethe{)}  \\
     \grNT{context\_assignment}  &\grRule & \textAlethe{(}    \,\grNT{sorted\_var}\,\textAlethe{)}  \\
                                 & \grOr  & \textAlethe{(:=}\, \grNT{symbol}\;\grNT{proof\_term}\,\textAlethe{)} \\
      \end{array}
      \]
      \caption{Alethe grammar}
      \label{fig:grammar}
\end{figure}

%  Lambdapi Definitions for Integer and Binary Arithmetic
\section{Lambdapi Formalizations for Integer and Binary Number Operations}
\label{app:lambdapi-func-def}

\noindent
\begin{minipage}[t]{0.48\textwidth}
% \textbf{\ttfamily add : \bb{P} \ra \bb{P} \ra \bb{P}}

\begin{align*}
&\tt{add} : \bb{P} \ra \bb{P} \ra \bb{P} \\
& \tt{add}~(\tt{I}~x)~(\tt{I}~q) \re \tt{O}~(\tt{add\_c}~x~q) \\
& \tt{add}~(\tt{I}~x)~(\tt{O}~q) \re \tt{I}~(\tt{add}~x~q) \\
& \tt{add}~(\tt{O}~x)~(\tt{I}~q) \re \tt{I}~(\tt{add}~x~q) \\
& \tt{add}~(\tt{O}~x)~(\tt{O}~q) \re \tt{O}~(\tt{add}~x~q) \\
& \tt{add}~x~\tt{H} \re \tt{succ}~x \\
& \tt{add}~\tt{H}~y \re \tt{succ}~y \\
\end{align*}
\end{minipage}
\hfill
\begin{minipage}[t]{0.48\textwidth}
\begin{align*}
&\tt{add\_c} : \bb{P} \ra \bb{P} \ra \bb{P} \\
& \tt{add\_c}~(\tt{I}~x)~(\tt{I}~q) \re \tt{I}~(\tt{add\_c}~x~q) \\
& \tt{add\_c}~(\tt{I}~x)~(\tt{O}~q) \re \tt{O}~(\tt{add\_c}~x~q) \\
& \tt{add\_c}~(\tt{O}~x)~(\tt{I}~q) \re \tt{O}~(\tt{add\_c}~x~q) \\
& \tt{add\_c}~(\tt{O}~x)~(\tt{O}~q) \re \tt{I}~(\tt{add}~x~q) \\
& \tt{add\_c}~x~\tt{H} \re \tt{add}~x~(\tt{O}~\tt{H}) \\
& \tt{add\_c}~\tt{H}~y \re \tt{add}~(\tt{O}~\tt{H})~y \\
\end{align*}
\end{minipage}

\begin{align*}
&\tt{sub} : \bb{P} \ra \bb{P} \ra \bb{Z} \\
& \tt{sub}~(\tt{I}~p)~(\tt{I}~q) \re \tt{double}~(\tt{sub}~p~q) \\
& \tt{sub}~(\tt{I}~p)~(\tt{O}~q) \re \tt{succ\_double}~(\tt{sub}~p~q) \\
& \tt{sub}~(\tt{I}~p)~\tt{H} \re \ZPos (\tt{O}~p) \\
& \tt{sub}~(\tt{O}~p)~(\tt{I}~q) \re \tt{pred\_double}~(\tt{sub}~p~q) \\
& \tt{sub}~(\tt{O}~p)~(\tt{O}~q) \re \tt{double}~(\tt{sub}~p~q) \\
& \tt{sub}~(\tt{O}~p)~\tt{H} \re \ZPos (\tt{pos\_pred\_double}~p) \\
& \tt{sub}~\tt{H}~(\tt{I}~q) \re \ZNeg    (\tt{O}~q) \\
& \tt{sub}~\tt{H}~(\tt{O}~q) \re \ZNeg    (\tt{pos\_pred\_double}~q) \\
& \tt{sub}~\tt{H}~\tt{H} \re \tt{Z0} \\
\end{align*}

\begin{align*}
&\tt{compare\_acc} : \bb{P} \ra \tt{Comp} \ra \bb{P} \ra \tt{Comp} \\
& \tt{compare\_acc}~(\tt{I}~x)~c~(\tt{I}~q) \re \tt{compare\_acc}~x~c~q \\
& \tt{compare\_acc}~(\tt{I}~x)~\_~(\tt{O}~q) \re \tt{compare\_acc}~x~\tt{Gt}~q \\
& \tt{compare\_acc}~(\tt{I}~\_)~\_~\tt{H} \re \tt{Gt} \\
& \tt{compare\_acc}~(\tt{O}~x)~\_~(\tt{I}~q) \re \tt{compare\_acc}~x~\tt{Lt}~q \\
& \tt{compare\_acc}~(\tt{O}~x)~c~(\tt{O}~q) \re \tt{compare\_acc}~x~c~q \\
& \tt{compare\_acc}~(\tt{O}~\_)~\_~\tt{H} \re \tt{Gt} \\
& \tt{compare\_acc}~\tt{H}~\_~(\tt{I}~\_) \re \tt{Lt} \\
& \tt{compare\_acc}~\tt{H}~\_~(\tt{O}~\_) \re \tt{Lt} \\
& \tt{compare\_acc}~\tt{H}~c~\tt{H} \re c \\
\\
&\tt{compare}~x~y \coloneq \tt{compare\_acc}~x~\tt{Eq}~y \\
\end{align*}
  
\begin{minipage}[t]{0.45\textwidth}
\begin{align*}
&\tt{mul} : \bb{P} \ra \bb{P} \ra \bb{P} \\
& \tt{mul}~(\tt{I}~x)~y \re \tt{add}~y~(\tt{O}~(\tt{mul}~x~y)) \\
& \tt{mul}~(\tt{O}~x)~y \re \tt{O}~(\tt{mul}~x~y) \\
& \tt{mul}~\tt{H}~y \re y \\
& \tt{mul}~y~\tt{H} \re y \\
\end{align*}
\end{minipage}
\begin{minipage}[t]{0.48\textwidth}
\begin{align*}
&\tt{*} : \bb{Z} \ra \bb{Z} \ra \bb{Z} \\
& \tt{Z0} *~\_ \re \tt{Z0} \\
& \_ *~\tt{Z0} \re \tt{Z0} \\
& \ZPos x *~\ZPos y \re \ZPos (\tt{mul}~x~y) \\
& \ZPos x *~\ZNeg    y \re \ZNeg    (\tt{mul}~x~y) \\
& \ZNeg    x *~\ZPos y \re \ZNeg    (\tt{mul}~x~y) \\
& \ZNeg    x *~\ZNeg    y \re \ZPos (\tt{mul}~x~y) \\
\end{align*}
\end{minipage}

\subsection{Confluence of the rewriting rules of integers and positive binary number}
\label{app:confluence-int-pos}

The rules presented below represent the relations $\ra_\bb{Z}$ and $\ra_\bb{P}$ encoded in the TRS\footnote{\url{http://www.lri.fr/~marche/tpdb/format.html}} format accepted by the \cite{CSI} tool.
These rules can be used to rerun the tool in order to verify the confluence property.

\begin{lstlisting}[language=trs, caption=Rewriting rule of $\bb{Z}$ and $\bb{P}$ in the TRS format]
(VAR
  a: Z
  b: Z
  x : P
  q : P
  y : P
)
(RULES
  ~(Z0) -> Z0
  ~(Zpos(p)) -> Zneg(p)
  ~(Zneg(p)) -> Zpos(p)
  ~(~(a)) -> a

  double(Z0) -> Z0
  double(Zpos(p)) -> Zpos(O(p))
  double(Zneg(p)) -> Zneg(O(p))
  
  succ_double(Z0) -> Zpos(H)
  succ_double(Zpos(p)) -> Zpos(I(p))
  succ_double(Zneg(p)) -> Zneg(pos_pred_double(p))
  
  pred_double(Z0) -> Zneg(H)
  pred_double(Zpos(p)) -> Zpos(pos_pred_double(p))
  pred_double(Zneg(p)) -> Zneg(I(p))

  sub(I(p), I(q)) -> double(sub(p, q))
  sub(I(p), O(q)) -> succ_double(sub(p, q))
  sub(I(p), H) -> Zpos(O(p))
  sub(O(p), I(q)) -> pred_double(sub(p, q))
  sub(O(p), O(q)) -> double(sub(p, q))
  sub(O(p), H) -> Zpos(pos_pred_double(p))
  sub(H, I(q)) -> Zneg(O(q))
  sub(H, O(q)) -> Zneg(pos_pred_double(q))
  sub(H, H) -> Z0

  +(Z0,a) -> a
  +(a,Z0) -> a
  +(Zpos(x), Zpos(y)) -> Zpos(add(x, y))
  +(Zpos(x), Zneg(y)) -> sub(x, y)
  +(Zneg(x), Zpos(y)) -> sub(y, x)
  +(Zneg(x), Zneg(y)) -> Zneg(add(x, y))
  
  mult(Z0, a) -> Z0
  mult(a, Z0) -> Z0
  mult(Zpos(x), Zpos(y)) -> Zpos(mul(x, y))
  mult(Zpos(x), Zneg(y)) -> Zneg(mul(x, y))
  mult(Zneg(x), Zpos(y)) -> Zneg(mul(x, y))
  mult(Zneg(x), Zneg(y)) -> Zpos(mul(x, y))


  succ(I(x)) -> O(succ(x))
  succ(O(x)) -> I(x)
  succ(H) -> O(H)
  add(I(x), I(q)) -> O(addcarry(x, q))
  add(I(x), O(q)) -> I(add(x, q))
  add(O(x), I(q)) -> I(add(x, q))
  add(O(x), O(q)) -> O(add(x, q))
  add(x, H) -> succ(x)
  add(H, y) -> succ(y)

  addcarry(I(x), I(q)) -> I(addcarry(x, q))
  addcarry(I(x), O(q)) -> O(addcarry(x, q))
  addcarry(O(x), I(q)) -> O(addcarry(x, q))
  addcarry(O(x), O(q)) -> I(add(x, q))
  addcarry(x, H) -> add(x, O(H))
  addcarry(H, y) -> add(O(H), y)
  
  pos_pred_double(I(x)) -> I(O(x))
  pos_pred_double(O(x)) -> I(pos_pred_double(x))
  pos_pred_double(H) -> H
  
  mul(I(x), y) -> add(x, O(mul(x,y)))
  mul(O(x), y) -> O(mul(x, y))
  mul(H, y) -> y
)
\end{lstlisting}


% Proof:
%  Church Rosser Transformation Processor (no redundant rules):
%   strict:
   
%   weak:
   
% critical peaks: 4
%   add(O(H()),H()) <-6|[]- addcarry(H(),H()) -7|[]-> add(H(),O(H()))
%   add(H(),O(H())) <-7|[]- addcarry(H(),H()) -6|[]-> add(O(H()),H())
%   succ(H()) <-12|[]- add(H(),H()) -13|[]-> succ(H())
%   succ(H()) <-13|[]- add(H(),H()) -12|[]-> succ(H())

%   Church Rosser Transformation Processor (critical pair closing system, Thm 2.4):
%   add(x,H()) -> succ(x)
%   add(H(),y) -> succ(y)
%   critical peaks: joinable
%   Matrix Interpretation Processor: dim=1
    
%     interpretation:
%     [succ](x0) = 4x0,
    
%     [add](x0, x1) = 4x0 + 4x1 + 2,
    
%     [H] = 4
%     orientation:
%     add(x,H()) = 4x + 18 >= 4x = succ(x)
    
%     add(H(),y) = 4y + 18 >= 4y = succ(y)
%     problem:
    
%     Qed


% Critical pair diagram
\section{Normalization details}

% t ≔ opp (add (var $k $x) (var $c $x))
% t ↪[] add (opp (var $k $x)) (opp (var $c $x)) ↪* var (— $k + — $c) $x
%   with opp (add $x' $y') ↪ add (opp $0') (opp $1')
% t ↪[1] opp (var ($0 + $2) $1) ↪* var (— ($0 + $2)) $1
%   with add (var $k $x) (var $c $x) ↪ var ($0 + $2) $1
\begin{center}
\cp
{
  \opp{\underline{\var{c_1}{x} \oplus \var{c_2}{x}}}
}
{
  \opp{\var{c_1}{x}} \oplus \opp{\var{c_2}{x}}
}
{
  \opp{\var{(c_1 + c_2)}{x}}
}
{11}{2}
\end{center}

% t ≔ opp (add (var $k $x) (add (var $l $x) $y))
% t ↪[] add (opp (var $k $x)) (opp (add (var $l $x) $y)) ↪* add (var (— $k + — $l) $x) (opp $y)
%   with opp (add $x' $y') ↪ add (opp $0') (opp $1')
% t ↪[1] opp (add (var ($0 + $2) $1) $3) ↪* add (var (— ($0 + $2)) $1) (opp $3)
%   with add (var $k $x) (add (var $l $x) $y) ↪ add (var ($0 + $2) $1) $3
\cp
{
  \opp{\underline{(\var{c_1}{x} \oplus (\var{c_2}{x} \oplus y))}}
}
{
  \opp{\var{c_1}{x}} \oplus \opp{(\var{c_2}{x} \oplus y)}
}
{
  \opp{\var{(c_1 + c_2)}{x} \oplus y}
}
{11}{3}

% t ≔ opp (add (cst $k) (cst $l))
% t ↪[] add (opp (cst $k)) (opp (cst $l)) ↪* cst (— $k + — $l)
%   with opp (add $x' $y') ↪ add (opp $0') (opp $1')
% t ↪[1] opp (cst ($0 + $1)) ↪* cst (— ($0 + $1))
%   with add (cst $k) (cst $l) ↪ cst ($0 + $1)
\cp{
  \opp{\underline{\cst{c_1} \oplus \cst{c_2}}}
}
{
  \opp{\cst{c_1}} \oplus \opp{\cst{c_2}}
}
{
  \opp{\cst{(c_1 + c_2)}}
}
{11}{4}


% t ≔ opp (add (cst $k) (add (cst $l) $y))
% t ↪[] add (opp (cst $k)) (opp (add (cst $l) $y)) ↪* add (cst (— $k + — $l)) (opp $y)
%   with opp (add $x' $y') ↪ add (opp $0') (opp $1')
% t ↪[1] opp (add (cst ($0 + $1)) $2) ↪* add (cst (— ($0 + $1))) (opp $2)
%   with add (cst $k) (add (cst $l) $y) ↪ add (cst ($0 + $1)) $2
\cp
{
  \opp{\underline{\cst{c_1} \oplus (\cst{c_2} \oplus y)}}
}
{
  \opp{\cst{c_1}} \oplus \opp{(\cst{c_2} \oplus y)}
}
{
  \opp{(\cst{c_1 + c_2} \oplus y)}
}
{11}{5}

% t ≔ add (var $k' $x) (add (var $k $x) (add (var $l $x) $y))
% t ↪[] add (var ($0' + $k) $x) (add (var $l $x) $y) ↪* add (var (($0' + $k) + $l) $x) $y
%   with add (var $k' $x') (add (var $l' $x') $y') ↪ add (var ($0' + $2') $1') $3'
% t ↪[1] add (var $k' $x) (add (var ($0 + $2) $1) $3) ↪* add (var ($k' + ($0 + $2)) $x) $3
%   with add (var $k $x) (add (var $l $x) $y) ↪ add (var ($0 + $2) $1) $3

\cp
{
  \var{c_1}{x} \oplus \underline{(\var{c_2}{x} \oplus (\var{c_3}{x} \oplus y))}
}
{
  \var{(c_1 + c_2)}{x} \oplus ((\var{c_3}{x} \oplus y))
}
{
  \var{c_1}{x} \oplus (\var{(c_2 + c_3)}{x} \oplus y))
}{3}{3}

% t ≔ add (cst $k') (add (cst $k) (add (cst $l) $y))
% t ↪[] add (cst ($0' + $k)) (add (cst $l) $y) ↪* add (cst (($0' + $k) + $l)) $y
%   with add (cst $k') (add (cst $l') $y') ↪ add (cst ($0' + $1')) $2'
% t ↪[1] add (cst $k') (add (cst ($0 + $1)) $2) ↪* add (cst ($k' + ($0 + $1))) $2
%   with add (cst $k) (add (cst $l) $y) ↪ add (cst ($0 + $1)) $2

\cp
{\cst{c_1} \oplus \underline{(\cst{c_2} \oplus (\cst{c_3} \oplus y))}}
{\cst{c_1 + c_2} \oplus (\cst{c_3} \oplus y) }
{\cst{c_1} \oplus (\cst{c_2 + c_3} \oplus y) }
{5}{5}

% t ≔ mul $k' (mul $k (mul $l $z))
% t ↪[] mul ($0' * $k) (mul $l $z) ↪* mul (($0' * $k) * $l) $z
%   with mul $k' (mul $l' $z') ↪ mul ($0' * $1') $2'
% t ↪[1] mul $k' (mul ($0 * $1) $2) ↪* mul ($k' * ($0 * $1)) $2
%   with mul $k (mul $l $z) ↪ mul ($0 * $1) $2

\cp
{ \mul{k'}{\underline{(\mul{k}{(\mul{l}{z})})}} } 
{ \mul{(k')}{ (\mul{(k * l)}{z}) } }
{ \mul{(k' * k)}{(\mul{l}{z}})} 
{16}{16}


\begin{lemma}[Confluence]
\begin{proof}
CSI automatically proves the confluence of $\ra_\bb{Z}$ and $\ra_\bb{P}$ by giving the polynomial interpretation:
\begin{align*}
[\tt{succ}(x)] = 4*x & &[\tt{add}(x, y)] = 4 * x + 4 * y + 2  & &[ {\tt{H}} ] = 4 \\
\end{align*}
\end{proof} 
\end{lemma}

\section{Example}
\label{app:example-translation}

We present here the proof \cref{lst:smtexampleproof} translated with our module added to Carcara.

\lstinputlisting[language=Lambdapi,mathescape=true,caption={Lambdapi proof of \cref{lst:smtexampleproof}}]{lp_example.lp}

\end{document}
