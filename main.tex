\documentclass[runningheads]{llncs}
\usepackage[T1]{fontenc}
\usepackage{graphicx}
\usepackage{listings}
\usepackage{mathtools}
\usepackage[dvipsnames]{xcolor}
\usepackage{wasysym}
\usepackage{annotate-equations}
\usepackage{quiver}
\usepackage{cleveref}
\usepackage{ebproof}
\usepackage[T1]{fontenc}
\usepackage{wasysym}

% comments with
% https://mirror.ox.ac.uk/sites/ctan.org/macros/latex/contrib/todonotes/todonotes.pdf
\usepackage[colorinlistoftodos]{todonotes}


%%%% configuration and new commands  %%%%

% ===== Unicode Alethe encoding in Lambdapi =========
\DeclareUnicodeCharacter{03C0}{\texttt{Prf}}
\DeclareUnicodeCharacter{27F9}{$\Rightarrow$}
\DeclareUnicodeCharacter{2254}{$\coloneqq$}
\DeclareUnicodeCharacter{2091}{$_e$}
\DeclareUnicodeCharacter{22A5}{$\bot$}
\DeclareUnicodeCharacter{25A9}{$\blacksquare$}
\DeclareUnicodeCharacter{0307}{$^{\bullet}$}
\DeclareUnicodeCharacter{27C7}{$\veedot$}
\DeclareUnicodeCharacter{2228}{$\lor$}
\DeclareUnicodeCharacter{1D9C}{$^c$}
\DeclareUnicodeCharacter{1D62}{$_i$}
\DeclareUnicodeCharacter{2081}{$_1$}
\DeclareUnicodeCharacter{2097}{$_l$}
\DeclareUnicodeCharacter{1D63}{$_r$}
\DeclareUnicodeCharacter{21AA}{$\hookrightarrow$}
\DeclareUnicodeCharacter{03C4}{\textcolor{blue}{\texttt{El}}}
\DeclareUnicodeCharacter{2115}{$\mathbb{N}$}
\DeclareUnicodeCharacter{03A0}{$\Pi$}
\DeclareUnicodeCharacter{2227}{$\land$}
\DeclareUnicodeCharacter{21D2}{$\Rightarrow$}
\DeclareUnicodeCharacter{2200}{$\forall$}
\DeclareUnicodeCharacter{2933}{$\leadsto$}
\DeclareUnicodeCharacter{2203}{$\exists$}
\DeclareUnicodeCharacter{03F5}{$\epsilon$}
\DeclareUnicodeCharacter{21D4}{$\Leftrightarrow$}
\DeclareUnicodeCharacter{22A4}{$\top$}
\DeclareUnicodeCharacter{2250}{$\doteq$}
\DeclareUnicodeCharacter{25A1}{$\square$}
\DeclareUnicodeCharacter{2082}{$_2$}

% ===== Grammar Alethe =========

\definecolor{SmtBlue}{HTML}{00007f}
\definecolor{SmtGreen}{HTML}{3b7f31}
\definecolor{SmtStepId}{HTML}{3b7f31}
\definecolor{indexClr}{HTML}{ffcc00}

\newcommand{\grNT}[1]{\textcolor{SmtGreen}{\langle\texttt{#1}\rangle}}
\newcommand{\grT}[1]{\textcolor{SmtBlue}{\texttt{#1}}}
\newcommand{\grRule}{=}
\newcommand{\grOr}{|}
\newcommand{\concat}{~+\!\!\!+~}

% ===== Command for expression in the paper =========

\newcommand\textAlethe[1]{\texttt{#1}}
\newcommand\C[1]{\mathcal{C}(#1)}
\newcommand\D[1]{\mathcal{D}(#1)}
\newcommand\F[1]{\mathcal{F}(#1)}
\newcommand\E[1]{\mathcal{E}(#1)}
\newcommand\Sort[1]{\mathcal{S}(#1)}
\newcommand\equivL{\equiv_{\beta\Sigma}}
\newcommand\pid{\textcolor{purple}{\texttt{\upshape{Prf}}^\bullet}}
\newcommand\pic{\textcolor{purple}{\texttt{\upshape{Prf}}^c}}

% ========== Alias ==========
\let\cal\mathcal
\let\bb\mathbb
\let\nil\blacksquare
\let\eps\epsilon
\let\ra\rightarrow
\let\lra\longrightarrow
\let\re\hookrightarrow
\let\ctxsep\triangleright
\let\tt\texttt
\newcommand\kw[1]{\ensuremath{\texttt{\upshape{#1}}}}
\newcommand\bool[0]{\kw{bool}}
\newcommand\true[0]{\kw{true}}
\newcommand\false[0]{\kw{false}}
\newcommand\eqb[0]{\kw{eqb}}
\newcommand\andb[0]{\kw{andb}}
\newcommand\eq[0]{\kw{eq}}


% ==========  Reification commands ==========
\newcommand\cst[1]{\ensuremath{(\kw{cst}~#1)}}
\newcommand\opp[1]{\ensuremath{\kw{opp}~#1}}
\newcommand\mul[2]{\ensuremath{\kw{mul}~#1~#2}}
\newcommand\var[2]{\ensuremath{(\kw{var}~#1~#2)}}
\newcommand\add[2]{\ensuremath{#1 \oplus #2}} 

\newcommand\RAC{\longrightarrow_{\mathcal{R}}^{AC}}
\newcommand\RsAC{\longrightarrow_{\mathcal{R} \slash AC}}
\newcommand\R{\longrightarrow_{\mathcal{R}}}
\newcommand\RRAC{\twoheadrightarrow^{AC}}

\newcommand\rwModAC{\longrightarrow_{\Sigma/AC}}
\newcommand\ACcanon{\longrightarrow^{AC}_\Sigma}

\newcommand\reify[1]{\ensuremath{\Uparrow #1}}
\newcommand\den[1]{\ensuremath{\Downarrow #1}}
% \newcommand\reify[1]{\ensuremath{\kw{reify}~#1}}
% \newcommand\den[1]{\ensuremath{\kw{den}~#1}}


\newcommand\lpm{\lambda\Pi/\mathop{\equiv}}
\newcommand\pre{\Sigma_{pre}}
\newcommand\prf{\textcolor{purple}{\texttt{\upshape{Prf}}}}
\newcommand\prop{\textcolor{blue}{\texttt{\upshape{Prop}}}}
\newcommand\el{\textcolor{purple}{\mathop{\texttt{\upshape{El}}}}}
\newcommand\set{\textcolor{blue}{\texttt{\upshape{Set}}}}
\newcommand\type{\textcolor{orange}{\texttt{\upshape{TYPE}}}}
\newcommand\kind{\textcolor{YellowOrange}{\texttt{KIND}}}

% ========== veedot symbol for Clause ==========

\makeatletter
\newcommand{\genericdot@}[2]{% #1=raising factor, #2=symbol
  \mathbin{\mathpalette\genericdot@@{{#1}{#2}{\cdot}}}%
}
\newcommand{\genericdotop@}[2]{% #1=raising factor, #2=symbol
  \DOTSB\mathop{\mathpalette\genericdot@@{{#1}{#2}{\boldsymbol{\cdot}}}}\slimits@
}
\newcommand{\genericdot@@}[2]{\genericdot@@@#1#2}
\newcommand{\genericdot@@@}[4]{%
  \vphantom{#3}%
  \begingroup
  \sbox\z@{$\m@th#1#3$}%
  \ooalign{%
    \usebox{\z@}\cr
    \hidewidth
    \raisebox{#2\ht\z@}[\z@][\z@]{$\m@th#1#4$}%
    \hidewidth\cr
  }%
  \endgroup
}
\newcommand{\veedot}{\genericdot@{0.3}{\vee}}
\newcommand{\bigveedot}{\genericdotop@{0.25}{\bigvee}}
\makeatletter

\let\cons\veedot
\newcommand\ZPos{\mathop{\tt{ZPos}}}
\newcommand\ZNeg{\mathop{\tt{ZNeg}}}

% ========== Comments ==========

\todostyle{sm}{color=red}
\todostyle{ac}{color=green}

\lstdefinestyle{mystyle}{
  basicstyle=\ttfamily\footnotesize\upshape,
  breakatwhitespace=false,
  breaklines=true,
  breakindent=0em,
  captionpos=b,
  keepspaces=true,
  numbers=left,
  numbersep=5pt,
  showspaces=false,
  showstringspaces=false,
  showtabs=false,
  tabsize=2,
  frame = single,
  numberstyle=\footnotesize,
}

\lstset{style=mystyle}

\lstdefinelanguage{trs}
{
  numbers=left,
  numbersep=5pt,
  inputencoding=utf8,
  extendedchars=true,
  numberstyle=\footnotesize,
  tabsize=2,
  basicstyle={\ttfamily\scriptsize\upshape},
  keywords=[1]{RULES, VAR, ->},
  alsoletter={*,/,+},
  sensitive=true,
  keywordstyle={[1]\bfseries\color{ForestGreen}},
  string=[b]{"},
  stringstyle=\color{orange},
  showstringspaces=false,
}

\lstdefinelanguage{SMT}
{
  numbers=left,
  numbersep=5pt,
  inputencoding=utf8,
  extendedchars=true,
  numberstyle=\footnotesize,
  tabsize=2,
  basicstyle={\ttfamily\scriptsize\upshape},
  keywords=[1]{anchor,step, :step,assume, declare-fun, declare-const, declare-sort, assert},
  keywords=[2]{:rule, :args,:premises,get-proof, check-sat, set-logic},
  keywords=[3]{and, or, not, distinct,forall, Int, Real, abs, div, mod},
  alsoletter={-,:,*,/},
  sensitive=true,
  keywordstyle={[1]\bfseries\color{ForestGreen}},
  keywordstyle={[2]\bfseries\color{blue}},
  keywordstyle={[3]\bfseries\color{purple}},
  string=[b]{"},
  stringstyle=\color{orange},
  showstringspaces=false,
}

\lstdefinelanguage{Lambdapi}
{
  numbers=left,
  numbersep=5pt,
  inputencoding=utf8,
  extendedchars=true,
  numberstyle=\footnotesize,
  tabsize=2,
  basicstyle={\ttfamily\scriptsize\upshape},
  keywords=[1]{abort,admit,admitted,apply,as,assert,assertnot,assume,builtin,compute,debug,fail,flag,focus,generalize,have,in,induction,inductive,infix,injective,left,let,notation,off,on,open,prefix,print,private,proofterm,protected,prover,prover_timeout,quantifier,refine,reflexivity,require,rewrite,right,rule,sequential,simplify,solve,symmetry,type,TYPE,unif_rule,verbose,why3,with},
  keywords=[2]{El, Prf},
  keywords=[3]{begin, symbol, end,associative,commutative,constant,opaque},
  keywords=[4]{Set,Prop},
  sensitive=true,
  keywordstyle={[1]\bfseries\color{purple}},
  keywordstyle={[2]\bfseries\color{purple}},
  keywordstyle={[3]\bfseries\color{violet}},
  keywordstyle={[4]\bfseries\color{blue}},
  morecomment=[l]{//},
  morecomment=[n]{/*}{*/},
  commentstyle={\itshape\color{red}},
  string=[b]{"},
  stringstyle=\color{orange},
  showstringspaces=false,
  literate=
  {λ}{$\lambda$}1
  {↪}{$\hookrightarrow$}2
  {→}{$\rightarrow$}2
  {Π}{$\Pi$}1
  {≔}{$\coloneqq$}1
  {⊢}{$\vdash$}1
  {≡}{$\equiv$}1
  {𝔹}{$\mathbb{B}$}1
  {𝕃}{$\mathbb{L}$}1
  {ℕ}{$\mathbb{N}$}1
  {α}{$\alpha$}1
  {β}{$\beta$}1
  {η}{$\eta$}1
  {π}{$\pi$}1
  {⤳}{$\rightcurvedarrow$}1
  {𝑰}{$\mathcal{I}$}1
  {ω}{$\omega$}1
  {∧}{$\wedge$}1
  {≤}{$\le$}1
  {≠}{$\neq$}1
  {∉}{$\notin$}1
  {×}{$\times$}1
  {⋅}{$\cdot$}1
  {⟇}{$\veedot$}1
  {▩}{$\blacksquare$}1
  {□}{$\Square$}1
  {⊤}{$\top$}1
  {φ₁}{$\varphi_1$}1
  {ₙ}{$_n$}1
  {ₖ}{$_k$}1
  {π}{$\pi$}1
  {¬}{$\neg$}1
  { ̇}{$\cdot$}1
  {∨}{$\lor$}1
  {≐}{$\doteq$}1
  {□}{$\square$}1
  {ᶜ}{${^c}$}1
  {ᵢ}{${\_i}$}1
  {₁}{${\_1}$}1
  {—}{$\sim$}1
  {⇒}{$\Rightarrow$}2
  {ₑ}{${\_e}$}1
  {ₗ}{${\_l}$}1
  {≥}{$\geq$}2
  {ᵣ}{$\_r$}1
  {⇑}{$\Uparrow$}1
  {⇓}{$\Downarrow$}1
}



\renewcommand{\floatpagefraction}{.99}

\begin{document}

% \title{Verifying SMT-Based Linear Integer Arithmetic in Lambdapi}
% \title{SMT Linear Integer Arithmetic proof checking in Lambdapi}
\title{Checking Linear Integer Arithmetic Proofs\\ in Lambdapi}

% \titlerunning{Verification of LIA proof with Lambdapi}

\author{Alessio Coltellacci\inst{1}\orcidID{0009-0005-3580-2075}
  \and
  Stephan Merz\inst{1}\orcidID{0000-0003-0974-1844}
}
%
\authorrunning{A. Coltellacci, S. Merz}
% First names are abbreviated in the running head.
% If there are more than two authors, 'et al.' is used.
%
\institute{University of Lorraine, CNRS, Inria, LORIA, Nancy 
%\email{alessio.coltellacci@inria.fr}
}

% \institute{University of Lorraine, CNRS, Inria, LORIA \and
% Nancy
% \email{alessio.coltellacci@inria.fr}\\
% \url{http://www.springer.com/gp/computer-science/lncs} \and
% ABC Institute, Rupert-Karls-University Heidelberg, Heidelberg, Germany\\
% \email{\{abc,lncs\}@uni-heidelberg.de}}
%
\maketitle
%
\begin{abstract}
Modern SMT solvers can generate proofs of unsatisfiability to ensure correctness independently of their implementation.
A dependable approach to verify these proofs is to reconstruct them within a proof assistant.
In previous work, the SMT checker Carcara was extended to reconstruct SMT proofs in Lambdapi — a proof assistant designed for interoperability,
supporting the import and export of proofs for integration with other proof assistants such as Rocq, Lean, or HOL-Light.
Whereas that work was limited to SMT theories without arithmetic, we here present an extension that enables the reconstruction of SMT proofs involving linear integer arithmetic.

\keywords{SMT \and Alethe \and integer arithmetic \and Lambdapi \and normal form \and proof by reflection}
\end{abstract}

\section{Introduction}

SMT solvers have become capable of producing proofs of unsatisfiability,  enabling independent verification of correctness without relying on the solver's implementation.
This development addresses the growing need for trustworthy verification in safety-critical applications and SMT solver usage in proof assistant, where the solver cannot be assumed to be fully trusted.
Alethe \cite{alethe,alethespec} is an established SMT proof format supported by the solvers cvc5 and veriT. In our previous work \cite{ColtellacciMD24}, we extended the Alethe proof checker and elaborator Carcara \cite{carcara} to
reconstruct Alethe SMT proofs within the Lambdapi proof assistant, thereby ensuring their validity.
Lambdapi \cite{lambdapi} is a proof assistant based on the $\lambda\Pi$-calculus modulo rewriting \cite{lpmodulo}, a logical framework \cite{lf} that extends the $\lambda$-calculus with dependent types and user-defined rewrite rules.
This foundation allows Lambdapi to serve as a framework for formalizing various logical systems.
Designed with interoperability in mind, Lambdapi can import and export proofs, facilitating integration with other proof assistants \cite{LPAR2024:Translating_HOL_Light_proofs} such as Rocq \cite{Rocq-refman}, Lean \cite{lean4:2021}, and HOL-Light. 

However, our prior work was limited to proofs expressed within the logic of Uninterpreted Functions (UF) and Quantifier-Free (QF), as well as their subtheories. 
In the present work, we extend this approach to support the reconstruction of proof steps involving linear integer arithmetic (LIA).
When the propositional model is unsatisfiable in the theory of linear (integer) arithmetic, the solver creates a proof certificate.
The conclusion is a tautological clause of linear inequalities and equations and the justification of the step is a list of coefficients so that the linear combination becomes a trivially contradictory inequality after simplification (e.g., $0 \geq 1$).
Most SMT solvers, including cvc5 and veriT, use the simplex method \cite{SRI:simplex:dpllt} to handle linear arithmetic.\todo[sm]{Is that relevant for proof checking?}
To verify these certificates, we implemented an automatic decision procedure based on proof by reflection \cite{reflection-origin-coq,ring-coq}.

The remainder of this paper is organized as follows.
In \cref{sec:background}, we provide an overview of Lambdapi and present the structure of Alethe proof certificates for linear arithmetic.
Then, in \cref{sec:elaboration-lia}, we describe how we leverage Carcara's elaboration process to reconstruct linear integer arithmetic steps, even when coefficient annotations are missing.
In \cref{sec:encoding} introduces our encoding of SMT linear arithmetic within Lambdapi, while in \cref{sec:lia-reconstruction} presents an automatic decision procedure, based on proof by reflection, to verify these arithmetic steps.
An empirical evaluation of our approach is provided in \cref{sec:evaluation}. We review related work in \cref{sec:related}, and we conclude in \cref{sec:conclusion}.

\section{Background}
\label{sect:background}

\subsection{An Overview of Lambdapi}
\label{ssect:lambdapi-overview}

Lambdapi is an implementation of $\lambda\Pi$ modulo theory ($\lpm$) \cite{lambdapi}, an extension of the Edinburgh Logical Framework $\lambda\Pi$ \cite{lf}, a simply typed $\lambda$-calculus with dependent types. $\lpm$ adds user-defined higher-order rewrite rules. Its syntax is given by
%
\begin{align*}
&\text{Universes}  &u &::= \tt{TYPE} ~|~ \tt{KIND} \\
&\text{Terms}   &t,v, A,B,C &::= c ~|~ x ~|~ u ~|~ \Pi\,x : A,\,B~|~ \lambda\,x : A,\,t ~|~t~v \\
&\text{Contexts}   &\Gamma &::= \langle \rangle ~|~ \Gamma, x : A \\
&\text{Signatures}  &\Sigma &::= \langle \rangle ~|~ \Sigma, c : C ~|~ \Sigma, c := t : C ~|~ \Sigma, t \hookrightarrow v 
\end{align*}
%
where $c$ is a constant and $x$ is a variable  (ranging over disjoint sets), $C$ is a closed term. \emph{Universes} are constants used to verify if a type is well-formed -- more details can be found in \cite[\S 2.1]{lf}. $\Pi\,x : A,\,B$ is the dependent product, and we write $A \rightarrow B$ when $B$ does not depend on $x$, $\lambda\,x : A.\,t$ is an abstraction, and  $t~v$ is an application. A \emph{(local) context} $\Gamma$ is a finite sequence of variable declarations $x:A$ introducing variables and their types.
A \emph{signature} $\Sigma$ representing the global context is a finite sequence of \emph{assumptions} $c : C$, indicating that constant $c$ is of type $C$, \emph{definitions} $c := t : C$, indicating that $c$ has the value $t$ and type $C$, and \emph{rewrite rules} $t \hookrightarrow v$ such that $t = c~v_1 \dots v_n$ where $c$ is a constant.

The relation $\hookrightarrow_{\beta\Sigma}$ is generated by $\beta$-reduction and by the rewrite rules of $\Sigma$. The relation $\hookrightarrow_{\beta\Sigma}^*$ denotes the reflexive and transitive closure of $\hookrightarrow_{\beta\Sigma}$, and the relation $\equiv_{\beta\Sigma}$ (called \emph{conversion}) the reflexive, symmetric, and transitive closure of $\hookrightarrow_{\beta\Sigma}$. 
The relation $\hookrightarrow_{\beta\Sigma}$ must be confluent, i.e.,
whenever $t \hookrightarrow_{\beta\Sigma}^* v_1$ and $t \hookrightarrow_{\beta\Sigma}^* v_2$, there exists a term $w$ such that $v_1 \hookrightarrow_{\beta\Sigma}^* w$ and $v_2 \hookrightarrow_{\beta\Sigma}^* w$, and it must preserve typing, i.e., 
whenever $\Gamma \vdash_\Sigma t: A$ and $t \hookrightarrow_{\beta\Sigma} v$ then $\Gamma \vdash_\Sigma v: A$ \cite{blanqui:LIPIcs.FSCD.2020.13}.

A Lambdapi typing judgment $\Gamma \vdash_\Sigma t : A$ asserts that term $t$ has type $A$ in the context $\Gamma$ and the signature $\Sigma$.
The typing rules of $\lpm$ are the one of  $\lambda\Pi$ \cite[\S 2]{lf}, except for the rule (Conv) where it use the version of \cref{fig:lp-typing-rules} that identifies types modulo~$\textcolor{orange}{\equiv_{\beta\Sigma}}$ instead of just modulo $\beta$-reduction. 

\begin{figure}
    \begin{center}
    \begin{prooftree}
    \hypo{\Gamma, \vdash_\Sigma B: u}
    \hypo{\Gamma \vdash_\Sigma t: A}
    \hypo{\textcolor{orange}{A \equiv_{\beta\Sigma} B}}
    \infer3[(Conv)]{ \Gamma \vdash_\Sigma t: B }
    \end{prooftree}
    \end{center}
    \caption{(Conv) rule in $\lpm$}
    \label{fig:lp-typing-rules}
\end{figure}

In our encoding presented \cite{ColtellacciMD24} we employ Tarski-style universe \cite[\S Universes]{intuitype} where types are represented by elements of a base type and interpreted via the decoding function.
We defined the constant $\prop: \type$ for the type of proposition and the decoding function $\pic: \prop \ra \type$ maps each proposition to $\type$.
Lambdapi does not support quantifiy over a variable of type $\type$. More precisely, it is not possible to assign the type $\Pi X : \type,(X \ra \prop) \ra \prop$ to the universal quantifier $\forall$.
To address this, we have the constant $\set : \type$ for the types of object-terms, and a decoding function ${\el}: \set \ra \type$ to embed the terms of type $\set$ into terms of type $\type$ giving us the quantifier $\forall: \Pi x: \set, (\el~x \ra \prop) \ra \prop$.
We have the constant $o: \set$ with the rewrite rule $\el\, o \re \prop$ to quantify over propositions.

\subsection{Alethe proof}
\label{ssect:alethe}

The Alethe proof trace format \cite{alethespec} for SMT solvers comprises two parts: the trace language based on SMT-LIB and a collection of proof rules. Traces witness proofs of unsatisfiability of a set of constraints.
They are sequences $a_1 \dots a_m~t_1 \dots t_n$ where
the $a_i$ corresponds to the constraints of the original SMT problem being refuted, each $t_i$ is a clause inferred from previous elements of the sequence, and $t_n$ is $\bot$ (the empty clause).
In the following, we designate the SMT-LIB problem as the \emph{input problem}.

\begin{lstlisting}[language=SMT,label={lst:smtexampleinput},caption={Input problem}]
(set-logic QF_LIA)
(declare-const x Int)
(declare-const y Int)
(assert (= 0 y))
(assert (= x 2))
(assert (or (< (+ x y) 1) (< 3 x)))
(get-proof)
\end{lstlisting}

\begin{center}
\lightning
\end{center}

\begin{lstlisting}[language=SMT,caption={The following example is the proof for the unsatisfiability of $(x+y < 1) \lor (3<x), x = 2$ and $0 = y$.},label={lst:smtexampleproof}]
(assume a0 (or (< (+ x y) 1) (< 3 x)))
(assume a1 (= x 2))
(assume a2 (= 0 y))
(step t1 (cl (< (+ x y) 1) (< 3 x)) :rule or :premises (a0))
(step t2 (cl (not (< 3 x)) (not (= x 2))) :rule la_generic :args (1/1 1/1))
(step t3 (cl (not (< 3 x))) :rule resolution :premises (a1 t2))
(step t4 (cl (< (+ x y) 1)) :rule resolution :premises (t1 t3))
(step t5 (cl (not (< (+ x y) 1)) (not (= x 2)) (not (= 0 y))) :rule la_generic :args (1/1 -1/1 1/1))
(step t6 (cl) :rule resolution :premises (t5 t4 a1 a2))
\end{lstlisting}

We will use the input problem shown in the top part of \cref{lst:smtexampleinput} with its Alethe proof (found by cvc5) in the bottom part as a running example to provide an overview of Alethe concepts and to illustrate our reconstruction of linear arithmetic step in Lambdapi.

\subsubsection{Alethe Trace Format Overview}
\label{sssect:alethe-trace-overview}

An Alethe proof trace inherits the declarations of its input problem. All symbols (sorts, functions, assertions, etc.) declared or defined in the input problem remain declared or defined, respectively. Furthermore, the syntax for terms, sorts, and annotations uses the syntactic rules defined in SMT-LIB \cite[\S 3]{smtlib} and the SMT signature context defined in \cite[\S 5.1 and \S 5.2]{smtlib}.
In the following we will represent an Alethe step as

\smallskip

\renewcommand{\eqnhighlightshade}{35}

\begin{equation}
\label{eq:step}
\tag{\textcolor{purple}{1}}
\eqnmarkbox[indexClr]{node2}{i}. \quad \eqnmarkbox[blue]{node1}{\Gamma} ~\triangleright~ \eqnmarkbox[green]{node3}{l_1 \dots l_n} \quad (\eqnmarkbox[purple]{node4}{\mathcal{R}}~\eqnmarkbox[red]{node5}{p_1 \dots p_m})~\eqnmarkbox[orange]{node6}{[a_1 \dots a_r]}
\end{equation}

\vspace{0.3em}
\annotate[yshift=-0.5em]{below, left}{node2}{index}
\annotate[yshift=-0.5em]{below, right}{node1}{context}
\annotate[yshift=0.5em]{above, left}{node3}{clause}
\annotate[yshift=-0.5em]{below, right}{node4}{rule}
\annotate[yshift=-0.5em]{below, right}{node5}{premises}
\annotate[yshift=-0.5em]{below, right}{node6}{arguments}

\vspace{0.3em}

\medskip

A step %\cref{eq:step} 
consists of an index \colorbox{indexClr!30}{$i$} $\in \mathbb{I}$ where $\mathbb{I}$ is a countable infinite set of indices (e.g. \kw{a0}, \kw{t1}), and a clause of formulae \colorbox{green!30}{$l_1, \dots, l_n$} representing an $n$-ary disjunction. Steps that are not assumptions are justified by a proof rule \colorbox{purple!30}{$\mathcal{R}$} that depends on a possibly empty set of premises $\{\colorbox{red!30}{$p_1 \dots  p_m$}\} \subseteq \mathbb{I}$ that only references earlier steps such that the proof forms
a directed acyclic graph. A rule might also depend on a list of arguments \colorbox{orange!30}{$[a_1 \dots a_r]$} where each argument $a_i$ is either a term or a pair $(x_i, t_i)$ where $x_i$ is a variable and $t_i$ is a term. The interpretation of the arguments is rule specific. The context \colorbox{blue!30}{$\Gamma$} of a step is a list $c_1 \dots c_l $ where each element $c_j$ is either a variable or a variable-term tuple denoted $x_j \mapsto t_j$. Therefore, steps with a non-empty context contain variables $x_j$ that appear in \colorbox{green!30}{$l_i$} and will be substituted by $t_j$. Proof rules \colorbox{purple!30}{$\mathcal{R}$} include theory lemmas and \texttt{resolution}, which corresponds to hyper-resolution on ground first-order clauses. 


\begin{table}[]
    \centering
    \begin{tabular}{ll}
    Rule & Description \\ \hline
    la\_generic & Tautologous disjunction of linear inequalities. \\
    lia\_generic & Tautologous disjunction of linear integer inequalities. \\
    la\_disequality & $t_1 \approx t_2 \lor \neg (t_1 \geq t_2) \lor \neg (t_2 \geq t_1)$ \\
    la\_totality & $t_1 \geq t_2 \lor t_2 \geq t_1$ \\
    la\_tautology & A trivial linear tautology \\
    la\_mult\_pos & $t_1 > 0 \land (t_2 \bowtie t_3) \rightarrow t_1 * t_2 \bowtie t_1 * t_3$ and $\bowtie \in \{<, >, \geq, \leq, =\}$ \\
    la\_mult\_neg & $t_1 < 0 \land (t_2 \bowtie t_3) \rightarrow t_1 * t_2 \bowtie_{inv} t_1 * t_3$ \\
    la\_rw\_eq & $(t \approx u) \approx (t \geq u \land u \geq t)$ \\
    comp\_simplify & Simplification of arithmetic comparisons. \\
    \end{tabular}
    \caption{Linear arithmetic rules in Alethe.}
    \label{table:linear-arith-rules}
\end{table}

We now have the key components to explain the guiding proof in the bottom part of \cref{lst:smtexampleproof}.
The proofs starts with \tt{assume} steps \tt{a0}, \tt{a1}, \tt{a2} that restate the assertions from the \textit{input problem} (\cref{lst:smtexampleproof}).
Step \tt{t1} transforms disjunction into clause by using the Alethe rule \tt{or}.
Steps \tt{t2} and \tt{t5} are tautologies introduced by the main rule \tt{la\_generic}
in Linear Real Arithmetic (LRA) logic and used also in LIA logic, where \colorbox{green!30}{$l_1, l_2,\dots, l_n$} represent linear inequalities.
These logics use closed linear formulas over the \lstinline[language=SMT,basicstyle=\ttfamily\footnotesize]{Real} signature and \lstinline[language=SMT,basicstyle=\ttfamily\footnotesize]{Int} respectively.
The \lstinline[language=SMT,basicstyle=\ttfamily\footnotesize]{Real} terms in \tt{LRA} logic are built over the Reals signature from SMT-LIB with free variables, but containing only linear atoms; that is
atoms of the form \lstinline[language=SMT,basicstyle=\ttfamily\footnotesize]{d}, \lstinline[language=SMT,basicstyle=\ttfamily\footnotesize]{(* d x)}, or \lstinline[language=SMT,basicstyle=\ttfamily\footnotesize]{(* x d)}  where \lstinline[language=SMT,basicstyle=\ttfamily\footnotesize]{x} is a free variable and  \lstinline[language=SMT,basicstyle=\ttfamily\footnotesize]{d} is an integer or rational constant.
Similarly, the \lstinline[language=SMT,basicstyle=\ttfamily\footnotesize]{Int} terms in \tt{LIA} logic are closed formulas built over the
Ints signature with free variables, but whose terms are also all linear, such that there is no occurrences of the function symbols \lstinline[language=SMT,basicstyle=\ttfamily\footnotesize]{*} (except variable multiplied by an \lstinline[language=SMT,basicstyle=\ttfamily\footnotesize]{Int} constant), \lstinline[language=SMT,basicstyle=\ttfamily\footnotesize]{/}, \lstinline[language=SMT,basicstyle=\ttfamily\footnotesize]{div}, \lstinline[language=SMT,basicstyle=\ttfamily\footnotesize]{mod}, and \lstinline[language=SMT,basicstyle=\ttfamily\footnotesize]{abs}.
A linear inequality is of term of the form

\begin{equation}
\sum_{i=0}^{n}c_i\times{}t_i + d_1\bowtie \sum_{i=n+1}^{m} c_i\times{}t_i + d_2
\label{eqn:inequality}
\end{equation}

where $\bowtie\;\in \{=, <, >, \leq, \geq\}$, where $m\geq n$, $c_i, d_1, d_2$ are either \lstinline[language=SMT,basicstyle=\ttfamily\footnotesize]{Int} or \lstinline[language=SMT,basicstyle=\ttfamily\footnotesize]{Real}
constants, and for each $i$ $c_i$ and $t_i$ have the same sort.
Checking the clause validity of \tt{t2} and \tt{t5} in \cref{lst:smtexampleproof}, amounts to checking the unsatisfiability of the system of linear equations (we provide more details in \cref{sssect:la-in-alethe}) e.g. $x < 3$ and $x = 2$ in \tt{t2}.
A coefficient for each inequality are pass as arguments e.g. $(\frac{1}{1},\frac{1}{1})$ in \tt{t2}.
Steps \tt{t3} (and \tt{t4}) applies the \colorbox{purple!30}{\texttt{resolution}} rule to the premises \tt{a1}, \tt{t2} (respectively \tt{t1} \tt{t3}).
Finally, the step \texttt{t6} concludes the proof by generating the empty clause $\bot$, concretely denoted as \kw{(cl)} in \cref{lst:smtexampleproof}.
Notice that the contexts \colorbox{blue!30}{$\Gamma$} of each step are all empty in this proof.

\subsubsection{Linear arithmetic in Alethe}
\label{sssect:la-in-alethe}

Proofs for linear arithmetic steps use a number of straightforward rules listed in \cref{table:linear-arith-rules}, such as \tt{la\_totality}: $(t_1 \leq t_2 \lor t_2 \geq t_1)$.
Simplification rules \tt{*\_simplify}, such as \tt{sum\_simplify}, transform arithmetic formulas by applying equivalence-preserving operations repeatedly until a fixed point is reached;
these operations are no more complex than constant folding.

Following our method to encode Alethe described in \cite{ColtellacciMD24}, the linear arithmetic tautology rules \tt{la\_disequality}, \tt{la\_totality} and \tt{la\_mult\_*} are encoded as lemmas in our embedding of Alethe in Lambdapi.
The simplification rule \tt{comp\_simplify} is encoded as a lemma for each rewrite case and applied multiple times.
We do not support the remaining \tt{*\_simplify} rules and the \tt{la\_tautology} rule in this work, primarily because cvc5 does not follow the Alethe standard for simplification step.
Instead, it extends the Alethe format with the RARE simplification rules \cite{rare}. As a result, cvc5 does not generate proofs using these standard rules for the SMT-LIB benchmarks.

A different approach is taken for the primary rules \tt{*\_generic},as they describe an algorithm.
While \tt{la\_generic rule} is primarily intended for LRA logic, it is also applied in LIA proofs when all variables in the (in)equalities are of integer sort.
A step of the rule \tt{la\_generic} represents a tautological clause of linear disequalities.  It can be checked by showing that the conjunction of
the negated disequalities is unsatisfiable. After the application of some strengthening rules, the resulting conjunction is unsatisfiable,
even if \lstinline[language=SMT,basicstyle=\ttfamily\footnotesize\upshape]{Int} variables are assumed to be \lstinline[language=SMT,basicstyle=\ttfamily\footnotesize\upshape]{Real} variables.
Although the rule may introduce rational coefficients, they often reduce to integers—as shown in \cref{lst:smtexampleproof}, where the coefficients are $(\frac{1}{1}, \frac{1}{1})$.
Cases where coefficients cannot be reduced to integers are rare in practice, however, we eliminate .
Let $\varphi_1,\dots, \varphi_n$ be linear inequalities and $a_1, \dots, a_n$ rational numbers, then a \tt{la\_generic} step has the general form

\[
\begin{matrix*}[c]
  i. & \ctxsep \quad & \varphi_1 , \dots , \varphi_n & \quad \tt{la\_generic}  & [a_1, \dots, a_n] \\
\end{matrix*}
\]

The constants $a_i$ are of sort \tt{Real}. To check the unsatisfiability of the negation of $\varphi_1, \dots, \varphi_n$ one performs the following steps for each literal. For each $i$, let $\varphi := \varphi_i$, $a := a_i$ and
we write $s1 \bowtie s2$ to denotes the left and right side of an inequality of \cref{eqn:inequality}.

\begin{enumerate}
    % \item If $\varphi = s_1 > s_2$, then let $\varphi := - s_1 \geq - s_2$.
    %   If $\varphi = s_1 \geq s_2$, then let $\varphi := - s_1 > - s_2$.
    %   If $\varphi = s_1 < s_2$, then let $\varphi := s_1 \geq s_2$.
    %   If $\varphi = s_1 \leq s_2$, then let $\varphi := s_1 > s_2$.
    \item If $\varphi =  \neg (s_1 < s_2)$  or $s_1 \geq s_2$, then let $\varphi := \neg(- s_1 \geq - s_2)$.
    If $\varphi =  \neg (s_1 \leq s_2)$ or $s_1 > s_2$, then let $\varphi := \neg(- s_1 > - s_2)$.
    If $\varphi = s_1 < s_2$, then let $\varphi := \neg(s_1 \geq s_2$).
    If $\varphi = s_1 \leq s_2$, then let $\varphi := \neg(s_1 > s_2$).
    This negates the literal. We want a canonical form that use only the operators $>, \geq$ and =.

    % \item If $\varphi = \neg (s_1 \bowtie s_2)$, then let $\varphi := s_1 \bowtie s_2$.
    
    % \item If $\varphi = s_1 < s_2$, then let $\varphi :=   - s_1 > - s_2$.
    %   If $\varphi = s_1 \leq s_2$, then let $\varphi :=  s_1 \geq - s_2$.
    %   We want a canonical form that use only the operators $>, \geq$ and =.

    \item Replace $\varphi = \sum_{i=0}^{n}c_i\times{}t_i + d_1 \bowtie \sum_{i=n+1}^{m} c_i\times{}t_i + d_2$  by $\sum_{i=0}^{n}c_i\times{}t_i - \sum_{i=n+1}^{m} c_i\times{}t_i
    \bowtie d_2 - d_1$.
    
    \item \label{la_generic:str}Now $\varphi$ has the form $s_1 \bowtie d$. If all
    variables in $s_1$ are integer sorted then replace $\bowtie d$ by $\bowtie \lceil d \rceil$,
    otherwise replace by $\bowtie \lfloor d\rfloor + 1$.

    % \begin{center}
    % \begin{tabular}{r|l|l}
    %     $\bowtie$  & If $d$ is an integer  & Otherwise \\
    %     \hline
    %     $>$        & $\geq d + 1$  & $\geq \lfloor d\rfloor + 1$  \\
    %     $\geq$     & $\geq d$      & $\geq \lfloor d\rfloor + 1$  \\
    % \end{tabular}
    % \end{center}

    \item If all variables of $\varphi$ are Int and coefficient $a_1 \dots a_n \in \mathbb{Q}$,
    then $a_i \coloneq a \times \mathit{lcd}(a_1 \dots a_n)$ where $\mathit{lcd}$ is the least common denominator of $[a_1 \dots a_n]$.
    
    \item If $\bowtie$ is $=$, then replace $\varphi$ by
    $\sum_{i=0}^{m}a\times{}c_i\times{}t_i = a\times{}d$, otherwise replace it by
    $\sum_{i=0}^{m}|a|\times{}c_i\times{}t_i \bowtie |a|\times{}d$.

    \item Finally, the sum of the resulting literals is trivially contradictory.
    The sum
    \[
        \sum_{k=1}^{n}\sum_{i=1}^{m}c_i^k*t_i^k \bowtie \sum_{k=1}^{n}d^k
    \]
  where $c_i^k$ and $t_i^k$ are the constant and term from the literal $\varphi_k$, and $d^k$ is the constant $d$ of $\varphi_k$.
  The operator $\bowtie$ is $=$ if all operators are $=$, $>$ if all are either $=$ or $>$, and $\geq$ otherwise. Finally, the sum on the left-hand side is $0$ and the right-hand side is $>0$ (or $\geq 0$ if $\bowtie$ is $>$).

\end{enumerate}

The step 1 has been added by our own, as the subsequent steps in the original algorithm are designed for $>$ and $\geq$ and do not clearly address how to handle $<$ and $\leq$.
Additionally, step 6  was added to ensure that our construction is independent of $\mathbb{Q}$.

\begin{example}
Consider the $\tt{la\_generic}$ step in the logic \tt{QF\_UFLIA} with the uninterpreted function symbol \lstinline[language=SMT,basicstyle=\ttfamily\footnotesize\upshape]|(f Int)|:
\begin{lstlisting}[language=SMT,label={lst:smtexampleinput}]
(step t11 (cl (not (<= f 0)) (<= (+ 1 (* 4 f)) 1))
  :rule la_generic :args (1/1 1/4))
\end{lstlisting} the algorithm run as follow in a natural deduction:
\begin{align}
&\vdash \neg (- f > 0),~ \neg(4f > 0) \label{eq:step2}\tag{Step 2}\\
&\vdash \neg (- f > 0),~ \neg(4f \geq 1) \label{eq:step3}\tag{Step 3}\\
&\text{Replace } a = [\frac{1}{1}, \frac{1}{4}] \text{ by } a = [4, 1] \label{eq:step4}\tag{Step 4}\\
&\vdash \neg (|4| * - f > |4| * 0 ), ~ \neg(|1| * 4f \geq |1| * 1) \label{eq:step5}\tag{Step 5} \\
&-4f + 4f \geq 1 \vdash \mathtt{False} \label{eq:step6}\tag{Step 6}
\end{align}
\todo[ac]{Je met $\vdash$ car je trouve que la dérivation ce comprend mieu de mon point de vue. Mais je peux l'enlever si trop de confusion est ajouté.}
Which sums to the contradiction  $0 \geq 1$. 
\end{example}

The \tt{lia\_generic} is structurally similar to \tt{la\_generic}, but omits the coefficients.
Since this rule can introduce a disjunction of arbitrary linear integer inequalities without any additional hints, proof checking is \emph{NP-complete} \cite{Schrijver:lia}.

\section{Elaborating lia\_generic steps}
\label{sec:elaboration-lia}

%\todo[sm]{Reviewer 1 asks us to improve the presentation.}
The rule \tt{lia\_generic} is similar to \tt{la\_generic}, but the SMT solver does not provide the coefficients,
i.e.\ \colorbox{orange!30}{$[a_1 \dots a_r]$} is empty.
%Since this rule can introduce a disjunction of arbitrary linear integer inequalities without any additional hints, proof checking is \emph{NP-complete} \cite{Schrijver:lia}.
We decided to leverage the elaboration process of \tt{lia\_generic} performed by Carcara, as doing otherwise would require implementing Fourier-Motzkin elimination for integers, as done in \cite{micromega,omegatest}, hence reimplementing work that was already done by the solver.

Carcara considers $\tt{lia\_generic}$ steps as holes in the proof, given that ``their checking is as hard as solving'' \cite[\S 3.2]{carcara}.
To address this, Carcara invokes an external SMT solver, such as cvc5, or any tool capable of reading SMT-LIB input and producing Alethe proofs, and attempts to generate an Alethe proof that avoid using $\tt{lia\_generic}$. 
Suppose the resulting proof still includes a $\tt{lia\_generic}$ step. In that case, Carcara repeats the process for up to three iterations, merging the final results if a complete proof without any $\tt{lia\_generic}$ steps is eventually found.
The proof is then imported and validated, replacing the original step.

\begin{example}[Sketch of lia\_generic elaboration]
%
Consider a step $S$ (List.~\ref{lst:pb_lia}) concluding the clause $\neg l_1 \lor \dots \neg l_n$ where all $l_i$ are inequalities and proved by $\tt{lia\_generic}$ rule.
    
\begin{lstlisting}[language=SMT,caption={Elaborated proof},label={lst:pb_lia}]
    (step S (cl (not l1) ... (not ln)) :rule lia_generic)
\end{lstlisting}
%
Carcara will generate an SMT-LIB problem asserting $l_1$, \dots, $l_n$ and invoke the solver cvc5 on it, expecting an Alethe proof of the unsatisfiability of $l_1 \land \dots \land l_n$
that does not use \tt{lia\_generic}. Carcara will check this subproof and then replace the original step by a proof of the form shown in List.~\ref{lst:elab_lia}.

\begin{lstlisting}[language=SMT,caption={Elaboration of \tt{lia\_generic}},label={lst:elab_lia}]
(anchor :step S.t_m+1)
(assume S.h_1 l1)
...
(assume S.h_n ln)
...
(step S.t_m (cl false) :rule ...)
(step t.t_m+1 (cl (not l1) ... (not ln) false) :rule subproof)
(step t.t_m+2 (cl (not false)) :rule false)
(step S (cl (not l1) ... (not ln)) 
        :rule resolution :premises (S.t_m+1 S.t_m+2))
\end{lstlisting}

In List.~\ref{lst:elab_lia}, steps $\tt{S.h\_1}$ until $\tt{S.t\_m}$ are imported from the cvc5 proof.
As a result the $\tt{lia\_generic}$ step $\tt{S}$ in the original proof (List.~\ref{lst:pb_lia}) will have been replaced by a detailed justification whose correctness can be independently established by Carcara.

\end{example}
% In the next section, we first present an overview of our embedding of Alethe in Lambdapi, and then our automatic procedure to reconstruct $\tt{la\_generic}$ step that appear in LIA problem.


\section{Reconstruction of \tt{la\_generic} step for LIA logic}
\label{sect:recon-lambdapi}

\subsection{An Overview of Lambdapi}
\label{ssect:lambdapi-overview}

Lambdapi is an implementation of $\lambda\Pi$ modulo theory ($\lpm$) \cite{lambdapi}, an extension of the Edinburgh Logical Framework $\lambda\Pi$ \cite{lf}, a simply typed $\lambda$-calculus with dependent types. $\lpm$ adds user-defined higher-order rewrite rules. Its syntax is given by
%
\begin{align*}
&\text{Universes}  &u &::= \type ~|~ \kind \\
&\text{Terms}   &t,v, A,B,C &::= c ~|~ x ~|~ u ~|~ \Pi\,x : A,\,B~|~ \lambda\,x : A,\,t ~|~t~v \\
&\text{Contexts}   &\Gamma &::= \langle \rangle ~|~ \Gamma, x : A \\
&\text{Signatures}  &\Sigma &::= \langle \rangle ~|~ \Sigma, c : C ~|~ \Sigma, c := t : C ~|~ \Sigma, t \hookrightarrow v 
\end{align*}
%
where $c$ is a constant and $x$ is a variable  (ranging over disjoint sets), $C$ is a closed term. \emph{Universes} are constants used to verify if a type is well-formed -- more details can be found in \cite[\S 2.1]{lf}. $\Pi\,x : A.\,B$ is the dependent product, and we write $A \rightarrow B$ when $x$ does not appear free in $B$, $\lambda\,x : A.\,t$ is an abstraction, and  $t~v$ is an application. A \emph{(local) context} $\Gamma$ is a finite sequence of variable declarations $x:A$ introducing variables and their types.
A \emph{signature} $\Sigma$ representing the global context is a finite sequence of \emph{assumptions} $c : C$, indicating that constant $c$ is of type $C$, \emph{definitions} $c := t : C$, indicating that $c$ has the value $t$ and type $C$, and \emph{rewrite rules} $t \hookrightarrow v$ such that $t = c~v_1 \dots v_n$ where $c$ is a constant.

The relation $\hookrightarrow_{\beta\Sigma}$ is generated by $\beta$-reduction and by the rewrite rules of $\Sigma$. The relation $\hookrightarrow_{\beta\Sigma}^*$ denotes the reflexive and transitive closure of $\hookrightarrow_{\beta\Sigma}$, and the relation $\equiv_{\beta\Sigma}$ (called \emph{conversion}) the reflexive, symmetric, and transitive closure of $\hookrightarrow_{\beta\Sigma}$. 
The relation $\hookrightarrow_{\beta\Sigma}$ must be confluent, i.e.,
whenever $t \hookrightarrow_{\beta\Sigma}^* v_1$ and $t \hookrightarrow_{\beta\Sigma}^* v_2$, there exists a term $w$ such that $v_1 \hookrightarrow_{\beta\Sigma}^* w$ and $v_2 \hookrightarrow_{\beta\Sigma}^* w$, and it must preserve typing, i.e., 
whenever $\Gamma \vdash_\Sigma t: A$ and $t \hookrightarrow_{\beta\Sigma} v$ then $\Gamma \vdash_\Sigma v: A$ \cite{blanqui:LIPIcs.FSCD.2020.13}.

A Lambdapi typing judgment $\Gamma \vdash_\Sigma t : A$ asserts that term $t$ has type $A$ in the context $\Gamma$ and the signature $\Sigma$.
The typing rules of $\lpm$ are the one of  $\lambda\Pi$ \cite[\S 2]{lf}, except for the rule (Conv) where it use the version of \cref{fig:lp-typing-rules} that identifies types modulo~$\textcolor{orange}{\equiv_{\beta\Sigma}}$ instead of just modulo $\beta$-reduction. 

\begin{figure}
    \begin{center}
    \begin{prooftree}
    \hypo{\Gamma, \vdash_\Sigma B: u}
    \hypo{\Gamma \vdash_\Sigma t: A}
    \hypo{\textcolor{orange}{A \equiv_{\beta\Sigma} B}}
    \infer3[(Conv)]{ \Gamma \vdash_\Sigma t: B }
    \end{prooftree}
    \end{center}
    \caption{(Conv) rule in $\lpm$}
    \label{fig:lp-typing-rules}
  \end{figure}

We now provide an overview of the encoding of Alethe linear integers arithmetic in Lambdapi.

\subsection{Encoding of Integers in Lambdapi}


\begin{figure}
\centering
\begin{align*}\label{eq:eq1}
&\bb{Z}: \type & &\bb{P}: \type  & &\tt{Comp}: \type & &\bb{B}: \type \\
&|~\tt{Z0}: \bb{Z} & &|~\tt{H} : \bb{P} & &|~\tt{Eq}: \tt{Comp} & &|~\tt{true}: \bb{B} \\
&|~\tt{ZPos}: \bb{P} \ra \bb{Z} & &|~\tt{O}: \bb{P} \ra \bb{P} & &|~\tt{Lt}: \tt{Comp} & &|~\tt{false}: \bb{B} \\
&|~\tt{ZNeg}: \bb{P} \ra \bb{Z} & &|~\tt{I}: \bb{P} \ra \bb{P} & &|~\tt{Gt}: \tt{Comp} & &\\
&\tt{int}: \set & &\tt{pos}: \set & &\tt{comp}: \set & &\tt{bool}: \set \\
&\el~\tt{int} \re \bb{Z} & &\el~\tt{pos} \re \bb{P} & &\el~\tt{comp} \re \tt{Comp} & &\el~\tt{comp} \re \bb{B}
\end{align*}
\caption{Overview of sorts, constructors, constants, and element relations}
\label{fig:sorts-constructors}
\end{figure}

The definition we use of integers in Lambdapi in \cref{fig:sorts-constructors} follows a common encoding found in many other theories, including the one adopted in the Rocq standard library \cite{Rocq-refman}.
First, the type $\bb{P}$  is an inductive type representing strictly positive integers in binary form.
Starting from 1 (represented by constructor \tt{H}), one can add a new least significant digit via the constructor \tt{O} (digit 0) or constructor \tt{I} (digit 1). 
The type $\bb{Z}$ is an inductive type representing integers in binary form.
An integer is either zero (with constructor \tt{Z0}) or a strictly positive number \tt{Zpos} (coded as a $\bb{P}$) or a strictly negative number \tt{Zneg} (whose opposite is stored as a $\bb{P}$ value).
%
As discussed in our previous work \cite{ColtellacciMD24}, $\lpm$ does not support quantifiy over a variable of type $\type$. More precisely, it is not possible to assign the type $\Pi X : \type,(X \ra \prop) \ra \prop$ to the universal quantifier $\forall$, where $\prop: \type$ is the type of proposition.
To address this, we introduce a constant $\set : \type$ for the types of object-terms, and a constant ${\el}$ to embed the terms of type $\set$ into terms of type $\type$ giving us the quantifier $\forall: \Pi x: \set, (\el~x \ra \prop) \ra \prop$.
%
To enable quantification over types such as integers, positive binary numbers, booleans, and comparison results, we introduce a constant of type $\set$ (e.g. $\tt{int}: \set$) that represents codes for these types — similar to the Tarski-style universe \cite[\S Universes]{intuitype},
where types are represented by elements of a base type and interpreted via the decoding function. In our setting, the decoding function $\el$  is realized through a rewriting rule that reduces the term to its corresponding type; for example, $\el~\tt{int} \re \bb{Z}$.
The comparison datatype $\tt{Comp}$ is utilized to define the decidable equality $\doteq$ between the $\bb{Z}$ and the function \tt{cmp} for $\bb{P}$ (as defined in \cref{app:lambdapi-func-def}).

\begin{figure}
\centering
\begin{minipage}[t]{0.48\textwidth}
\begin{align*}
&+: \bb{Z} \ra \bb{Z} \ra \bb{Z} \\
& \tt{Z0} + y \re y \\
& x + \tt{Z0} \re \tt{Z0} \\
& (\tt{Zpos x}) + (\tt{Zpos y}) \re (\tt{Zpos}~(\tt{add}~x~y))  \\
& (\tt{Zpos x}) + (\tt{Zneg y}) \re (\tt{sub}~x~y)  \\
& (\tt{Zneg x}) + (\tt{Zpos y}) \re (\tt{sub}~y~x)  \\
& (\tt{Zneg x}) + (\tt{Zneg y}) \re \tt{Zpos}(\tt{add}~x~y)  \\
\end{align*}
\hfill
\end{minipage}
\begin{minipage}[t]{0.48\textwidth}
\begin{align*}
&\doteq : \bb{Z} \ra \bb{Z} \ra \tt{Comp} \\
& \tt{Z0} \doteq \tt{Z0} \re \tt{Eq} \\
& \tt{Z0} \doteq \tt{Zpos}~\_ \re \tt{Lt} \\
& \tt{Z0} \doteq \tt{Zneg}~\_ \re \tt{Gt} \\
& \tt{Zpos}~\_ \doteq \tt{Z0} \re \tt{Gt} \\
& \tt{Zpos}~p \doteq \tt{Zpos}~q \re \tt{cmp}~p~q \\
& \tt{Zpos}~\_ \doteq \tt{Zneg}~\_ \re \tt{Gt} \\
& \tt{Zneg}~\_ \doteq \tt{Z0} \re \tt{Lt} \\
& \tt{Zneg}~\_ \doteq \tt{Zpos}~\_ \re \tt{Lt} \\
& \tt{Zneg}~p \doteq \tt{Zneg}~q \re \tt{cmp}~q~p \\
\end{align*}
\end{minipage}
\noindent
\[
\begin{array}{l@{\hspace{4em}}l@{\hspace{4em}}l}
\begin{aligned}
  &\tt{isEq} : \tt{Comp} \ra \bb{B} \\
  &\tt{isEq}~\tt{Eq} \re \tt{true} \\
  &\tt{isEq}~\tt{Lt} \re \tt{false} \\
  &\tt{isEq}~\tt{Gt} \re \tt{false} \\
\end{aligned}
&
\begin{aligned}
  &\tt{isLt} : \tt{Comp} \ra \bb{B} \\
  &\tt{isLt}~\tt{Eq} \re \tt{false} \\
  &\tt{isLt}~\tt{Lt} \re \tt{true} \\
  &\tt{isLt}~\tt{Gt} \re \tt{false} \\
\end{aligned}
&
\begin{aligned}
  &\tt{isGt} : \tt{Comp} \ra \bb{B} \\
  &\tt{isGt}~\tt{Eq} \re \tt{false} \\
  &\tt{isGt}~\tt{Lt} \re \tt{false} \\
  &\tt{isGt}~\tt{Gt} \re \tt{true} \\
\end{aligned}
\end{array}
\]
\noindent
\begin{align*}
&\leq: \bb{Z} \ra \bb{Z} \ra \prop  \coloneq \lambda x,\lambda y, \neg (\tt{istrue}(\tt{isGt}(x \doteq y))) & &\tt{istrue} : \bb{B} \ra \prop \\
&<: \bb{Z} \ra \bb{Z} \ra \prop  \coloneq \lambda x,\lambda y, (\tt{istrue}(\tt{isLt}(x \doteq y))) & &\tt{istrue}~\tt{true} \re \top \\
&\geq: \bb{Z} \ra \bb{Z} \ra \prop  \coloneq \lambda x,\lambda y, \neg (x < y) & &\tt{istrue}~\tt{false} \re \bot \\
&>: \bb{Z} \ra \bb{Z} \ra \prop  \coloneq \lambda x,\lambda y, \neg (x \leq y) & & \\
\end{align*}

\caption{Decidable equality, $+$ operator and inequalities relations definitions for $\bb{Z}$}
\label{fig:arith-ops}
\end{figure}

The arithmetic operator such as \tt{add}, \tt{sub}, and others, as presented in \cref{fig:arith-ops} are constants defined by rewriting rules. In the following sections, we will refers 
to the rewriting rules for integers as $\ra_\bb{Z}$ and positive binary numbers as $\ra_\bb{P}$.
The confluence of the rewriting rules for the arithmetic of $\mathbb{Z}$ and $\mathbb{P}$ has been proven using CSI \cite{CSI}. A detailed proof of confluence can be found in \cref{app:confluence-int-pos}.
The inequality symbols for $\bb{Z}$ are binary predicates defined by rewriting rules over the decidable equality $\doteq$. They reduce to $\top$, $\bot$ (or negated) by $\equivL$ with rules of $\ra_\bb{Z}$ and $\ra_\bb{P}$.
For example, $1 < 2 \hookrightarrow \tt{istrue}(\tt{isLt}(1 \doteq 2)) \hookrightarrow \tt{istrue}(\tt{isLt}(\tt{Lt})) \hookrightarrow \tt{istrue}(\tt{true}) \hookrightarrow \top$.

\subsection{Functions used in the translation}

We now provide an overview of how input problems expressed in a given SMT-LIB signature \cite[\S 5.2.1]{smtlib} are encoded.
A comprehensive description of the encoding can be found in \cite{ColtellacciMD24}, we will focus here on the arithmetic.
In order to avoid a notational clash with the Lambdapi signature $\Sigma$, we denote the set of SMT-LIB sorts as $\Theta^\mathcal{S}$, the set of function symbols $\Theta^\cal{F}$, and the set of variables $\Theta^\cal{X}$.
Our translation is based on the following functions:

\begin{itemize}
\item $\cal{D}$ translates declarations of sorts and functions in $\Theta^\cal{S}$ and $\Theta^\mathcal{F}$ into constants,
\item $\cal{S}$ maps sorts to $\Sigma$ types,
\item $\cal{E}$ translates SMT expression to $\lpm$ terms,
\item $\cal{C}$ translates a list of commands  $c_1 \dots c_n$ of the form $i.~\Gamma \triangleright~\varphi~(\mathcal{R}~P)[A]$ to typing judgments $\Gamma \vdash_\Sigma i := M: N$.
\end{itemize}

\begin{definition}[Function $\mathcal{D}$ translating SMT sort and function symbol declarations]
For each sort symbol $s$ with arity $n$ in $\Theta^\cal{S}$ we create a constant $s: \set \ra \dots \ra \set$.
For each function symbol $f~\sigma^+$ in $\Theta^\cal{F}$ we create a constant $f: \cal{S}(\sigma^+)$.
\end{definition}

\begin{definition}[Function $\mathcal{S}$ translating sorts of expression] 
  The definition of $\mathcal{S}$(s) is as follows.
  \begin{itemize}
    \item Case $s = \textbf{Bool}$, then $\Sort{s} = \el\,o$,
    \item Case $s = \textbf{Int}$, then $\Sort{s} = \el~\texttt{int}$,
    \item Case $s = \sigma_1\,\sigma_2 \dots \sigma_n$ then $\Sort{s} = \el{} (\mathcal{S}(\sigma_1) \leadsto \dots \leadsto \mathcal{S}(\sigma_n))$,
    \item otherwise $\Sort{s} = \el\, \mathcal{D}(s)$.
  \end{itemize}
  with the constant $o: \set$ and $\el\, o \re \prop$ to quantify over propositions.
\end{definition}

\begin{definition}[Function $\mathcal{E}$ translating SMT expressions]
The definition of $\E{e}$ is as follows.
\begin{itemize}
\setlength{\parskip}{0pt}
\item Case e $= (p~t_1~t_2\dots~t_n)$ and $p$ a logical connector,\\
  then $\E{e} = \E{t_1}~p^c~\dots~p^c~\E{t_n}$.
\item Case e $= (+~t_1\dots~t_n)$, then $\E{e} = \E{t_1} + ~\dots~ +\E{t_n}$.
\item Case e $= (*~t_1\dots~t_n)$, then $\E{e} = \E{t_1} * ~\dots~ *\E{t_n}$.
\item Case e $= (-t)$, then $\E{e} = \sim \E{t_1}$.
\item Case e $= (-~t_1\dots~t_n)$, then $\E{e} = \E{t_1} - ~\dots~ - \E{t_n}$.
\item Case e $= (\approx~t_1~t_2)$ then $\E{e} = (\E{t_1} = \E{t_2})$.
\item Case e $= (Q~x_1 : \sigma_1  \dots x_n : \sigma_n ~t)$ where $Q\in \{\kw{forall}, \kw{exists}\}$, then $\E{e} = Q^c x_1: \cal{S}(\sigma_1), \dots, Q^c x_n: \cal{S}(\sigma_n), \E{t}$. 
\item Case $e = (x: \sigma )$ with $x \in \Theta^\mathcal{X}$ a sorted variable, then $\E{e} = x: \cal{S}(\sigma)$.
\end{itemize}
\end{definition}

\begin{definition}
We define the function $\cal{C}$ that translates an Alethe step of the form $i.~\Gamma~\triangleright~l_1 \dots l_n\,(R\,P)[A]$ into a constant
$i: \pid (\E{l_1}\cons\dots\cons\E{l_n}\cons \nil) \coloneq M$ where $M$ is a proof term of appropriate type.
The function $\cal{C}$ is defined by cases on the rule $R$.
\end{definition}

We introduce an embedding $\pid: \tt{Clause} \ra \type$ of clause into types, mapping each clause $C$ to the type $\pid~C$ of its proofs.
Similarly, the constant $\pic: \prop \ra \type$ maps each proposition A. The type $\tt{Clause}: \type$ represent the type of clause encoded as list \cite[\S 3]{ColtellacciMD24}
with the constructor $\veedot: \prop \ra \tt{Clause} \ra \tt{Clause}$ and the empty clause $\nil: \tt{clause}$.


\begin{example}{Translation of the proof goal of the prelude, step t2 and t5 in \cref{lst:smtexampleproof}}
\begin{lstlisting}[language=Lambdapi,mathescape=true]
symbol x: El int;
symbol y: El int;
...
opaque symbol t2: $\pid$ (¬ (3 < x)  ⟇ ¬ (x = 2) ⟇ ▩) ≔ begin ... end;
opaque symbol t5: $\pid$ (¬ (x + y < 1) ⟇ (¬ (x = 2))  ⟇ (¬ (0 = y)) ⟇ ▩) ≔ begin ... end;
...
\end{lstlisting}
\end{example}

The proof terms generated by $\mathcal{C}$ for steps \texttt{t2} and \texttt{t5} must faithfully represent the algorithm presented in \cref{ssect:la-in-alethe}.
While steps 1 through 7 of the algorithm correspond to explicit rewriting steps, the final step (step 8) — which involves summing all inequalities represents a multi-step rewriting sequence.
This sequence reduced the initial sum to a decidable comparison between constants (e.g. $0 > 1 \re \bot$), which serves as the conclusion of the reduction.
%
The reflection technique introduced by \cite{reflection-origin-coq} leverages the reduction system of the proof assistant to produce an efficient decidable automatic
procedure for comparing \emph{Ring} terms. We decided to follow this approach to implement our decision procedure for evaluating inequality.
In the following section, we describe how we implemented this procedure for our case, and how we extended the definition of $\mathcal{C}$.

\section{Reconstruction of linear arithmetic for LIA logic}
\label{sec:lia-reconstruction}

Proof by reflection is a technique used to write certified automation procedure by reducing the validity of a logical statement to a symbolic computation.
It relies on the following definitions: let $P: Z \ra \prop$ be a predicate over a data type Z and we have a function $f: Z \ra \tt{bool}$ such that the following theorem holds:

\begin{equation*}
\tt{f\_correct} : \forall z: Z, (f~z = \tt{true}) \ra (P~z)
\end{equation*}

If $\mathop{f} z$ reduces to \tt{true}, then the proof term  $\tt{f\_correct}~z~(\tt{relf}\,\tt{bool}\,\tt{true})$ with $\tt{refl}: \Pi A: \set, \Pi x: \el\,A, \pic (x = x)$, constitutes a proof of predicate $(P~z)$. In step 6 of the $\tt{la\_generic}$,
the primary challenge lies in reasoning modulo associativity and commutativity when manipulating expressions over $\bb{Z}$.
The key idea is to provide a normalization function that transforms a $\bb{Z}$ expression into a canonical form,
such that it can be reduce to a constant because variables will cancel each other, as is the case with $f$ in \cref{ex:la_generic_example_red}.


\subsection{Representation}
\label{ssec:representation}

The procedure is based on a group structure, denoted as  $\bb{G}$ defined in \cref{fig:grp} which represents linear polynomials.
The base type for the elements of this group is specified as $\bb{G}: \type$. The unary operator $\tt{cst}$ denotes a constant from $\bb{Z}$.
A $\tt{var}$ constructor for "catch-all" case for subexpressions that we cannot model. These subexpressions will correspond to actual variables in $\bb{Z}$.
The constructor $\tt{mul}$  represents the multiplication of an element of $\bb{G}$ by a constant. The constructor $\tt{opp}$ corresponds to the inverse operator within the group.
Lastly, the constructor $\tt{add}$ represent the addition between two elements of \cref{fig:grp}.

To support associative and commutative operations, Lambdapi provides the modifiers \lstinline[language=Lambdapi,basicstyle=\ttfamily\footnotesize\upshape]{associative commutative symbol},
ensuring that terms are systematically placed into a canonical form given a builtin ordering relation, as described in \cite{ACorigin} and \cite[\S 5]{univAC}.
Applying the \lstinline[language=Lambdapi,basicstyle=\ttfamily\footnotesize\upshape]{associative commutative} modifiers to the infix constructor $\add{}{}$,
ensures that expressions involving sums of products are systematically canonicalized. Thus, equal variables are placed next to each other, facilitating simplification.

\begin{figure}
\begin{align*}
& \bb{G}: \type & & \reify{} : \bb{Z} \ra \bb{G} & & \den{}: \bb{G} \ra \bb{Z} \\
&|~\add{}{}: \bb{G} \ra \bb{G} \ra \bb{G} & & \reify{\tt{Z0}} \re \cst{\tt{Z0}} & & \den{\cst{c}} \re c \\
&|~\tt{var}: \bb{Z} \ra \bb{Z} \ra \bb{G} & & \mathop{\reify{\tt{ZPos}}} c \re \cst{c} & & \den{\opp{x}} \re  \sim (\den{x}) \\
&|~\tt{mul}: \bb{Z} \ra \bb{G} \ra \bb{G} & & \mathop{\reify{\tt{ZNeg}}} c \re \cst{c} & & \den{\mul{c}{x}} \re  c \times (\den{x}) \\
&|~\tt{opp}: \bb{G} \ra \bb{G} & & \reify{(x + y)} \re \add{(\reify{x})}{(\reify{y})} & & \den{\add{x}{y}} \re (\den{x}) + (\den{y}) \\
&|~\tt{cst}: \bb{Z} \ra \bb{G} & & \reify{(\sim x)} \re \opp{\reify{x}} & & \den{\var{c}{x}} \re  c \times x \\
&\tt{grp}: \set & & \mathop{\reify{(\mathop{\tt{Zpos}} c) * x}} \re \mul{c}{(\reify{x})}  & & \\
&\el~\tt{grp} \re \bb{G} & & \mathop{\reify{(\mathop{\tt{ZNeg}} c) * x}} \re \mul{c}{(\reify{x})} & & \\
& & & \mathop{\reify{x * (\mathop{\tt{Zpos}} c)}} \re \mul{c}{(\reify{x})}  & & \\
& & & \mathop{\reify{x * (\mathop{\tt{ZNeg}} c)}} \re \mul{c}{(\reify{x})} & & \\
& & & \mathop{\reify{x}} \re \var{1}{x} & &
\end{align*}
\caption{Definition of $\bb{G}$  Algebra and its reification ($\reify{}$) and denotation ($\den{}$) functions}
\label{fig:grp}
\end{figure}

\subsection{Normalization}
\label{ssec:normalization}

\begin{figure}
\begin{align*}
&\add{\var{x}{c_1}}{\var{x}{c2}} \re \kw{var}~x~(c_1 + c_2) \\
&\add{\var{x}{c_1}}{(\add{\var{x}{c_2}}{y})} \re \add{\var{x}{(c_1 + c_2)}}{y} \\
&\add{\cst{c_1}}{\cst{c_2}} \re \cst{c_1 + c_2} \\
&\add{\cst{c_1}}{(\add{\cst{c_2}}{y})} \re \add{\cst{c_1 + c_2}}{y} \\
&\add{\cst{0}}{x} \re x \\
&\add{x}{\cst{0}} \re x \\
&\opp{\var{x}{c}} \re \var{x}{(-c)} \\
&\opp{\cst{c}} \re \cst{(-c)} \\
&\opp{\opp{x}} \re x \\
&\opp{\add{x}{y}} \re \add{(\opp{x})}{(\opp{y})} \\
&\mul{k}{\var{x}{c}} \re \var{x}{(k * c)} \\
&\mul{k}{\opp{x}} \re \mul{(-k)}{x} \\
&\mul{k}{(\add{x}{y})} \re \add{(\mul{k}{x})}{(\mul{k}{y})} \\
&\mul{k}{\cst{c}} \re \cst{(k * c)} \\
&\mul{c_1}{(\mul{c_2}{x})} \re \mul{(c_1 * c_2)}{x} \\
\end{align*}
\caption{Rewrite system on canonical forms}
\label{fig:grp-rw}
\end{figure}

% \begin{example}
% \begin{align}
% &\reify{(x + y + (- x) + (- y))} \\
% &\re \var{1}{x} + \var{1}{y} + \tt{opp}(\var{1}{x}) + \tt{opp}(\var{1}{y})  \\
% &\re \var{1}{x} + \var{1}{y} + \var{-1}{x} + \var{-1}{y} \\
% &\simeq_{AC} \var{1}{x} + \var{-1}{x} + \var{1}{y} + \var{-1}{y} \\
% &\re \cst{0}
% \end{align}
% \end{example}

For \lstinline[language=Lambdapi,basicstyle=\ttfamily\footnotesize\upshape]{associative commutative} symbols, Lambdapi does not use matching modulo AC \cite{matching-mod-AC,kirchner_rsp} as this problem is NP-complete \cite{ac-modulo-np-complete}.
Instead, Lambdapi transforms sum expression of the form $\add{\var{c_1}{a_1}}{\add{\var{c_2}{a_2}}{\add{\dots}{\var{c_n}{a_n}}}}$ into a canonical form, ensuring that any pair of terms $\var{p}{x}$ and $\var{q}{x}$ involving the same variable $x$ are placed adjacent to each other.
In this way, we will use the rewriting rules $\cref{fig:grp-rw}$ to reduce expressions in canonical form. We use the notation $\re_{\bb{G}}$ to refers to the rule of \cref{fig:grp-rw}.

\begin{definition}[AC-canonical form]
Let $\leq$ be a total order on $\bb{G}$-terms defined as follows:
Constants are ordered such that $\tt{cst}(c_1) \leq \tt{cst}(c_2) < \var{p}{x}$ for any constants $c_1, c_2$ and any variable term $\var{p}{x}$, with $c_1 \leq c_2$.
For variable terms, $\var{p}{x} \leq \var{q}{y}$ if either $x < y$, or $x = y$ and $p \leq q$.

Let $\twoheadrightarrow^{AC}$ be the relation mapping every term t to its unique AC-canonical form denoted $[t]$.
\end{definition}

 Two terms are AC-equivalent iff their AC-canonical forms are equal. Two terms $t, u$ are equivalent, written $t \simeq u$, if for all valuation $\mu: \cal{V} \ra \bb{Z}$ where $\cal{V}$ a set of variables of type $\bb{Z}$, we have $t\mu = u \mu$.
 In the following, we will denote $\simeq_{AC}$ the equational sub-theory of $\simeq$ generated by the equations $\add{x}{y} = \add{y}{x}$ and $\add{x}{(\add{y}{z})} = \add{(\add{x}{y})}{z}$.

\begin{definition}[Rewriting modulo AC-canonization]
Let $\longrightarrow^{AC}_\Sigma = \re_\Sigma\twoheadrightarrow^{AC}$, where $\Sigma$ contains the rewrite rules of \cref{fig:arith-ops,fig:grp-rw}. 
 \end{definition}

An $\longrightarrow^{AC}_\Sigma$ step is a standard $\re_\Sigma$ step with syntactic matching followed by ACcanonization. We now prove that the relation $\longrightarrow^{AC}_\Sigma$ terminates and is confluent.

\begin{lemma} The matching modulo AC relation $\longrightarrow_{\Sigma/AC} = \mathop{\simeq_{AC}}\re_\Sigma\mathop{\simeq_{AC}}$, which contains $\longrightarrow^{AC}_\Sigma$, terminates.
\begin{proof} \upshape{AProVE \cite{aprove}} automatically proves the termination of $\longrightarrow_{\Sigma/AC}$. \end{proof}
\end{lemma}

\begin{lemma} $\longrightarrow^{AC}_\Sigma$ is locally confluent on AC-canonical.
\begin{proof}
We show that every critical pair is joinable using $\longrightarrow^{AC}_\Sigma$ and confluence of $\ra_\bb{Z}$ and $\ra_\bb{P}$.
In the following, the terms that are not between square brackets are in AC-canonical form. We also write $[\add{p}{q}]$ to denote either $\add{p}{q}$ or $\add{q}{p}$.
\end{proof}
\end{lemma}

Hence, every $\bb{Z}$-terms has, after being reifed into an $\bb{G}-term$, a unique normal form with respect to $\longrightarrow^{AC}_\Sigma$.
We now prove that this normal form is almost a canonical form, and that it is sufficient to decide $\simeq$.

\begin{lemma} For all $\bb{Z}$-terms t and u, we have $t \simeq u$ iff $[|$t$|]$ and $[|$u$|]$ have the same normal form wrt $\longrightarrow^{AC}_\Sigma$,
where $[|$t$|]$ is the AC-canonical form of the reification of t in $\bb{G}$.
\end{lemma}


\begin{lemma}[Conversion]
For all $x: \bb{Z}$, we have $x = (\den{\reify{x}}$).
\label{lem:conv}
\end{lemma}

We will use the \cref{lem:conv} to convert.

% https://q.uiver.app/#q=WzAsOCxbMSwyLCJcXG1hdGhiYntafSJdLFsxLDAsIlxcbWF0aGNhbHtSfSJdLFszLDAsIlxcbWF0aGNhbHtSfSJdLFszLDIsIlxcbWF0aGJie1p9Il0sWzAsMiwidF8xID1fXFxtYXRoYmJ7Wn0gdF8yIl0sWzQsMiwiZGVub3RlKHRfMSkgPV9cXG1hdGhiYntafSBkZW5vdGUodF8yKSJdLFswLDAsInJlaWZ5KHRfMSkgPV9cXG1hdGhjYWx7Un0gcmVpZnkodF8yKSJdLFs0LDAsIm5vcm0odF8xKSA9X1xcbWF0aGNhbHtSfSBub3JtKHRfMikiXSxbMCwxLCJyZWlmeSJdLFsxLDIsIlxccmlnaHRhcnJvd197QUN9IiwwLHsic3R5bGUiOnsiYm9keSI6eyJuYW1lIjoiZG90dGVkIn19fV0sWzIsMywiZGVub3RlIl0sWzAsMywiXFxlcXVpdiIsMSx7InN0eWxlIjp7ImJvZHkiOnsibmFtZSI6ImRvdHRlZCJ9LCJoZWFkIjp7Im5hbWUiOiJub25lIn19fV1d
\begin{tikzcd}[ampersand replacement=\&,column sep=small]
	{\reify{(t_1)} \mathrel{\bowtie} \reify{(t_2)}} \& {\bb{G}} \&\& {\bb{G}} \& {{[t_1]} \bowtie [t_2]} \\
	\\
	{t_1 =_\mathbb{Z} t_2} \& {\mathbb{Z}} \&\& {\mathbb{Z}} \& {\den{({[t_1]})} =_\mathbb{Z} \den{({[t_2]})}}
	\arrow["AC", dotted, from=1-2, to=1-4]
	\arrow["denote", from=1-4, to=3-4]
	\arrow["reify", from=3-2, to=1-2]
	\arrow["\iff"{description}, dotted, no head, from=3-2, to=3-4]
\end{tikzcd}


\subsection{Generation of the proof term for \tt{la\_generic} in \tt{LIA}}
\label{ssec:gen-lia-proof}

\section{Evaluation}
\label{sec:evaluation}

\begin{table}[]
\centering
\begin{tabular}{|c|c|c|c|c|c|c|}
\hline                                                             % Succ - Err -Timeout   % Succ - Error        % Succ - Err - Timeout % Succ - Err - Timeout
\textbf{Logic}                & \textbf{Bench}  & \textbf{Samples} & \textbf{Proofs}       & \textbf{Elaborate} & \textbf{Translate} & \textbf{Check} \\ \hline
\multirow{4}{*}{\tt{LIA}}     & tptp            & 36               &  36 - 0 - 0           &  36 - 0             & 36 - 0 - 0          & 28 - 8 - 0       \\ \cline{2-7} 
                              & Ultimate        & 153              &  120 - 0 - 33         &  73 - 80            & 68 - 5 - 0          & 50 - 18 - 0      \\ \cline{2-7} 
                              & Svcomp'19      & 27               &  27 - 0 - 0           &  25 - 2             & 0 - 25 - 0          & 0                \\ \cline{2-7} 
                              & psyco           & 50               &  48 - 0 - 2           &  48 - 2             & 43 - 0 - 5          & 0 - 39 - 6       \\ \hline
\multirow{2}{*}{\tt{QFLIA}}   & SMPT            & 1568             &  1501 - 67 - 0        &  1491 - 32          & 1470 - 0 - 21       & 804 - 638 - 34   \\ \cline{2-7}
                              & rings           & 294              &  63 - 0 - 231         &  49 - 21            & 49 - 0 - 0          & 7 - 0 - 42       \\ \cline{2-7} 
                              & CAV\_2009       & 85               &  85 - 0 - 0           &  19 - 0             & 19 - 0 - 0          & 19 - 0 - 0       \\ \hline
\multirow{2}{*}{\tt{UFLIA}}   & sledgeh    & 1521             &  1343 - 0 - 178       &  1278 - 222         & 1258 - 13 - 7       & 713 - 467 - 80   \\ \cline{2-7} 
                              & tokeneer        & 1732             &  1732 - 0 - 0         & 1689 - 43           & 1689 - 0 - 0        & 1482 - 197 - 10  \\ \hline
\end{tabular}
\caption{Benchmark samples.}
\label{table:benchmarks-description}
\end{table}

Our benchmark suite, \cref{table:benchmarks-description} is composed of files from the SMT-LIB benchmarks\footnote{\url{https://smtlib.cs.uiowa.edu/benchmarks.shtml}}.
The suite includes a total of 5,466 samples drawn from 9 benchmarks spanning 3 SMT-LIB logics: \texttt{LIA}, \texttt{QFLIA}, and \texttt{UFLIA} that correspond to those covered by our method.
%These were selected because our previous work supported the \texttt{QF} and \texttt{UF} fragments.
Within the logics \texttt{QFLIA} and \texttt{UFLIA}, we prioritized benchmarks with the most significant number of available samples.
Table~\ref{table:benchmarks-description} provides a detailed breakdown of the benchmarks and corresponding results.
The \emph{Logic} column indicates the SMT theory used, while the \emph{Bench} column lists abbreviated benchmark names for conciseness.
The column \emph{Samples} describes the number of problems available with status \emph{unsat}. 
Each of the columns \emph{Proofs}, \emph{Translate}, and \emph{Check} reports a triple in the format \tt{success - error - timeout}, representing respectively the number of successful executions, failed attempts, and timeouts. 
The \emph{Proofs} column reports the number of proofs generated by cvc5 that do not contain \lstinline[language=SMT,basicstyle=\ttfamily\footnotesize]|Real| or \lstinline[language=SMT,basicstyle=\ttfamily\footnotesize]|to_real| cast operator.\todo[sm]{Should explain why (if not mentioned before).}
The \emph{Elaborate} column shows a pair of values: the number of proofs that were successfully elaborated by Carcara, and the number that failed. Only proofs successfully elaborated by Carcara are considered for translation.
The \emph{Translate} column gives the number of proof traces successfully translated into Lambdapi proofs, while the \emph{Check} column indicates the number of these translated proofs that were successfully type-checked by Lambdapi. 

We enforced a timeout of 30 seconds for cvc5 to find a proof and 30 seconds for the translation step with Carcara.
No timeout was imposed during the elaboration step, as the runtime is negligible.
A timeout of 20 seconds\todo[sm]{Justify timeout values, and in particular lower timeout for Lambdapi.} was set for Lambdapi when type-checking the final proofs.\footnote{All benchmarks were executed in parallel using GNU parallel \cite{tange_2025_15071920} with an Apple silicon M2.} % Ask by the tool to be cited

All nine benchmarks demonstrated consistently reliable proof generation, with few or no timeouts, except for the \emph{rings} benchmark.
The elaboration phase was generally robust, except in the \emph{Ultimate} benchmark, where an error in the elaborator caused failures.
For \texttt{LIA}, performance in the translation and checking stages was mixed. In particular, for \emph{Svcomp'19} and \emph{psyco}, no or few proofs could be translated or verified, mainly due to unsupported simplification rules involving the \texttt{ite} operator.
The \texttt{QFLIA} benchmarks exhibited more reliable proof checking in \emph{SMPT} and \emph{CAV\_2009},\todo[sm]{Why drop from 85 to 19 for CAV\_2009?} while only a small number of proofs from the \emph{rings} benchmark were successfully checked.
This limitation is due to the presence of \texttt{la\_generic} terms with hundreds of uninterpreted variables; the current normalization mechanism, described in \cref{ssec:normalization}, relies on a built-in term ordering that is not sufficiently efficient in this setting.
Benchmarks under \texttt{UFLIA} performed better throughout the pipeline. Most proofs were successfully generated and elaborated, and a large portion were translated and verified by Lambdapi, particularly in the \emph{tokeneer} dataset.\todo[sm]{Explain the high number of errors for SMPT, sledgehammer and tokeneer.}

\begin{table}[]
\centering
\begin{tabular}{|c|c|c|c|c|c|}
\hline
\textbf{Bench}      & \textbf{Min} & \textbf{Q1} & \textbf{Mean} & \textbf{Q3} & \textbf{Max} \\ \hline
tptp                & 240          & 262         & 263           & 267         & 336          \\ \hline
Ultimate            & 82           & 96          & 100           & 111         & 346          \\ \hline
SMPT                & 52           & 55          & 55            & 56          & 293          \\ \hline
rings               & 670          & 704         & 773           & 826         & 1197         \\ \hline
CAV\_2009           & 54           & 296         & 403           & 498         & 683          \\ \hline
sledgeh             & 53           & 166         & 354           & 552         & 1594         \\ \hline
tokeneer            & 55           & 60          & 61            & 248         & 753          \\ \hline
\end{tabular}
\caption{Lambdapi checking times in milliseconds.}
\label{table:benchmarks-list}
\end{table}

Table~\ref{table:benchmarks-list} reports the Lambdapi proof-checking times in milliseconds, presenting the minimum, first quartile (Q1), mean, third quartile (Q3), and maximum durations for each benchmark.
In most cases, the checking time remains below one second.
% , with the exception of a few outliers particularly in the \emph{sledgehammer} and \emph{rings} benchmarks that exhibit higher upper tails.


\section{Related work}
\label{sec:related}

The reconstruction for Alethe proofs produced by the veriT solver has been implemented within Isabelle/HOL to enhance the reconstruction success rate compared to Z3.
The authors employed Isabelle's \emph{linarith} procedure to replay arithmetic reasoning steps. This tactic serves as a decision procedure over real numbers, rather than integers or natural numbers, and is based on Fourier-Motzkin elimination.
However, the authors note that Isabelle's arithmetic procedure is incomplete and relatively inefficient for reconstructing arithmetic proofs; consequently, specific rules are reconstructed heuristically to improve performance.

Besson \cite{micromega} introduces a modular framework for integrating SMT solvers with the Coq\todo[sm]{Rocq? used in the abstract} proof assistant, emphasizing the generation and efficient verification of proof certificates directly within Coq.
His approach enables fast, reflexive proof checking using Nelson-Oppen-style theory combination, and it supports both Linear Integer Arithmetic (LIA) and Linear Arithmetic (LA) logics.
To efficiently generate Farkas certificates for LA, the technique relies on the Simplex algorithm, while for LIA, a variant of the Omega test \cite{omegatest} is adopted.
The technique has been evaluated on SMT-LIB 2 benchmarks using the Z3 solver.
\todo[sm]{I think you need to discuss how performances of the techniques compare, as far as they are comparable.}



\section{Conclusion}
\label{sec:conclusion}

\bibliographystyle{splncs04}
\todo[sm]{Fix capitalization in references.}
\bibliography{refs}


\appendix


\section{Alethe}
\label{app:alethe}


\subsection{The Syntax}

\begin{figure}[htb]%[H]
    \[
      \begin{array}{r c l}
     \grNT{proof}           &\grRule & \grNT{proof\_command}^{*} \\
     \grNT{proof\_command}  &\grRule & \textAlethe{(assume}\; \grNT{symbol}\; \grNT{proof\_term}\,\textAlethe{)} \\
                            &\grOr   & \textAlethe{(step}\; \grNT{symbol}\; \grNT{clause}
                                            \; \textAlethe{:rule}\; \grNT{symbol} \\
                            &        & \quad \grNT{premises\_annotation}^{?} \\
                            &        & \quad \grNT{context\_annotation}^{?}\;\grNT{attribute}^{*}\,\textAlethe{)} \\
                            & \grOr  & \textAlethe{(anchor :step}\; \grNT{symbol}\;
                                                \\
                            &        & \quad \grNT{args\_annotation}^{?}\;\grNT{attribute}^{*}\,\textAlethe{)} \\
                            & \grOr  & \textAlethe{(define-fun}\; \grNT{function\_def}\,\textAlethe{)} \\
     \grNT{clause}          &\grRule & \textAlethe{(cl}\; \grNT{proof\_term}^{*}\,\textAlethe{)} \\
     \grNT{proof\_term}     &\grRule & \grNT{term}\text{ extended with } \\
                            &        & \textAlethe{(choice (}\, \grNT{sorted\_var}\,\textAlethe{)}\; \grNT{proof\_term}\,\textAlethe{)}  \\
     \grNT{premises\_annotation} &\grRule & \textAlethe{:premises (}\; \grNT{symbol}^{+}\textAlethe{)} \\
     \grNT{args\_annotation}     &\grRule & \textAlethe{:args}\,\textAlethe{(}\,\grNT{step\_arg}^{+}\,\textAlethe{)}  \\
     \grNT{step\_arg}            &\grRule & \grNT{symbol} \grOr
                                              \textAlethe{(}\; \grNT{symbol}\; \grNT{proof\_term}\,\textAlethe{)} \\
     \grNT{context\_annotation}  &\grRule & \textAlethe{:args}\,\textAlethe{(}\,\grNT{context\_assignment}^{+}\,\textAlethe{)}  \\
     \grNT{context\_assignment}  &\grRule & \textAlethe{(}    \,\grNT{sorted\_var}\,\textAlethe{)}  \\
                                 & \grOr  & \textAlethe{(:=}\, \grNT{symbol}\;\grNT{proof\_term}\,\textAlethe{)} \\
      \end{array}
      \]
      \caption{Alethe grammar}
      \label{fig:grammar}
\end{figure}

%  Lambdapi Definitions for Integer and Binary Arithmetic
\section{Lambdapi Formalizations for Integer and Binary Number Operations}
\label{app:lambdapi-func-def}

\noindent
\begin{minipage}[t]{0.48\textwidth}
% \textbf{\ttfamily add : \bb{P} \ra \bb{P} \ra \bb{P}}

\begin{align*}
&\tt{add} : \bb{P} \ra \bb{P} \ra \bb{P} \\
& \tt{add}~(\tt{I}~x)~(\tt{I}~q) \re \tt{O}~(\tt{add\_c}~x~q) \\
& \tt{add}~(\tt{I}~x)~(\tt{O}~q) \re \tt{I}~(\tt{add}~x~q) \\
& \tt{add}~(\tt{O}~x)~(\tt{I}~q) \re \tt{I}~(\tt{add}~x~q) \\
& \tt{add}~(\tt{O}~x)~(\tt{O}~q) \re \tt{O}~(\tt{add}~x~q) \\
& \tt{add}~x~\tt{H} \re \tt{succ}~x \\
& \tt{add}~\tt{H}~y \re \tt{succ}~y \\
\end{align*}
\end{minipage}
\hfill
\begin{minipage}[t]{0.48\textwidth}
\begin{align*}
&\tt{add\_c} : \bb{P} \ra \bb{P} \ra \bb{P} \\
& \tt{add\_c}~(\tt{I}~x)~(\tt{I}~q) \re \tt{I}~(\tt{add\_c}~x~q) \\
& \tt{add\_c}~(\tt{I}~x)~(\tt{O}~q) \re \tt{O}~(\tt{add\_c}~x~q) \\
& \tt{add\_c}~(\tt{O}~x)~(\tt{I}~q) \re \tt{O}~(\tt{add\_c}~x~q) \\
& \tt{add\_c}~(\tt{O}~x)~(\tt{O}~q) \re \tt{I}~(\tt{add}~x~q) \\
& \tt{add\_c}~x~\tt{H} \re \tt{add}~x~(\tt{O}~\tt{H}) \\
& \tt{add\_c}~\tt{H}~y \re \tt{add}~(\tt{O}~\tt{H})~y \\
\end{align*}
\end{minipage}

\begin{align*}
&\tt{sub} : \bb{P} \ra \bb{P} \ra \bb{Z} \\
& \tt{sub}~(\tt{I}~p)~(\tt{I}~q) \re \tt{double}~(\tt{sub}~p~q) \\
& \tt{sub}~(\tt{I}~p)~(\tt{O}~q) \re \tt{succ\_double}~(\tt{sub}~p~q) \\
& \tt{sub}~(\tt{I}~p)~\tt{H} \re \ZPos (\tt{O}~p) \\
& \tt{sub}~(\tt{O}~p)~(\tt{I}~q) \re \tt{pred\_double}~(\tt{sub}~p~q) \\
& \tt{sub}~(\tt{O}~p)~(\tt{O}~q) \re \tt{double}~(\tt{sub}~p~q) \\
& \tt{sub}~(\tt{O}~p)~\tt{H} \re \ZPos (\tt{pos\_pred\_double}~p) \\
& \tt{sub}~\tt{H}~(\tt{I}~q) \re \ZNeg    (\tt{O}~q) \\
& \tt{sub}~\tt{H}~(\tt{O}~q) \re \ZNeg    (\tt{pos\_pred\_double}~q) \\
& \tt{sub}~\tt{H}~\tt{H} \re \tt{Z0} \\
\end{align*}

\begin{align*}
&\tt{compare\_acc} : \bb{P} \ra \tt{Comp} \ra \bb{P} \ra \tt{Comp} \\
& \tt{compare\_acc}~(\tt{I}~x)~c~(\tt{I}~q) \re \tt{compare\_acc}~x~c~q \\
& \tt{compare\_acc}~(\tt{I}~x)~\_~(\tt{O}~q) \re \tt{compare\_acc}~x~\tt{Gt}~q \\
& \tt{compare\_acc}~(\tt{I}~\_)~\_~\tt{H} \re \tt{Gt} \\
& \tt{compare\_acc}~(\tt{O}~x)~\_~(\tt{I}~q) \re \tt{compare\_acc}~x~\tt{Lt}~q \\
& \tt{compare\_acc}~(\tt{O}~x)~c~(\tt{O}~q) \re \tt{compare\_acc}~x~c~q \\
& \tt{compare\_acc}~(\tt{O}~\_)~\_~\tt{H} \re \tt{Gt} \\
& \tt{compare\_acc}~\tt{H}~\_~(\tt{I}~\_) \re \tt{Lt} \\
& \tt{compare\_acc}~\tt{H}~\_~(\tt{O}~\_) \re \tt{Lt} \\
& \tt{compare\_acc}~\tt{H}~c~\tt{H} \re c \\
\\
&\tt{compare}~x~y \coloneq \tt{compare\_acc}~x~\tt{Eq}~y \\
\end{align*}
  
\begin{minipage}[t]{0.45\textwidth}
\begin{align*}
&\tt{mul} : \bb{P} \ra \bb{P} \ra \bb{P} \\
& \tt{mul}~(\tt{I}~x)~y \re \tt{add}~y~(\tt{O}~(\tt{mul}~x~y)) \\
& \tt{mul}~(\tt{O}~x)~y \re \tt{O}~(\tt{mul}~x~y) \\
& \tt{mul}~\tt{H}~y \re y \\
& \tt{mul}~y~\tt{H} \re y \\
\end{align*}
\end{minipage}
\begin{minipage}[t]{0.48\textwidth}
\begin{align*}
&\tt{*} : \bb{Z} \ra \bb{Z} \ra \bb{Z} \\
& \tt{Z0} *~\_ \re \tt{Z0} \\
& \_ *~\tt{Z0} \re \tt{Z0} \\
& \ZPos x *~\ZPos y \re \ZPos (\tt{mul}~x~y) \\
& \ZPos x *~\ZNeg    y \re \ZNeg    (\tt{mul}~x~y) \\
& \ZNeg    x *~\ZPos y \re \ZNeg    (\tt{mul}~x~y) \\
& \ZNeg    x *~\ZNeg    y \re \ZPos (\tt{mul}~x~y) \\
\end{align*}
\end{minipage}

\subsection{Confluence of the rewriting rules of integers and positive binary number}
\label{app:confluence-int-pos}

The rules presented below represent the relations $\ra_\bb{Z}$ and $\ra_\bb{P}$ encoded in the TRS\footnote{\url{http://www.lri.fr/~marche/tpdb/format.html}} format accepted by the \cite{CSI} tool.
These rules can be used to rerun the tool in order to verify the confluence property.

\begin{lstlisting}[language=trs, caption=Rewriting rule of $\bb{Z}$ and $\bb{P}$ in the TRS format]
(VAR
  a: Z
  b: Z
  x : P
  q : P
  y : P
)
(RULES
  ~(Z0) -> Z0
  ~(Zpos(p)) -> Zneg(p)
  ~(Zneg(p)) -> Zpos(p)
  ~(~(a)) -> a

  double(Z0) -> Z0
  double(Zpos(p)) -> Zpos(O(p))
  double(Zneg(p)) -> Zneg(O(p))
  
  succ_double(Z0) -> Zpos(H)
  succ_double(Zpos(p)) -> Zpos(I(p))
  succ_double(Zneg(p)) -> Zneg(pos_pred_double(p))
  
  pred_double(Z0) -> Zneg(H)
  pred_double(Zpos(p)) -> Zpos(pos_pred_double(p))
  pred_double(Zneg(p)) -> Zneg(I(p))

  sub(I(p), I(q)) -> double(sub(p, q))
  sub(I(p), O(q)) -> succ_double(sub(p, q))
  sub(I(p), H) -> Zpos(O(p))
  sub(O(p), I(q)) -> pred_double(sub(p, q))
  sub(O(p), O(q)) -> double(sub(p, q))
  sub(O(p), H) -> Zpos(pos_pred_double(p))
  sub(H, I(q)) -> Zneg(O(q))
  sub(H, O(q)) -> Zneg(pos_pred_double(q))
  sub(H, H) -> Z0

  +(Z0,a) -> a
  +(a,Z0) -> a
  +(Zpos(x), Zpos(y)) -> Zpos(add(x, y))
  +(Zpos(x), Zneg(y)) -> sub(x, y)
  +(Zneg(x), Zpos(y)) -> sub(y, x)
  +(Zneg(x), Zneg(y)) -> Zneg(add(x, y))
  
  mult(Z0, a) -> Z0
  mult(a, Z0) -> Z0
  mult(Zpos(x), Zpos(y)) -> Zpos(mul(x, y))
  mult(Zpos(x), Zneg(y)) -> Zneg(mul(x, y))
  mult(Zneg(x), Zpos(y)) -> Zneg(mul(x, y))
  mult(Zneg(x), Zneg(y)) -> Zpos(mul(x, y))


  succ(I(x)) -> O(succ(x))
  succ(O(x)) -> I(x)
  succ(H) -> O(H)
  add(I(x), I(q)) -> O(addcarry(x, q))
  add(I(x), O(q)) -> I(add(x, q))
  add(O(x), I(q)) -> I(add(x, q))
  add(O(x), O(q)) -> O(add(x, q))
  add(x, H) -> succ(x)
  add(H, y) -> succ(y)

  addcarry(I(x), I(q)) -> I(addcarry(x, q))
  addcarry(I(x), O(q)) -> O(addcarry(x, q))
  addcarry(O(x), I(q)) -> O(addcarry(x, q))
  addcarry(O(x), O(q)) -> I(add(x, q))
  addcarry(x, H) -> add(x, O(H))
  addcarry(H, y) -> add(O(H), y)
  
  pos_pred_double(I(x)) -> I(O(x))
  pos_pred_double(O(x)) -> I(pos_pred_double(x))
  pos_pred_double(H) -> H
  
  mul(I(x), y) -> add(x, O(mul(x,y)))
  mul(O(x), y) -> O(mul(x, y))
  mul(H, y) -> y
)
\end{lstlisting}


% Proof:
%  Church Rosser Transformation Processor (no redundant rules):
%   strict:
   
%   weak:
   
% critical peaks: 4
%   add(O(H()),H()) <-6|[]- addcarry(H(),H()) -7|[]-> add(H(),O(H()))
%   add(H(),O(H())) <-7|[]- addcarry(H(),H()) -6|[]-> add(O(H()),H())
%   succ(H()) <-12|[]- add(H(),H()) -13|[]-> succ(H())
%   succ(H()) <-13|[]- add(H(),H()) -12|[]-> succ(H())

%   Church Rosser Transformation Processor (critical pair closing system, Thm 2.4):
%   add(x,H()) -> succ(x)
%   add(H(),y) -> succ(y)
%   critical peaks: joinable
%   Matrix Interpretation Processor: dim=1
    
%     interpretation:
%     [succ](x0) = 4x0,
    
%     [add](x0, x1) = 4x0 + 4x1 + 2,
    
%     [H] = 4
%     orientation:
%     add(x,H()) = 4x + 18 >= 4x = succ(x)
    
%     add(H(),y) = 4y + 18 >= 4y = succ(y)
%     problem:
    
%     Qed
\begin{lemma}[Confluence]
\begin{proof}
CSI automatically proves the confluence of $\ra_\bb{Z}$ and $\ra_\bb{P}$ by giving the polynomial interpretation:
\begin{align*}
[\tt{succ}(x)] = 4*x & &[\tt{add}(x, y)] = 4 * x + 4 * y + 2  & &[ {\tt{H}} ] = 4 \\
\end{align*}
\end{proof} 
\end{lemma}

\end{document}
